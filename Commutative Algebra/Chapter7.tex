\documentclass{solution}

\begin{document}

\begin{problem}
    Let $A$ be a non-Noetherian ring and let $\Sigma$ be the set of ideals in $A$ which are not finitely generated. Show that $\Sigma$ has maximal elements and that the maximal elements of $\Sigma$ are prime ideals.

    Hence a ring in which every prime ideal is finitely generated is Noetherian (I.S. Cohen)
\end{problem}

\begin{proof}
    The existence of maximal elements in $\Sigma$ is standard Zorn's lemma argument and is thus omitted. Now let $I$ be a maximal element in $\Sigma$, suppose it is not a prime, then there are $x, y \notin I$ such that $xy \in I$. Consider the ideal $I + (x)$. It is strictly larger than $I$, so by maximality of $I$, $I + (x)$ is finitely generated. Suppose it is generated by $\left\lbrace z_i + a_ix \right\rbrace_{i = 1}^n$ where $z_i \in I, a_i \in A$. Then take any element $z \in I \subset I + (x)$, we can write it as:
    \begin{equation}\label{eq:prob1}
        z = \sum\limits_{i = 1}^{n} b_i(z_i + a_ix) = \sum\limits_{i = 1}^{n} b_iz_i + \sum\limits_{i = 1}^{n} a_ib_ix
    \end{equation}
    where $b_i \in A$. Note that $\sum\limits_{i = 1}^{n} a_ib_i x = z - \sum\limits_{i = 1}^{n} b_i z_i \in (x) \cap I$, we must have $\sum\limits_{i = 1}^{n} a_ib_i \in (I : x)$. As a result, Eq \ref{eq:prob1} implies that $I = I_0 + x(I : x)$ where $I_0 = (z_1, \cdots, z_n)$ (Actually Eq \ref{eq:prob1} shows "$\subset$" but the other direction is trivial). Since $(I : x)$ is strictly larger than $I$ ($y \in (I : x) \setminus I$), by maximality of $I$ it is finitely generated by some $w_1, \cdots, w_m$. It follows that $I$ is generated by $z_1, \cdots, z_n, xw_1, \cdots, xw_m$, a contradiction to $I \in \Sigma$.

    The above has shown that any non-Noetherian ring must have a prime ideal that is not finitely generated. The converse shows that any ring whose prime ideals are all finitely generated must be Noetherian.
\end{proof}

\begin{problem}
    Let $A$ be a Noetherian ring and let $f = \sum\limits_{n = 0}^{\infty} a_n X^n \in A[[X]]$. Prove that $f$ is nilpotent if and only if each $a_n$ is nilpotent.
\end{problem}

\begin{proof}
    The "only if" part is by Problem 5 in Chapter 1. For the "if" part, note that by Corollary 7.15, $\mathfrak{N}_A^n = 0$ for some $n \gt 0$. Since the coefficients of $f$ are all in $\mathfrak{N}_A$, the coefficients of $f^n$ are all in $\mathfrak{N}_A^n = 0$, namely $f^n = 0$ and $f$ is nilpotent.
\end{proof}

\begin{problem}
    Let $I$ be an irreducible ideal in a ring $A$, Then the followings are equivalent:
    \begin{enumerate}
        \item $I$ is primary.
        \item For every multiplicatively closed subset $S$ of $A$ we have $(S ^{-1}I)^c = (I : x)$ for some $x \in S$
        \item The sequence $(I : x^n)$ stabilizes, for every $x \in A$
    \end{enumerate}
\end{problem}

\begin{proof}
    As before, denote $S(I) = (S ^{-1} I)^c$.

    \begin{enumerate}
        \item (1) $\Rightarrow$ (2): Suppose $I$ is $\mathfrak{p}$-primary. By Proposition 4.8, there are two possibilities:
        \begin{enumerate}
            \item $\mathfrak{p} \cap S \ne \emptyset$, then $S(I) = A$. Pick $x \in S \cap \mathfrak{p}$, we have $x^n \in S \cap I$ for some $n \gt 0$ as $\mathfrak{p} = \sqrt{I}$ and $S$ is multiplicatively closed. Then $(I : x^n) = A$
            \item $\mathfrak{p} \cap S = \emptyset$, then $S(I) = I$. Pick arbitrary $x \in S$, then $x \notin \mathfrak{p}$ and thus $(I : x) = I$ by Proposition 4.4
        \end{enumerate}
        (Note that for this part we do not appeal to the fact that $I$ is irreducible)
        \item (2) $\Rightarrow$ (3): Take arbitrary ideal $I$ and $x \in A$, note that by Proposition 3.11, $S(I) = I^{ec} = \bigcup\limits_{n} (I : x^n)$. By our hypothesis, $S(I) = (I : x^k)$ for some $k$, it follows that the chain stabilizes. (Note that for this part we still do not appeal to the fact that $I$ is irreducible.)
        \item (3) $\Rightarrow$ (1): This is by the proof of Lemma 7.12, but we repeat it here. Replace $A$ by $A / I$, ISTS if the chain $\mathrm{Ann}(x) \subset \mathrm{Ann}(x^2) \subset \cdots$ stabilizes for any $x \in A$, and the ideal $0$ is irreducible, then every zero-divisor is nilpotent. Take $x \in A$ an arbitrary zero-divisor, suppose $xy = 0$ for some $y \ne 0$. By our hypothesis, the chain $\mathrm{Ann}(x) \subset \mathrm{Ann}(x^2) \subset \cdots$ stabilizes at some $\mathrm{Ann}(x^n)$. Then if $z x^{n + 1} = 0$, we must have $zx^n = 0$, namely $(x^n) \cap \mathrm{Ann}(x) = 0$. As $0$ is irreducible, either $x^n = 0$ or $\mathrm{Ann}(x) = 0$, but $\mathrm{Ann}(x) \ne 0$ as $y \in \mathrm{Ann}(x)$. So $x$ is nilpotent, which completes the proof.
    \end{enumerate}
\end{proof}

\begin{problem}
    Which of the following rings are Noetherian?
    \begin{enumerate}
        \item The ring of rational functions of $z$ having no pole on the circle $\left\lvert z \right\rvert = 1$.
        \item The ring of power series in $z$ with a positive radius of convergence.
        \item The ring of power series in $z$ with an infinite radius of convergence.
        \item The ring of polynomials in $z$ whose first $k$ derivatives vanish at the origin ($k$ being a fixed integer).
        \item The ring of polynomials in $z, w$ all of whose partial derivatives with respect to $w$ vanish for $z = 0$.
    \end{enumerate}
    In all cases the coefficients are complex numbers.
\end{problem}

\begin{proof}
    I omit the verifications that the sets are indeed rings.

    \TODO
\end{proof}

\begin{problem}
    Let $A$ be a Noetherian ring, $B$ a finitely generated $A$-algebra, $G$ a finite group of $A$-automorphisms of $B$, and $B^G$ the set of all elements of $B$ which are left fixed by every element of $G$. Show that $B^G$ is a finitely generated $A$-algebra
\end{problem}

\begin{proof}
    By Problem 12 of Chapter 5, $B$ is integral over $B^G$, then apply Proposition 7.8. (We may replace $A$ by its homomorphic image)
\end{proof}

\begin{problem}
    If a finitely generated ring $K$ is a field, it is a finite field.
\end{problem}

\begin{proof}
    Follow the hint.
    
    If $K$ has characteristic $0$, then the characteristic map $\mathbb{Z} \rightarrow K$ is injective. So we may regard $\mathbb{Z} \subset \mathbb{Q} \subset K$. If $K$ is finitely generated over $\mathbb{Z}$, it is finitely generated over $\mathbb{Q}$, by Proposition 7.9, it is finite over $\mathbb{Q}$. Then by Proposition 7.8, $\mathbb{Q}$ is finitely generated over $\mathbb{Z}$, which is absurd. (Contradict similarly as in Proposition 7.9, or see the lemma below)

    If $K$ has characteristic $p \gt 0$, then the characteristic map induces $\mathbb{Z} / (p) \hookrightarrow K$. It follows that $K$ is a finitely generated $\mathbb{Z} / (p)$ algebra. By Proposition 7.9, $K$ is finite over $\mathbb{Z} / (p)$, namely $K = (\mathbb{Z} / (p))^n / I$ for some ideal $I$, which is finite.
\end{proof}

\begin{problem}
    Let $X$ be an affine algebraic variety given by a family of equations $F_{\lambda}(X_1, \cdots, X_n) = 0(\lambda \in \Lambda)$(Chapter 1, Problem 27). Show that there exists a finite subset $\Lambda_0$ of $\Lambda$ such that $X$ is given by the equations $F_{\lambda}(X_1, \cdots, X_n) = 0$ for $\lambda \in \Lambda_0$
\end{problem}

\begin{proof}
    Let $I = (F_{\lambda})_{\lambda \in \Lambda}$ be the ideal defining $X$. Note that $k$ is a field and thus Noetherian, so $k[X_1, \cdots, X_n]$ is also Noetherian by Hilbert Basis Theorem. It follows that $I$ is finitely generated by $\left\lbrace F_{\lambda} \right\rbrace_{\lambda \in \Lambda_0}$ where $\Lambda_0 \subset \Lambda$ finite, which completes the proof.
\end{proof}

{\color{red} Problem 7 implies that $\mathrm{MSpec}(k[X_1, \cdots, X_n])$ is compact. But I am not sure whether the converse implication holds. \TODO}

\begin{problem}
    If $A[X]$ is Noetherian, is $A$ necessarily Noetherian?
\end{problem}

\begin{proof}
    Consider the homomorphism $A[X] \rightarrow A$ defined by $X \mapsto 0, a \mapsto a$ for $a \in A$. Then it is surjective. By Proposition 7.1, $A$ is Noetherian.
\end{proof}

\begin{problem}
    Let $A$ be a ring such that:
    \begin{enumerate}
        \item for each maximal ideal $\mathfrak{m}$ of $A$, the local ring $A_{\mathfrak{m}}$ is Noetherian;
        \item for each $x \ne 0$ in $A$, the set of maximal ideals of $A$ which contain $x$ is finite.
    \end{enumerate}
    Show that $A$ is Noetherian.
\end{problem}

\begin{proof}
    Follow the hint. Let $I$ be an arbitrary ideal, by Proposition 3.8 and Proposition 3.11, part 5, it suffices to find elements $x_1, \cdots, x_n \in I$ such that $I_{\mathfrak{m}} = (x_1, \cdots, x_n)_{\mathfrak{m}}$ for arbitrary maximal ideal $\mathfrak{m}$. We achieve this in two steps:
    \begin{enumerate}
        \item The maximal ideals that contain $I$: By condition 2, the maximal ideals that contain $I$ is finite, denote them as $\mathfrak{m}_1, \cdots, \mathfrak{m}_r$. For each $\mathfrak{m}_i$, since $A_{\mathfrak{m}_i}$ is Noetherian, there are $x_{i, 1}, \cdots, x_{i, n_i} \in A$ such that their images generate the extension of $I$ in $A_{\mathfrak{m}_i}$. Then we have $(\left\lbrace x_{i, j} \right\rbrace_{i, j})_{\mathfrak{m}} = I_{\mathfrak{m}}$ for arbitrary $I \subset \mathfrak{m}$.
        \item The maximal ideals that do not contain $I$: Note that if $I \not \subset \mathfrak{m}$, then $I_{\mathfrak{m}} = (1)$. So it suffices to find a finite set of $x_i$'s such that $(\left\lbrace x_i \right\rbrace_i)_{\mathfrak{m}} = (1)$ whenever $I \not \subset \mathfrak{m}$. Pick any $x_0 \in I$, by condition 2, there are only finitely many maximal ideals that contain $x_0$, let $\mathfrak{m}_1, \cdots, \mathfrak{m}_r$ be those maximal ideals that contain $x_0$ but not contain $I$. For each $\mathfrak{m}_i$, pick $x_i \in I$ such that $x_i \notin \mathfrak{m}_i$. We claim that $(x_0, x_1, \cdots, x_r)_{\mathfrak{m}} = (1)$ for arbitrary $I \not \subset \mathfrak{m}$: For each $I \not \subset \mathfrak{m}$, if $x_0 \notin \mathfrak{m}$, then $(x_0, x_1, \cdots, x_r)_{\mathfrak{m}} \supset (x_0)_{\mathfrak{m}} = (1)$; Otherwise, $\mathfrak{m}$ must be one of $\mathfrak{m}_i$ and thus $x_i \notin \mathfrak{m} \Rightarrow (x_0, x_1, \cdots, x_r)_{\mathfrak{m}} \supset (x_i)_{\mathfrak{m}} = (1)$.
    \end{enumerate}
    Merging the two sets of generators will do the job.
\end{proof}

\begin{problem}
    Let $M$ be a Noetherian $A$-module. Show that $M[X]$ (Chapter 2, Problem 6) is a Noetherian $A[X]$-module.
\end{problem}

\begin{proof}
    The proof is basically the same as the proof of Hilbert Basis Theorem.
\end{proof}

\begin{problem}
    Let $A$ be a ring such that each local ring $A_{\mathfrak{p}}$ is Noetherian. Is $A$ necessarily Noetherian?
\end{problem}

\begin{proof}
    We can use the same counter-example as in Problem 12 of Chapter 6. Let $A = \prod\limits_{i = 1}^{\infty} \mathbb{F}_2$, then the prime ideals of $A$ are $\prod\limits_{i = 1}^{n - 1} \mathbb{F}_2 \times 0 \times \prod\limits_{i = n + 1}^{\infty} \mathbb{F}_2$, and $A_{\mathfrak{p}}$ contains only one prime ideal. So $A_{\mathfrak{p}}$ are all Noetherian, but $A$ is not Noetherian (Actually Problem 12 shows that even $\mathrm{Spec}(A)$ is not Noetherian)
\end{proof}

{\color{red} Problem 11 shows that Noetherian is not a local property.}

\begin{problem}
    Let $A$ be a ring and $B$ a faithfully flat $A$-algebra(Chapter 3, Problem 16). If $B$ is Noetherian, show that $A$ is Noetherian.
\end{problem}

\begin{proof}
    Let $I_1 \subset I_2 \subset \cdots$ be an ascending chain of ideals in $A$, then $I_1^e \subset I_2^e \subset \cdots$ is an ascending chain of ideals in $B$. Since $B$ is Noetherian, the chain stabilizes at some $n$. By Problem 16 of Chapter 3, $I_m = I_m^{ec} = I_n^{ec} = I_n$ for arbitrary $m \ge n$, namely the chain $I_1 \subset I_2 \subset \cdots$ stabilizes.
\end{proof}

\begin{problem}
    Let $f: A \rightarrow B$ be a ring homomorphism of finite type. Show that the fibers of $\mathrm{Spec}(f)$ are Noetherian subspaces of $B$.
\end{problem}

{\color{red} I assume by 'Noetherian subspace', the author means a subset isomorphic to a Noetherian ring.}

\begin{proof}
    By Problem 21 in Chapter 3, take arbitrary $\mathfrak{p} \in \mathrm{Spec}(A)$, the fiber of $\mathfrak{p}$ is isomorphic to $B_{\mathfrak{p}} / \mathfrak{p} B_{\mathfrak{p}}$. Since $B$ is of finite type over $A$, $B_{\mathfrak{p}}$ is of finite type over $A_{\mathfrak{p}}$, and furthermore $B_{\mathfrak{p}} / \mathfrak{p} B_{\mathfrak{p}}$ is of finite type over $A_{\mathfrak{p}} / \mathfrak{p} A_{\mathfrak{p}} \cong k(\mathfrak{p})$, a field and hence Noetherian. By Corollary 7.7, $B_{\mathfrak{p}} / \mathfrak{p} B_{\mathfrak{p}}$ is Noetherian.
\end{proof}

\begin{problem}
    (Hilbert's Nullstellensatz, strong form) Let $k$ be an algebraically closed field, let $A$ denote the polynomial ring $k[X_1, \cdots, X_n]$ and let $I$ be an ideal in $A$. Let $V$ be the variety in $k^n$ defined by the ideal $I$, so that $V$ is the set of all $x = (x_1, \cdots, x_n) \in k^n$ such that $f(x) = 0$ for all $f \in I$. Let $I(V)$ be the ideal of $V$, i.e. the ideal of all polynomials $g \in A$ such that $g(x) = 0$ for all $x \in V$. Then $I(V) = \sqrt{I}$
\end{problem}

\begin{proof}
    I've seen a lot of proof of Hilbert's Nullstellensatz. So for this problem I will use a generalized proof appealing to previous problems about Jacobson ring. According to Problem 24 of Chapter 5, $A = k[X_1, \cdots, X_n]$ is a Jacobson ring. It follows that for any ideal $I$ of $A$, we have
    $$\sqrt{I} = \bigcap\limits_{I \subset \mathfrak{m} \in \mathrm{MSpec}(A) } \mathfrak{m}$$
    We know that the maximal ideals in $A$ are in a one-to-one correspondence with points in $k^n$ (see the proof of Problem 19 in Chapter 5). Under this correspondence, for ideal $I$ of $A$, we have
    \begin{equation} \label{eq:prob14}
        V(I) \leftrightarrow \left\lbrace \mathfrak{m} \in \mathrm{MSpec}(A): I \subset \mathfrak{m} \right\rbrace
    \end{equation}
    And:
    $$I(V(I)) = \left\lbrace f \in A: f \in \mathfrak{m}, \forall \mathfrak{m} \in V(I) \right\rbrace = \bigcap\limits_{I \subset \mathfrak{m} \in \mathrm{MSpec}(A)} \mathfrak{m} = \sqrt{I}$$
    where $V(I)$ denotes the corresponding sets in Eq \ref{eq:prob14}. This completes the proof.
\end{proof}

\begin{problem}
    Let $A$ be a Noetherian local ring, $\mathfrak{m}$ its maximal ideal and $k$ its residue field, and let $M$ be a finitely generated $A$-module. Then the followings are equivalent:
    \begin{enumerate}
        \item $M$ is free;
        \item $M$ is flat;
        \item the mapping of $\mathfrak{m} \otimes M$ into $A \otimes M$ is injective;
        \item $\mathrm{Tor}_{1}^{A}(k, M) = 0$
    \end{enumerate}
\end{problem}

\begin{proof}
    \begin{enumerate}
        \item (1) $\Rightarrow$ (2): It does not require any hypothesis that free modules are flat: By part 3 and 4 of Proposition 2.14.
        \item (2) $\Rightarrow$ (3): Note that $\mathfrak{m} \hookrightarrow A$ is an injective $A$-module homomorphism, then apply flatness.
        \item (3) $\Rightarrow$ (4): Consider the exact sequence:
        $$0 \rightarrow \mathfrak{m} \rightarrow A \rightarrow k \rightarrow 0$$
        It induces the Tor exact sequence:
        $$\mathrm{Tor}_{1}^{A}(k, M) \rightarrow \mathfrak{m} \otimes_A M \rightarrow A \otimes_A M \rightarrow k \otimes_A M \rightarrow 0$$
        By our hypothesis, $\mathfrak{m} \otimes_A M \rightarrow A \otimes_A M$ is injective so $\mathrm{Tor}_{1}^{A}(k, M) = 0$
        \item (4) $\Rightarrow$ (1): Follow the hint. Note that $k \otimes M \cong M / \mathfrak{m} M$ is a finitely dimensional $k$-vector space, pick $x_1, \cdots, x_n \in M$ such that their images in $M / \mathfrak{m} M$ form a basis. Then take $F = A^n$ and $\varphi: F \rightarrow M: e_i \mapsto x_i$. Let $E = \mathrm{ker}(\varphi)$. Then $0 \rightarrow E \rightarrow F \rightarrow M \rightarrow 0$ will be an exact sequence, which induces the Tor exact sequence by our hypothesis:
        $$0 = \mathrm{Tor}_{1}^{A}(k, M) \rightarrow k \otimes_A E \rightarrow k \otimes_A F \rightarrow k \otimes_A M \rightarrow 0$$
        Note that $\mathds{1}_k \otimes \varphi: k \otimes_A F \rightarrow k \otimes_A M$ is a surjective linear mapping of $k$-vector spaces of the same dimension (Note that $k \otimes_A A^n \cong k^n$), so it must be an isomorphism, namely $k \otimes_A E = 0$. Since $E$ is a submodule of $F = A^n$, which is Noetherian by Corollary 6.4, $E$ must be finitely generated. It follows that $E / \mathfrak{m}E = 0 \Rightarrow E = 0$ by Nakayama's lemma.
    \end{enumerate}
\end{proof}

\begin{problem}
    Let $A$ be a Noetherian ring, $M$ a finitely generated $A$-module. Then the followings are equivalent:
    \begin{enumerate}
        \item $M$ is a flat $A$-module;
        \item $M_{\mathfrak{p}}$ is a free $A_{\mathfrak{p}}$-module, for all prime ideals $\mathfrak{p}$
        \item $M_{\mathfrak{m}}$ is a free $A_{\mathfrak{m}}$-module, for all maximal ideals $\mathfrak{m}$
    \end{enumerate}
    In other words, flat = locally free.
\end{problem}

\begin{proof}
    \begin{enumerate}
        \item (1) $\Rightarrow$ (2): Note that $M_{\mathfrak{p}}$ is a flat $A_{\mathfrak{p}}$-module by Proposition 3.10, and $A_{\mathfrak{p}}$ is Noetherian local, then apply Problem 15 to show that $M_{\mathfrak{p}}$ is free.
        \item (2) $\Rightarrow$ (3): Clear
        \item (3) $\Rightarrow$ (1): It follows from Proposition 3.10 and the fact that free modules are flat.
    \end{enumerate}
\end{proof}

\begin{problem}
    Let $A$ be a ring and $M$ a Noetherian $A$-module. Show (by imitating the proofs of (7.11) and (7.12)) that every submodule $N$ of $M$ has a primary decomposition. (Chapter 4, Problems 20 - 23)
\end{problem}

\begin{proof}
    As in Lemma 7.11 and 7.12, we split the proof into two halves.

    \begin{enumerate}
        \item Every submodule of $M$ can be written as a finite intersection of irreducible submodules of $M$: By irreducible submodule, we mean a submodule $N$ of $M$ such that $N = N_1 \cap N_2$ implies $N_1 = N$ or $N_2 = N$. Suppose otherwise, the set of submodules that cannot be written as a finite intersection of irreducible submodules is non-empty. Then by Noetherian, it must contain a maximal element $N$. But then $N$ is reducible (if it is irreducible, it can be written an intersection of irreducible submodules, namely itself), suppose $N = N_1 \cap N_2$ for some $N_1, N_2 \ne N$. As $N_1, N_2$ are strictly larger than $N$, by our selection of $N$, they can be written as a finite intersection of irreducible submodules. It follows that $N$ is a finite intersection of irreducible submodules, a contradiction.
        \item Every irreducible submodule is primary: Take $Q$ an irreducible submodule of $M$. We may replace $M$ by $M / Q$ and suppose $0$ is irreducible. Then ISTS any zero-divisor in $M$ is nilpotent. Let $x$ be any zero-divisor of $M$, define $\varphi_x: m \mapsto xm$. Then consider the $\mathrm{ker}(\varphi_{x^n})$. It is clear that $\mathrm{ker}(\varphi_{x^n}) \subset \mathrm{ker}(\varphi_{x^{n + 1}})$. By Noetherian, the chain $\mathrm{ker}(\varphi_{x}) \subset \mathrm{ker}(\varphi_{x^2}) \subset \cdots$ stabilizes at some $n$. Then take any $m \in M$, if $x^{n + 1} m = 0$, then $x^n m = 0$, namely $x^nM \cap \mathrm{ker}(\varphi_x) = 0$. By definition of zero-divisor, $\mathrm{ker}(\varphi_x) \ne 0$ as $x$, since $0$ is irreducible, this implies $x^nM = 0$, namely $x$ is nilpotent \textbf{in $M$}.
    \end{enumerate}
\end{proof}

\begin{problem}
    Let $A$ be a Noetherian ring, $\mathfrak{p}$ a prime ideal of $A$, and $M$ a finitely generated $A$-module. Show that the followings are equivalent:
    \begin{enumerate}
        \item $\mathfrak{p}$ belongs to $0$ in $M$;
        \item there exists $x \in M$ such that $\mathrm{Ann}(x) = \mathfrak{p}$
        \item there exists a submodule of $M$ isomorphic to $A / \mathfrak{p}$
    \end{enumerate}
    Deduce that there exists a chain of submodules:
    $$0 = M_0 \subset M_1 \subset \cdots M_r = M$$
    such that each quotient $M_i / M_{i - 1}$ is of the form $A / \mathfrak{p}_i$, where $\mathfrak{p}_i$ is a prime ideal of $A$
\end{problem}

\begin{proof}
    \begin{enumerate}
        \item (1) $\Rightarrow$ (2): Note that by Problem 20 - 23 in Chapter 4, we have \TODO
    \end{enumerate}
\end{proof}

\begin{problem}
    Let $I$ be an ideal in a Noetherian ring $A$. Let
    $$I = \bigcap\limits_{i = 1}^{r} J_i = \bigcap\limits_{j = 1}^{s} J_j'$$
    be two minimal decomposition of $I$ as intersections of irreducible ideals. Prove that $r = s$ and that (possible after re-indexing the $J_j'$) $r(J_i) = r(J'_i)$ for all $i$.

    State and prove an analogous result for modules.
\end{problem}

\begin{proof}
    {\color{red} A little notice before the proof: It is true that irreducible ideals in a Noetherian ring are primary. But a minimal decomposition of $I$ as intersections of irreducible ideals is not necessarily a minimal prime decomposition: It only requires that no ideals in the decomposition contain the intersection of all other ideals, it does not require that they have distinct radicals. This is because the intersection of $\mathfrak{p}$-primary ideals are $\mathfrak{p}$-primary, but the intersection of irreducible ideals are not necessarily irreducible, so we cannot impose the constraint of radical ideals on irreducible decomposition.}

    Now for the proof. Pick any $i_0$ from $1$ to $r$, denote $\tilde{I}_{i_0} = \bigcap\limits_{i \ne i_0}$. Since $I_{i_0}$ is irreducible, $I_{i_0} \cap \tilde{I}_{i_0}$ is irreducible (as $A$-module) in the 
\end{proof}

\end{document}