\documentclass{solution}

\begin{document}

\begin{problem}
    Let $A$ be a non-Noetherian ring and let $\Sigma$ be the set of ideals in $A$ which are not finitely generated. Show that $\Sigma$ has maximal elements and that the maximal elements of $\Sigma$ are prime ideals.

    Hence a ring in which every prime ideal is finitely generated is Noetherian (I.S. Cohen)
\end{problem}

\begin{proof}
    The existence of maximal elements in $\Sigma$ is standard Zorn's lemma argument and is thus omitted. Now let $I$ be a maximal element in $\Sigma$, suppose it is not a prime, then there are $x, y \notin I$ such that $xy \in I$. Consider the ideal $I + (x)$. It is strictly larger than $I$, so by maximality of $I$, $I + (x)$ is finitely generated. Suppose it is generated by $\left\lbrace z_i + a_ix \right\rbrace_{i = 1}^n$ where $z_i \in I, a_i \in A$. Then take any element $z \in I \subset I + (x)$, we can write it as:
    \begin{equation}\label{eq:prob1}
        z = \sum\limits_{i = 1}^{n} b_i(z_i + a_ix) = \sum\limits_{i = 1}^{n} b_iz_i + \sum\limits_{i = 1}^{n} a_ib_ix
    \end{equation}
    where $b_i \in A$. Note that $\sum\limits_{i = 1}^{n} a_ib_i x = z - \sum\limits_{i = 1}^{n} b_i z_i \in (x) \cap I$, we must have $\sum\limits_{i = 1}^{n} a_ib_i \in (I : x)$. As a result, Eq \ref{eq:prob1} implies that $I = I_0 + x(I : x)$ where $I_0 = (z_1, \cdots, z_n)$ (Actually Eq \ref{eq:prob1} shows "$\subset$" but the other direction is trivial). Since $(I : x)$ is strictly larger than $I$ ($y \in (I : x) \setminus I$), by maximality of $I$ it is finitely generated by some $w_1, \cdots, w_m$. It follows that $I$ is generated by $z_1, \cdots, z_n, xw_1, \cdots, xw_m$, a contradiction to $I \in \Sigma$.

    The above has shown that any non-Noetherian ring must have a prime ideal that is not finitely generated. The converse shows that any ring whose prime ideals are all finitely generated must be Noetherian.
\end{proof}

\begin{problem}
    Let $A$ be a Noetherian ring and let $f = \sum\limits_{n = 0}^{\infty} a_n X^n \in A[[X]]$. Prove that $f$ is nilpotent if and only if each $a_n$ is nilpotent.
\end{problem}

\begin{proof}
    The "only if" part is by Problem 5 in Chapter 1. For the "if" part, note that by Corollary 7.15, $\mathfrak{N}_A^n = 0$ for some $n \gt 0$. Since the coefficients of $f$ are all in $\mathfrak{N}_A$, the coefficients of $f^n$ are all in $\mathfrak{N}_A^n = 0$, namely $f^n = 0$ and $f$ is nilpotent.
\end{proof}

\begin{problem}
    Let $I$ be an irreducible ideal in a ring $A$, Then the followings are equivalent:
    \begin{enumerate}
        \item $I$ is primary.
        \item For every multiplicatively closed subset $S$ of $A$ we have $(S ^{-1}I)^c = (I : x)$ for some $x \in S$
        \item The sequence $(I : x^n)$ stabilizes, for every $x \in A$
    \end{enumerate}
\end{problem}

\begin{proof}
    As before, denote $S(I) = (S ^{-1} I)^c$.

    \begin{enumerate}
        \item (1) $\Rightarrow$ (2): Suppose $I$ is $\mathfrak{p}$-primary. By Proposition 4.8, there are two possibilities:
        \begin{enumerate}
            \item $\mathfrak{p} \cap S \ne \emptyset$, then $S(I) = A$. Pick $x \in S \cap \mathfrak{p}$, we have $x^n \in S \cap I$ for some $n \gt 0$ as $\mathfrak{p} = \sqrt{I}$ and $S$ is multiplicatively closed. Then $(I : x^n) = A$
            \item $\mathfrak{p} \cap S = \emptyset$, then $S(I) = I$. Pick arbitrary $x \in S$, then $x \notin \mathfrak{p}$ and thus $(I : x) = I$ by Proposition 4.4
        \end{enumerate}
        (Note that for this part we do not appeal to the fact that $I$ is irreducible)
        \item (2) $\Rightarrow$ (3): Take arbitrary ideal $I$ and $x \in A$, note that by Proposition 3.11, $S(I) = I^{ec} = \bigcup\limits_{n} (I : x^n)$. By our hypothesis, $S(I) = (I : x^k)$ for some $k$, it follows that the chain stabilizes. (Note that for this part we still do not appeal to the fact that $I$ is irreducible.)
        \item (3) $\Rightarrow$ (1): This is by the proof of Lemma 7.12, but we repeat it here. Replace $A$ by $A / I$, ISTS if the chain $\mathrm{Ann}(x) \subset \mathrm{Ann}(x^2) \subset \cdots$ stabilizes for any $x \in A$, and the ideal $0$ is irreducible, then every zero-divisor is nilpotent. Take $x \in A$ an arbitrary zero-divisor, suppose $xy = 0$ for some $y \ne 0$. By our hypothesis, the chain $\mathrm{Ann}(x) \subset \mathrm{Ann}(x^2) \subset \cdots$ stabilizes at some $\mathrm{Ann}(x^n)$. Then if $z x^{n + 1} = 0$, we must have $zx^n = 0$, namely $(x^n) \cap \mathrm{Ann}(x) = 0$. As $0$ is irreducible, either $x^n = 0$ or $\mathrm{Ann}(x) = 0$, but $\mathrm{Ann}(x) \ne 0$ as $y \in \mathrm{Ann}(x)$. So $x$ is nilpotent, which completes the proof.
    \end{enumerate}
\end{proof}

\begin{problem}
    Which of the following rings are Noetherian?
    \begin{enumerate}
        \item The ring of rational functions of $z$ having no pole on the circle $\left\lvert z \right\rvert = 1$.
        \item The ring of power series in $z$ with a positive radius of convergence.
        \item The ring of power series in $z$ with an infinite radius of convergence.
        \item The ring of polynomials in $z$ whose first $k$ derivatives vanish at the origin ($k$ being a fixed integer).
        \item The ring of polynomials in $z, w$ all of whose partial derivatives with respect to $w$ vanish for $z = 0$.
    \end{enumerate}
    In all cases the coefficients are complex numbers.
\end{problem}

\begin{proof}
    I omit the verifications that the sets are indeed rings.

    \TODO
\end{proof}

\begin{problem}
    Let $A$ be a Noetherian ring, $B$ a finitely generated $A$-algebra, $G$ a finite group of $A$-automorphisms of $B$, and $B^G$ the set of all elements of $B$ which are left fixed by every element of $G$. Show that $B^G$ is a finitely generated $A$-algebra
\end{problem}

\begin{proof}
    By Problem 12 of Chapter 5, $B$ is integral over $B^G$, then apply Proposition 7.8. (We may replace $A$ by its homomorphic image)
\end{proof}

\begin{problem}
    If a finitely generated ring $K$ is a field, it is a finite field.
\end{problem}

\begin{proof}
    Follow the hint.
    
    If $K$ has characteristic $0$, then the characteristic map $\mathbb{Z} \rightarrow K$ is injective. So we may regard $\mathbb{Z} \subset \mathbb{Q} \subset K$. If $K$ is finitely generated over $\mathbb{Z}$, it is finitely generated over $\mathbb{Q}$, by Proposition 7.9, it is finite over $\mathbb{Q}$. Then by Proposition 7.8, $\mathbb{Q}$ is finitely generated over $\mathbb{Z}$, which is absurd. (Contradict similarly as in Proposition 7.9, or see the lemma below)

    If $K$ has characteristic $p \gt 0$, then the characteristic map induces $\mathbb{Z} / (p) \hookrightarrow K$. It follows that $K$ is a finitely generated $\mathbb{Z} / (p)$ algebra. By Proposition 7.9, $K$ is finite over $\mathbb{Z} / (p)$, namely $K = (\mathbb{Z} / (p))^n / I$ for some ideal $I$, which is finite.
\end{proof}

\begin{problem}
    Let $X$ be an affine algebraic variety given by a family of equations $F_{\lambda}(X_1, \cdots, X_n) = 0(\lambda \in \Lambda)$(Chapter 1, Problem 27). Show that there exists a finite subset $\Lambda_0$ of $\Lambda$ such that $X$ is given by the equations $F_{\lambda}(X_1, \cdots, X_n) = 0$ for $\lambda \in \Lambda_0$
\end{problem}

\begin{proof}
    Let $I = (F_{\lambda})_{\lambda \in \Lambda}$ be the ideal defining $X$. Note that $k$ is a field and thus Noetherian, so $k[X_1, \cdots, X_n]$ is also Noetherian by Hilbert Basis Theorem. It follows that $I$ is finitely generated by $\left\lbrace F_{\lambda} \right\rbrace_{\lambda \in \Lambda_0}$ where $\Lambda_0 \subset \Lambda$ finite, which completes the proof.
\end{proof}

{\color{red} Problem 7 implies that $\mathrm{MSpec}(k[X_1, \cdots, X_n])$ is compact. But I am not sure whether the converse implication holds. \TODO}

\begin{problem}
    If $A[X]$ is Noetherian, is $A$ necessarily Noetherian?
\end{problem}

\begin{proof}
    Consider the homomorphism $A[X] \rightarrow A$ defined by $X \mapsto 0, a \mapsto a$ for $a \in A$. Then it is surjective. By Proposition 7.1, $A$ is Noetherian.
\end{proof}

\begin{problem}
    Let $A$ be a ring such that:
    \begin{enumerate}
        \item for each maximal ideal $\mathfrak{m}$ of $A$, the local ring $A_{\mathfrak{m}}$ is Noetherian;
        \item for each $x \ne 0$ in $A$, the set of maximal ideals of $A$ which contain $x$ is finite.
    \end{enumerate}
    Show that $A$ is Noetherian.
\end{problem}

\begin{proof}
    Follow the hint. Let $I$ be an arbitrary ideal, by Proposition 3.8 and Proposition 3.11, part 5, it suffices to find elements $x_1, \cdots, x_n \in I$ such that $I_{\mathfrak{m}} = (x_1, \cdots, x_n)_{\mathfrak{m}}$ for arbitrary maximal ideal $\mathfrak{m}$. We achieve this in two steps:
    \begin{enumerate}
        \item The maximal ideals that contain $I$: By condition 2, the maximal ideals that contain $I$ is finite, denote them as $\mathfrak{m}_1, \cdots, \mathfrak{m}_r$. For each $\mathfrak{m}_i$, since $A_{\mathfrak{m}_i}$ is Noetherian, there are $x_{i, 1}, \cdots, x_{i, n_i} \in A$ such that their images generate the extension of $I$ in $A_{\mathfrak{m}_i}$. Then we have $(\left\lbrace x_{i, j} \right\rbrace_{i, j})_{\mathfrak{m}} = I_{\mathfrak{m}}$ for arbitrary $I \subset \mathfrak{m}$.
        \item The maximal ideals that do not contain $I$: Note that if $I \not \subset \mathfrak{m}$, then $I_{\mathfrak{m}} = (1)$. So it suffices to find a finite set of $x_i$'s such that $(\left\lbrace x_i \right\rbrace_i)_{\mathfrak{m}} = (1)$ whenever $I \not \subset \mathfrak{m}$. Pick any $x_0 \in I$, by condition 2, there are only finitely many maximal ideals that contain $x_0$, let $\mathfrak{m}_1, \cdots, \mathfrak{m}_r$ be those maximal ideals that contain $x_0$ but not contain $I$. For each $\mathfrak{m}_i$, pick $x_i \in I$ such that $x_i \notin \mathfrak{m}_i$. We claim that $(x_0, x_1, \cdots, x_r)_{\mathfrak{m}} = (1)$ for arbitrary $I \not \subset \mathfrak{m}$: For each $I \not \subset \mathfrak{m}$, if $x_0 \notin \mathfrak{m}$, then $(x_0, x_1, \cdots, x_r)_{\mathfrak{m}} \supset (x_0)_{\mathfrak{m}} = (1)$; Otherwise, $\mathfrak{m}$ must be one of $\mathfrak{m}_i$ and thus $x_i \notin \mathfrak{m} \Rightarrow (x_0, x_1, \cdots, x_r)_{\mathfrak{m}} \supset (x_i)_{\mathfrak{m}} = (1)$.
    \end{enumerate}
    Merging the two sets of generators will do the job.
\end{proof}

\begin{problem}
    Let $M$ be a Noetherian $A$-module. Show that $M[X]$ (Chapter 2, Problem 6) is a Noetherian $A[X]$-module.
\end{problem}

\begin{proof}
    The proof is basically the same as the proof of Hilbert Basis Theorem.
\end{proof}

\begin{problem}
    Let $A$ be a ring such that each local ring $A_{\mathfrak{p}}$ is Noetherian. Is $A$ necessarily Noetherian?
\end{problem}

\begin{proof}
    We can use the same counter-example as in Problem 12 of Chapter 6. Let $A = \prod\limits_{i = 1}^{\infty} \mathbb{F}_2$, then the prime ideals of $A$ are $\prod\limits_{i = 1}^{n - 1} \mathbb{F}_2 \times 0 \times \prod\limits_{i = n + 1}^{\infty} \mathbb{F}_2$, and $A_{\mathfrak{p}}$ contains only one prime ideal. So $A_{\mathfrak{p}}$ are all Noetherian, but $A$ is not Noetherian (Actually Problem 12 shows that even $\mathrm{Spec}(A)$ is not Noetherian)
\end{proof}

{\color{red} Problem 11 shows that Noetherian is not a local property.}

\begin{problem}
    Let $A$ be a ring and $B$ a faithfully flat $A$-algebra(Chapter 3, Problem 16). If $B$ is Noetherian, show that $A$ is Noetherian.
\end{problem}

\begin{proof}
    Let $I_1 \subset I_2 \subset \cdots$ be an ascending chain of ideals in $A$, then $I_1^e \subset I_2^e \subset \cdots$ is an ascending chain of ideals in $B$. Since $B$ is Noetherian, the chain stabilizes at some $n$. By Problem 16 of Chapter 3, $I_m = I_m^{ec} = I_n^{ec} = I_n$ for arbitrary $m \ge n$, namely the chain $I_1 \subset I_2 \subset \cdots$ stabilizes.
\end{proof}

\begin{problem}
    Let $f: A \rightarrow B$ be a ring homomorphism of finite type. Show that the fibers of $\mathrm{Spec}(f)$ are Noetherian subspaces of $B$.
\end{problem}

{\color{red} I assume by 'Noetherian subspace', the author means a subset isomorphic to a Noetherian ring.}

\begin{proof}
    By Problem 21 in Chapter 3, take arbitrary $\mathfrak{p} \in \mathrm{Spec}(A)$, the fiber of $\mathfrak{p}$ is isomorphic to $B_{\mathfrak{p}} / \mathfrak{p} B_{\mathfrak{p}}$. Since $B$ is of finite type over $A$, $B_{\mathfrak{p}}$ is of finite type over $A_{\mathfrak{p}}$, and furthermore $B_{\mathfrak{p}} / \mathfrak{p} B_{\mathfrak{p}}$ is of finite type over $A_{\mathfrak{p}} / \mathfrak{p} A_{\mathfrak{p}} \cong k(\mathfrak{p})$, a field and hence Noetherian. By Corollary 7.7, $B_{\mathfrak{p}} / \mathfrak{p} B_{\mathfrak{p}}$ is Noetherian.
\end{proof}

\begin{problem}
    (Hilbert's Nullstellensatz, strong form) Let $k$ be an algebraically closed field, let $A$ denote the polynomial ring $k[X_1, \cdots, X_n]$ and let $I$ be an ideal in $A$. Let $V$ be the variety in $k^n$ defined by the ideal $I$, so that $V$ is the set of all $x = (x_1, \cdots, x_n) \in k^n$ such that $f(x) = 0$ for all $f \in I$. Let $I(V)$ be the ideal of $V$, i.e. the ideal of all polynomials $g \in A$ such that $g(x) = 0$ for all $x \in V$. Then $I(V) = \sqrt{I}$
\end{problem}

\begin{proof}
    I've seen a lot of proof of Hilbert's Nullstellensatz. So for this problem I will use a generalized proof appealing to previous problems about Jacobson ring. According to Problem 24 of Chapter 5, $A = k[X_1, \cdots, X_n]$ is a Jacobson ring. It follows that for any ideal $I$ of $A$, we have
    $$\sqrt{I} = \bigcap\limits_{I \subset \mathfrak{m} \in \mathrm{MSpec}(A) } \mathfrak{m}$$
    We know that the maximal ideals in $A$ are in a one-to-one correspondence with points in $k^n$ (see the proof of Problem 19 in Chapter 5). Under this correspondence, for ideal $I$ of $A$, we have
    \begin{equation} \label{eq:prob14}
        V(I) \leftrightarrow \left\lbrace \mathfrak{m} \in \mathrm{MSpec}(A): I \subset \mathfrak{m} \right\rbrace
    \end{equation}
    And:
    $$I(V(I)) = \left\lbrace f \in A: f \in \mathfrak{m}, \forall \mathfrak{m} \in V(I) \right\rbrace = \bigcap\limits_{I \subset \mathfrak{m} \in \mathrm{MSpec}(A)} \mathfrak{m} = \sqrt{I}$$
    where $V(I)$ denotes the corresponding sets in Eq \ref{eq:prob14}. This completes the proof.
\end{proof}

\begin{problem}
    Let $A$ be a Noetherian local ring, $\mathfrak{m}$ its maximal ideal and $k$ its residue field, and let $M$ be a finitely generated $A$-module. Then the followings are equivalent:
    \begin{enumerate}
        \item $M$ is free;
        \item $M$ is flat;
        \item the mapping of $\mathfrak{m} \otimes M$ into $A \otimes M$ is injective;
        \item $\mathrm{Tor}_{1}^{A}(k, M) = 0$
    \end{enumerate}
\end{problem}

\begin{proof}
    \begin{enumerate}
        \item (1) $\Rightarrow$ (2): It does not require any hypothesis that free modules are flat: By part 3 and 4 of Proposition 2.14.
        \item (2) $\Rightarrow$ (3): Note that $\mathfrak{m} \hookrightarrow A$ is an injective $A$-module homomorphism, then apply flatness.
        \item (3) $\Rightarrow$ (4): Consider the exact sequence:
        $$0 \rightarrow \mathfrak{m} \rightarrow A \rightarrow k \rightarrow 0$$
        It induces the Tor exact sequence:
        $$\mathrm{Tor}_{1}^{A}(k, M) \rightarrow \mathfrak{m} \otimes_A M \rightarrow A \otimes_A M \rightarrow k \otimes_A M \rightarrow 0$$
        By our hypothesis, $\mathfrak{m} \otimes_A M \rightarrow A \otimes_A M$ is injective so $\mathrm{Tor}_{1}^{A}(k, M) = 0$
        \item (4) $\Rightarrow$ (1): Follow the hint. Note that $k \otimes M \cong M / \mathfrak{m} M$ is a finitely dimensional $k$-vector space, pick $x_1, \cdots, x_n \in M$ such that their images in $M / \mathfrak{m} M$ form a basis. Then take $F = A^n$ and $\varphi: F \rightarrow M: e_i \mapsto x_i$. Let $E = \mathrm{ker}(\varphi)$. Then $0 \rightarrow E \rightarrow F \rightarrow M \rightarrow 0$ will be an exact sequence, which induces the Tor exact sequence by our hypothesis:
        $$0 = \mathrm{Tor}_{1}^{A}(k, M) \rightarrow k \otimes_A E \rightarrow k \otimes_A F \rightarrow k \otimes_A M \rightarrow 0$$
        Note that $\mathds{1}_k \otimes \varphi: k \otimes_A F \rightarrow k \otimes_A M$ is a surjective linear mapping of $k$-vector spaces of the same dimension (Note that $k \otimes_A A^n \cong k^n$), so it must be an isomorphism, namely $k \otimes_A E = 0$. Since $E$ is a submodule of $F = A^n$, which is Noetherian by Corollary 6.4, $E$ must be finitely generated. It follows that $E / \mathfrak{m}E = 0 \Rightarrow E = 0$ by Nakayama's lemma.
    \end{enumerate}
\end{proof}

\begin{problem}
    Let $A$ be a Noetherian ring, $M$ a finitely generated $A$-module. Then the followings are equivalent:
    \begin{enumerate}
        \item $M$ is a flat $A$-module;
        \item $M_{\mathfrak{p}}$ is a free $A_{\mathfrak{p}}$-module, for all prime ideals $\mathfrak{p}$
        \item $M_{\mathfrak{m}}$ is a free $A_{\mathfrak{m}}$-module, for all maximal ideals $\mathfrak{m}$
    \end{enumerate}
    In other words, flat = locally free.
\end{problem}

\begin{proof}
    \begin{enumerate}
        \item (1) $\Rightarrow$ (2): Note that $M_{\mathfrak{p}}$ is a flat $A_{\mathfrak{p}}$-module by Proposition 3.10, and $A_{\mathfrak{p}}$ is Noetherian local, then apply Problem 15 to show that $M_{\mathfrak{p}}$ is free.
        \item (2) $\Rightarrow$ (3): Clear
        \item (3) $\Rightarrow$ (1): It follows from Proposition 3.10 and the fact that free modules are flat.
    \end{enumerate}
\end{proof}

\begin{problem}
    Let $A$ be a ring and $M$ a Noetherian $A$-module. Show (by imitating the proofs of (7.11) and (7.12)) that every submodule $N$ of $M$ has a primary decomposition. (Chapter 4, Problems 20 - 23)
\end{problem}

\begin{proof}
    As in Lemma 7.11 and 7.12, we split the proof into two halves.

    \begin{enumerate}
        \item Every submodule of $M$ can be written as a finite intersection of irreducible submodules of $M$: By irreducible submodule, we mean a submodule $N$ of $M$ such that $N = N_1 \cap N_2$ implies $N_1 = N$ or $N_2 = N$. Suppose otherwise, the set of submodules that cannot be written as a finite intersection of irreducible submodules is non-empty. Then by Noetherian, it must contain a maximal element $N$. But then $N$ is reducible (if it is irreducible, it can be written an intersection of irreducible submodules, namely itself), suppose $N = N_1 \cap N_2$ for some $N_1, N_2 \ne N$. As $N_1, N_2$ are strictly larger than $N$, by our selection of $N$, they can be written as a finite intersection of irreducible submodules. It follows that $N$ is a finite intersection of irreducible submodules, a contradiction.
        \item Every irreducible submodule is primary: Take $Q$ an irreducible submodule of $M$. We may replace $M$ by $M / Q$ and suppose $0$ is irreducible. Then ISTS any zero-divisor in $M$ is nilpotent. Let $x$ be any zero-divisor of $M$, define $\varphi_x: m \mapsto xm$. Then consider the $\mathrm{ker}(\varphi_{x^n})$. It is clear that $\mathrm{ker}(\varphi_{x^n}) \subset \mathrm{ker}(\varphi_{x^{n + 1}})$. By Noetherian, the chain $\mathrm{ker}(\varphi_{x}) \subset \mathrm{ker}(\varphi_{x^2}) \subset \cdots$ stabilizes at some $n$. Then take any $m \in M$, if $x^{n + 1} m = 0$, then $x^n m = 0$, namely $x^nM \cap \mathrm{ker}(\varphi_x) = 0$. By definition of zero-divisor, $\mathrm{ker}(\varphi_x) \ne 0$ as $x$, since $0$ is irreducible, this implies $x^nM = 0$, namely $x$ is nilpotent \textbf{in $M$}.
    \end{enumerate}
\end{proof}

\begin{problem}
    Let $A$ be a Noetherian ring, $\mathfrak{p}$ a prime ideal of $A$, and $M$ a finitely generated $A$-module. Show that the followings are equivalent:
    \begin{enumerate}
        \item $\mathfrak{p}$ belongs to $0$ in $M$;
        \item there exists $x \in M$ such that $\mathrm{Ann}(x) = \mathfrak{p}$
        \item there exists a submodule of $M$ isomorphic to $A / \mathfrak{p}$
    \end{enumerate}
    Deduce that there exists a chain of submodules:
    $$0 = M_0 \subset M_1 \subset \cdots M_r = M$$
    such that each quotient $M_i / M_{i - 1}$ is of the form $A / \mathfrak{p}_i$, where $\mathfrak{p}_i$ is a prime ideal of $A$
\end{problem}

\begin{proof}
    \begin{enumerate}
        \item (1) $\Rightarrow$ (2): Note that by Problem 20 - 23 in Chapter 4, we have \TODO
    \end{enumerate}
\end{proof}

\begin{problem}
    Let $I$ be an ideal in a Noetherian ring $A$. Let
    $$I = \bigcap\limits_{i = 1}^{r} J_i = \bigcap\limits_{j = 1}^{s} J_j'$$
    be two minimal decomposition of $I$ as intersections of irreducible ideals. Prove that $r = s$ and that (possible after re-indexing the $J_j'$) $r(J_i) = r(J'_i)$ for all $i$.

    State and prove an analogous result for modules.
\end{problem}

\begin{proof}
    {\color{red} A little notice before the proof: It is true that irreducible ideals in a Noetherian ring are primary. But a minimal decomposition of $I$ as intersections of irreducible ideals is not necessarily a minimal prime decomposition: It only requires that no ideals in the decomposition contain the intersection of all other ideals, it does not require that they have distinct radicals. This is because the intersection of $\mathfrak{p}$-primary ideals are $\mathfrak{p}$-primary, but the intersection of irreducible ideals are not necessarily irreducible, so we cannot impose the constraint of radical ideals on irreducible decomposition.}

    The result for modules: Let $A$ be a Noetherian ring, $M$ be an arbitrary $A$-module. Let $N$ be a submodule of $M$, then by Problem 17 there $N$ has irreducible decompositions. Suppose
    $$N = \bigcap\limits_{i = 1}^{r} P_i = \bigcap\limits_{j = 1}^{s} Q_j$$
    are two minimal irreducible decompositions of $N$, then $r = s$ and after re-indexing, $r_M(P_i) = r_M(Q_i)$.

    Note that the result for ideals is a special case of the result for modules: we can take $M = A$ and note that $r_A(I) = \sqrt{(I : A)} = \sqrt{I}$. So we only prove the result for modules.

    Now for the proof. By replacing $M$ by $M / N$, we may assume $N = 0$.  For arbitrary $i_0 = 1, \cdots, r$, denote $\tilde{P}_{i_0} = P_1 \cap \cdots P_{i_0 - 1} \cap P_{i_0 + 1} \cap \cdots \cap P_r$. Note that $\tilde{P}_{i_0} \cap P_{i_0} = 0 = \bigcap\limits_{j = 1}^{s} \tilde{P}_{i_0} \cap Q_j$, by the proposition below, we must have $\tilde{P}_{i_0} \cap P_{i_0} = \tilde{P}_{i_0} \cap Q_{j_0}$ for some $j_0$, namely:
    $$0 = P_1 \cap \cdots \cap P_{i_0 - 1} \cap Q_{j_0} \cap P_{i_0 + 1} \cap \cdots \cap P_{r}$$
    So for each $i$, we can trade $P_i$ with some $Q_j$, start from $i = 1$ to $i = r$, repeatedly replace $P_i$ with some $Q_{j}$, we obtain:
    $$0 = Q_{j_1} \cap Q_{j_2} \cap \cdots \cap Q_{j_r}$$
    By minimality of the decomposition $\bigcap\limits_{j = 1}^{s} Q_j$, each $j = 1, \cdots, s$ must appear at least once in the $j_1, \cdots, j_r$. This proves that $r \ge s$. By symmetry, $s \ge r$ and hence $r = s$. As a result, there is a one-to-one correspondence between $P_i$ and $Q_j$ in the trading process, and in each step we obtain a minimal decomposition. After re-indexing, we may assume $P_i$ corresponds to $Q_i$.

    For the radical, WLOG we only show the case for $i = 1$. Note that $N = P_1 \cap \tilde{P}_1 = Q_1 \cap \tilde{P}_1$, by minimality, take $x \in \tilde{P}_1 \setminus P_1$, then $P_1 \cap \tilde{P}_1 = Q_1 \cap \tilde{P}_1$ implies $x \in \tilde{P}_1 \setminus Q_1$, we have $(P_1 : x) = (N : x) = (Q_1 : x)$. By Problem 17 and Problem 20 - 23 in Chapter 4, we have $r_M(P_1) = \sqrt{(P_1 : x)} = \sqrt{(Q_1 : x)} = r_M(Q_1)$
    which completes the proof.
\end{proof}

\begin{proposition}
    Let $N$ be an irreducible submodule of $A$-module $M$, and $P$ an arbitrary submodule of $M$, then $N \cap P$ is irreducible as a submodule of $P$.

    As a special case ($M = A$): Let $I_0$ be an irreducible ideal of ring $A$, and $I$ an arbitrary ideal, then $I \cap I_0$ is irreducible as a submodule of $A$-module $I$
\end{proposition}

\begin{proof}
    Take arbitrary submodule $P_1, P_2 \subset P$, we need to show if $P_1 \cap P_2 = P \cap N$, then we have $P_1 = P \cap N$ or $P_2 = P \cap N$.

    By replacing $M$ by $M / P \cap N$, we may assume $P \cap N = 0$. Then ISTS $P_1 \cap P_2 = 0$ implies $P_1 = 0$ or $P_2 = 0$. Consider the natural homomorphism $\pi: M \rightarrow M / N$, since $P_1 \cap P_2 = 0$, we have $\pi(P_1) \cap \pi(P_2) = 0$: If $e \in \pi(P_1) \cap \pi(P_2)$, then there is $x \in P_1, y \in P_2$ such that $\overline{x} = \overline{y} = e \Rightarrow x - y \in N$, but $x - y \in P$ as $P_1, P_2 \subset P$, so we must have $x - y \in P \cap N \Rightarrow x = y$. But then as $x \in P_1, y \in P_2$, we have $x = y = 0$ since $P_1 \cap P_2 = 0$. By the irreducibility of $P_0$, this implies $\pi(P_1) = 0$ or $\pi(P_2) = 0$, then $P_1 \subset \mathrm{ker}(\pi) = N$ or $P_2 \subset N$, but $P_i \subset P$, so we have $P_1 = 0$ or $P_2 = 0$.
\end{proof}

\begin{problem}
    Let $X$ be a topological space and let $\mathcal{F}$ be the smallest collection of subsets of $X$ which contains all open subsets of $X$ and is closed with respect to the formation of finite intersections and complements.

    \begin{enumerate}
        \item Show that a subset $E$ of $X$ belongs to $\mathcal{F}$ if and only if $E$ is a finite union of sets of the form $U \cap C$, where $U$ is open and $C$ is closed
        \item Suppose that $X$ is irreducible and let $E \in \mathcal{F}$. Show that $E$ is dense in $X$ (i.e. that $\overline{E} = X$) if and only if $E$ contains a non-empty open set in $X$
    \end{enumerate}
\end{problem}

\begin{proof}
    \begin{enumerate}
        \item By definition, the sets of the form $\bigcup\limits_{i = 1}^{n} U_i \cap C_i$ where $U_i$ open and $C_i$ closed belongs to $\mathcal{F}$. ISTS that these elements are closed under finite intersections and complements, then ISTS these elements are closed intersection and complement:
        \begin{enumerate}
            \item Intersection: Note that:
            $$\left(\bigcup\limits_{i = 1}^{n} U_i \cap C_i \right)\cap \left(\bigcup\limits_{j = 1}^{n} U_j' \cap C_j'\right) = \bigcup\limits_{i, j} (U_i \cap U_j') \cap (C_i \cap C_j')$$
            and $U_i \cap U_j'$ is open, $C_i \cap C_j'$ is closed for arbitrary $i, j$
            \item Complement: Note that:
            $$\left(\bigcup\limits_{i = 1}^{n} U_i \cap C_i\right)^c = \bigcap\limits_{i = 1}^{n} U_i^c \cup C_i^c$$
            By the previous part, ISTS each $U_i^c \cup C_i^c$ can be written as a finite union of $U \cap C$'s. But note that
            $$U_i^c \cup C_i^c = (U_i^c \cap X) \cup (C_i^c \cap X)$$
            and $X$ is both open and closed, which completes the proof.
        \end{enumerate}
        \item For the "if" part, simply note that any non-empty open set is dense as $X$ is irreducible. For the "only if" part, write $E = \bigcup\limits_{i = 1}^{n} (U_i \cap C_i)$, where the RHS does not contain any empty sets in the union. We claim that $C_i = X$ for some $i$, then $\emptyset \ne U_i \subset E$, which completes the proof. Consider the open set $X \setminus \bigcup\limits_{i = 1}^{n} C_i$, if it is non-empty, then it is a non-empty open set that does not intersect $E$, a contradiction as $E$ is dense. So we must have $\bigcup\limits_{i = 1}^{n} C_i = X$. Then by irreducibility of $X$, we must have $C_i = X$ for some $i$.
    \end{enumerate}
\end{proof}

{\color{red} Note that $E$ contains a non-empty open sets $\Leftrightarrow$ $E^c$ is not dense. So the last part equivalently says $E$ dense $\Leftrightarrow$ $E^c$ not dense.}

\begin{problem}
    Let $X$ be a Noetherian topological space (Chapter 6, Problem 5) and let $E \subset X$. Show that $E \in \mathcal{F}$ (Problem 20) if and only if, for each irreducible closed set $X_0 \subset X$, either $\overline{E \cap X_0} \ne X_0$ or else $E \cap X_0$ contains a non-empty open subset of $X_0$.

    The sets belonging to $\mathcal{F}$ are called the \textit{constructible} subsets of $X$
\end{problem}

\begin{proof}
    Let $\mathcal{F}(X_0)$ denotes the family of sets constructed on $X_0$ according to Problem 20 (namely the smallest family closed under finite intersections and complements that contains all open set in $X_0$). Then it is easy to verify that sets in $\mathcal{F}(X_0)$ are all of the form $X_0 \cap E$ where $E \in \mathcal{F}$. By Problem 20, this proves the "only if" part.

    Now for the "if" part. Follow the hint. Suppose $E \notin \mathcal{F}$, then the family $\Sigma = \left\lbrace X' \subset X: X' \text{ closed}, X' \cap E \notin \mathcal{F} \right\rbrace$ is non-empty (as $X \in \Sigma$). By Noetherian, there is a minimal element $X_0 \in \Sigma$. Then $X_0$ is irreducible: Suppose otherwise, $X_0 = X_1 \cup X_2$ for some proper closed subsets $X_1, X_2$ of $X_0$ (and thus of $X$ since $X_0$ is closed). By minimality of $X_0$, $X_1 \cap E, X_2 \cap E \in \mathcal{F}$, it follows that $X_0 \cap E = (X_1 \cap E) \cup (X_2 \cap E) \in \mathcal{F}$ as $\mathcal{F}$ is closed under finite union. This is a contradiction since $X_0 \in \Sigma$.

    By our hypothesis, either $\overline{X_0 \cap E} \ne X_0$ or $X_0 \cap E$ contains some non-empty open subset of $X_0$. But then:
    \begin{enumerate}
        \item If $\overline{X_0 \cap E} \ne X_0$, then there is a proper closed subset $X'$ of $X_0$ (and thus of $X$) such that $X_0 \cap E \subset X'$. Then $X_0 \cap E = X' \cap E \in \mathcal{F}$, since $X_0$ is minimal, a contradiction.
        \item If $X_0 \cap E$ contains some non-empty open subset of $X_0$, then $X_0 \cap E^c$ is not dense in $X_0$. Note that for any closed (or open, or any subset in $\mathcal{F}$) set $X'$, $X' \cap E \in \mathcal{F} \Leftrightarrow X' \cap E \in \mathcal{F}(X_0) \Leftrightarrow X' \cap E^c \in \mathcal{F}(X_0) \Leftrightarrow X' \cap E^c \in \mathcal{F}$, so $X'$ is also minimal in the family of closed subsets that intersect $E^c$ to $\mathcal{F}$. We have reduced to case 1.
    \end{enumerate}
\end{proof}

\begin{problem}
    Let $X$ be a Noetherian topological space and let $E$ be a subset of $X$. Show that $E$ is open in $X$ if and only if, for each irreducible closed subset $X_0$ in $X$, either $E \cap X_0 = \emptyset$ or else $E \cap X_0$ contains a non-empty open subset of $X_0$.
\end{problem}

\begin{proof}
    The proof is similar to Problem 21, and we repeat it here:

    The "if" part is trivial. For the "only if" part, denote $\tau$ the topology on $X$, and denote $\tau(X')$ the topology on subspace $X' \subset X$. Suppose $E$ is not open, then the family $\Sigma = \left\lbrace X' \subset X: X' \text{ closed}, X' \cap E \notin \tau(X') \right\rbrace$ is non-empty (as $X \in \Sigma$). By Noetherian, there is a minimal element $X_0 \in \Sigma$. Then $X_0$ is irreducible: Suppose otherwise, $X_0 = X_1 \cup X_2$ for some proper closed subsets $X_1, X_2$ of $X_0$ (and thus of $X$ since $X_0$ is closed). By minimality of $X_0$, $X_1 \cap E \in \tau(X_1) X_2 \cap E \in \tau(X_2)$, then $X_1 \setminus E$ is closed in $X_1$ and $X_2 \setminus E$ is closed in $X_2$ $\Rightarrow$ $X_1 \cap E, X_2 \cap E$ are closed in $X_0$ $\Rightarrow$ $X_0 \setminus E = (X_1 \cup X_2) \setminus E = (X_1 \setminus E) \cup (X_2 \setminus E)$ is closed in $X_0$ $\Rightarrow$ $E \cap X_0 \in \tau(X_0)$, a contradiction since $X_0 \in \Sigma$.

    By our hypothesis, either $X_0 \cap E = \emptyset$ or $X_0 \cap E$ contains some non-empty open subset of $X_0$. But the first case is not possible as the empty set is open. If $X_0 \cap E$ contains some non-empty open subset $U$ of $X_0$, let $C = X_0 \setminus U$. Then $C$ is properly closed in $X_0$ and thus closed in $X$. By minimality of $X_0$, we have $C \cap E \in \tau(C)$. Then $C \setminus E$ is closed in $C$ and thus closed in $X_0$. But then:
    $$X_0 \setminus E = (X_0 \setminus U) \setminus E = C \setminus E$$
    it follows that $X_0 \cap E \in \tau(X_0)$, a contradiction.
\end{proof}

\begin{problem}
    Let $A$ be a Noetherian ring, $f: A \rightarrow B$ a ring homomorphism of finite type (so that $B$ is Noetherian). Let $X = \mathrm{Spec}(A), Y = \mathrm{Spec}(B)$ and let $\mathrm{Spec}(f): Y \rightarrow X$ be the mapping associated with $f$. Then the image under $\mathrm{Spec}(f)$ of a constructible subset $E$ of $Y$ is a constructible subset of $X$
\end{problem}

{\color{red} If $A$ is Noetherian, then $\mathrm{Spec}(A)$ is Noetherian, and every open subspace is quasi-compact by Problem 6 of Chapter 6. Then every open set can be written as a finite union of basic open sets $X_f$. This did show that the constructible sets are closed sets in the constructible topology (Problem 27 of Chapter 3). (In the constructible topology, basic open sets are closed by Problem 28 of Chapter 3, then by the above argument every open set is closed, so the closed sets in the constructible topology contains all open sets, closed sets and clearly the finite union / intersection of them) But does that imply that the constructible sets are \textbf{exactly} the closed sets in constructible topology? \TODO}

\begin{proof}
    By Problem 20, it suffices to consider the set of $Y$ of the form $U \cap C$, where $U$ open and $C$ closed. Suppose $C = V(J)$ for some ideal $J$ of $B$, then $U \cap C$ corresponds to an open set of the subspace $\mathrm{Spec}(B / J) \simeq V(J)$. Replace $B$ by $B / J$, note that $B / J$ is still Noetherian and $f: A \rightarrow B / J$ is still of finite type, ISTS the image of open set is constructible.

    Since $B$ is Noetherian, any open subspace is quasi-compact by Problem 6 of Chapter 6. It follows that any open set $U$ of $B$ is covered by finitely many basic open set $B_{g_i}$ for some $i = 1, \cdots, n$. Then ISTS $\mathrm{Spec}(f)(B_{g_i})$ is constructible for each $i$. Replace $B$ by $B_{g_i}$, note that $B_{g_i}$ is Noetherian and $f: A \rightarrow B_{g_i}$ is of finite type, ISTS $\mathrm{Spec}(f)(\mathrm{Spec}(B))$ is constructible.

    By Problem 21, ISTS $\mathrm{Spec}(f)(Y) \cap X_0$ dense in $X_0$ implies $\mathrm{Spec}(f)(Y) \cap X_0$ contains an open set of $X_0$ for arbitrary irreducible closed subspace $X_0$ of $X$. Let $X_0 = V(\mathfrak{p})$, consider the homomorphism $\overline{f}: A / \mathfrak{p} \rightarrow B / \mathfrak{p}^e$, since $\mathrm{Spec}(f)(\mathfrak{q}) \in V(\mathfrak{p}) \Leftrightarrow \mathfrak{p}^e \subset \mathfrak{q}$, we have $\mathrm{Spec}(f)(Y) \cap X_0 \simeq \mathrm{Spec}(\overline{f})(B / \mathfrak{p}^e)$. Replace $A$ by $A / \mathfrak{p}$ and $B$ by $B / \mathfrak{p}$, ISTS $\mathrm{Spec}(f)(Y)$ dense in $X$ implies that $\mathrm{Spec}(f)(Y)$ contains an open subset of $X$, for $X$ the spectra of domain $A$.

    Suppose $\mathrm{Spec}(f)(Y)$ is dense in $X$. Since $Y$ is Noetherian, by Problem 7 of Chapter 6, the number of irreducible components of $Y$ is finite, denote them as $Y_1, \cdots, Y_n$. Then
    $$\overline{\mathrm{Spec}(f) \left(\bigcup\limits_{i = 1}^{n} Y_i\right)} = \bigcup\limits_{i = 1}^{n} \overline{\mathrm{Spec}(f)(Y_i)} = X$$
    By irreducibility of $X$ ($A$ is a domain), we have $\mathrm{Spec}(f(Y_i)) = X$ for some $i$. Suppose $Y_i = \mathrm{Spec}(B / \mathfrak{q})$ for some prime ideal $\mathfrak{q}$ of $B$. Replace $B$ by $B / \mathfrak{q}$, ISTS $\mathrm{Spec}(f)(Y)$ dense in $X$ implies that $\mathrm{Spec}(f)(Y)$ contains an open subset of $X$, for $f: A \rightarrow B$ of finite type, $A, B$ integral domain and $X = \mathrm{Spec}(A), Y = \mathrm{Spec}(B)$.

    By part 5 of Problem 21, Chapter 1, if $\mathrm{Spec}(f)(Y)$ is dense, $\mathrm{ker}(f) \subset \mathfrak{N}_A = 0$, then $f$ is injective, and we may assume $A \subset B$. By Problem 21 of Chapter 5, there is $s \in A$ such that for any algebraically closed field $\Omega$ and $\varphi: A \rightarrow \Omega$ such that $\varphi(s) \ne 0$, then $\varphi$ can be extended to a homomorphism $B \rightarrow \Omega$. Let $\mathfrak{p}$ be an arbitrary prime ideal of $A$ that avoids $s$, or equivalently $\mathfrak{p}$ is a prime ideal of $A_s$, take $\varphi: A \rightarrow A / \mathfrak{p} \rightarrow k = k(A / \mathfrak{p}) \rightarrow \Omega = \overline{k}$. Clearly $\varphi(s) \ne 0$, by our selection of $s$, $\varphi$ can be extended to $\psi: B \rightarrow \Omega$. Now consider $\mathfrak{q} = \mathrm{ker}(\psi)$, we clearly have $\mathfrak{q} \cap A = \mathrm{ker}(\varphi) = \mathfrak{p}$. This shows that $\mathrm{Spec}(B) \rightarrow \mathrm{Spec}(A_s)$ is surjective, namely $\mathrm{Spec}(f)(Y)$ contains the open set $\mathrm{Spec}(A_s)$, which completes the proof.
\end{proof}

\begin{problem}
    With the notation and hypothesis of Problem 23, $\mathrm{Spec}(f)$ is an open mapping $\Leftrightarrow$ $f$ has the going-down property.
\end{problem}

\begin{proof}
    The "only if" part is proved in Problem 10 of Chapter 5. As in Problem 23, ISTS $\mathrm{Spec}(f)(Y)$ is open in $X$. By Problem 22, ISTS for any irreducible closed subspace $X_0$ of $X$, $\mathrm{Spec}(f)(Y) \cap X_0 \ne \emptyset$ implies $\mathrm{Spec}(f)(Y) \cap X_0$ contains an open subset of $X_0$. Suppose $X_0 = V(\mathfrak{p})$, replace $A$ by $A / \mathfrak{p}$ and $B$ by $B / \mathfrak{p}^e$, ISTS $\mathrm{Spec}(f)(Y) \ne \emptyset$ implies $\mathrm{Spec}(f)(Y)$ contains a non-empty open subset of $X$, where $A$ is a domain. But $\mathrm{Spec}(f)(Y) = \emptyset$ if and only if $Y = \emptyset$. So ISTS $Y \ne \emptyset$ implies $\mathrm{Spec}(f)(Y)$ contains an open subset of $X$. By Problem 10 of Chapter 5, for arbitrary $\mathfrak{q}$ of $B$, take $\mathfrak{p} = \mathfrak{q}^c$ then $\mathrm{Spec}(f)(B_{\mathfrak{q}}) = \mathrm{Spec}(A_{\mathfrak{p}})$, a non-empty open set. 
\end{proof}

\begin{problem}
    Let $A$ be Noetherian, $f: A \rightarrow B$ of finite type and flat. Then $\mathrm{Spec}(f): \mathrm{Spec}(B) \rightarrow \mathrm{Spec}(A)$ is an open mapping.
\end{problem}

\begin{proof}
    By Problem 11 of Chapter 5 and Problem 24.
\end{proof}

\begin{problem}
    Let $A$ be a Noetherian ring and let $F(A)$ denote the set of all isomorphism classes of finitely generated $A$-modules. Let $C$ be the free abelian group generated by $F(A)$. With each short exact sequence $0 \rightarrow M' \rightarrow M \rightarrow M'' \rightarrow 0$ of finitely generated $A$-modules we associate the element $(M') - (M) + (M'')$ of $C$, where $(M)$ is the isomorphism class of $M$, etc. Let $D$ be the subgroup of $C$ generated by these elements, for all short exact sequences. The quotient group $C / D$ is called the \textit{Grothendieck group} of $A$, and is denoted by $K(A)$. If $M$ is a finitely generated $A$-module, let $\gamma(M)$, or $\gamma_A(M)$, denote the image of $(M)$ in $K(A)$.
    \begin{enumerate}
        \item Show that $K(A)$ has the following universal property: for each additive function $\lambda$ on the class of finitely generated $A$-modules, with values in an abelian group $G$, there exists a unique homomorphism $\lambda_0: K(A) \rightarrow G$ such that $\lambda(M) = \lambda_0(\gamma(M))$ for all $M$
        \item Show that $K(A)$ is generated by the elements $\gamma(A / \mathfrak{p})$, where $\mathfrak{p}$ is a prime ideal of $A$.
        \item If $A$ is a field, or more generally if $A$ is a principal ideal domain, then $K(A) \cong \mathbb{Z}$
        \item Let $f: A \rightarrow B$ be a \textit{finite} ring homomorphism. Show that restriction of scalars gives rise to a homomorphism $f_{!}: K(B) \rightarrow K(A)$ such that $f_{!}(\gamma_B(N)) = \gamma_A(N)$ for a $B$-module $N$. If $g: B \rightarrow C$ is another finite ring homomorphism, show that $(g \circ f)_{!} = f_{!} \circ g_{!}$
    \end{enumerate}
\end{problem}

\begin{proof}
    \begin{enumerate}
        \item For arbitrary isomorphism class $\gamma(M)$, define $\lambda_0(\gamma(M)) = \lambda(M)$. If $\lambda_0$ is well-defined, then it is clearly uniquely defined by $\lambda$. So ISTS $\lambda_0$ is well-defined. Let $\gamma(M) = \gamma(N)$, then
        \begin{equation} \label{eq:prob26-1}
            (M) - (N) = \sum\limits_{i = 1}^{n} (M_i') - (M_i) + (M_i'')
        \end{equation}
        for $n$ exact sequences $0 \rightarrow M_i' \rightarrow M_i \rightarrow M_i'' \rightarrow 0$. Apply $\lambda$ to the both sides of Eq \ref{eq:prob26-1} to conclude (it is clear that $\lambda$ is well-defined for the isomorphism classes).
        \item By Problem 18, for each finitely generated $A$-module $M$, we have exact sequences:
        \begin{equation} \label{eq:prob26-2}
            0 \rightarrow M_{i - 1} \rightarrow M_{i} \rightarrow A / \mathfrak{p}_i \rightarrow 0
        \end{equation}
        for each $i = 1, \cdots, r$ where $0 = M_0 \subset M_1 \subset \cdots \subset M_r = M$ and $\mathfrak{p}_i$'s are prime ideals of $A$. By Eq \ref{eq:prob26-2}, we have:
        $$\gamma(M_{i - 1}) - \gamma(M_i) + \gamma(A / \mathfrak{p}_i) = 0$$
        Adding all $i$ equations together, we have $\gamma(M) = \sum\limits_{i = 1}^{n} \gamma(A / \mathfrak{p}_i)$. Since $M$ is arbitrary, the proof is complete.
        \item We directly prove the case for $A$ a PID. Then the prime ideals in $A$ are of the form $(p)$ for $p$ irreducible. Consider the exact sequence:
        $$0 \rightarrow A \xrightarrow{\times p} A \rightarrow A / (p) \rightarrow 0$$
        Then we have $\gamma(A / (p)) = 0$ unless $p = 0$. So $K(A)$ is the free abelian group generated by $\gamma(A)$, namely $K(A) \cong \mathbb{Z}$
        \item Note that for categories based on sets, the exact sequences are a property concerning only set map. So a $B$-module exact sequence will be an $A$-module exact sequence. Moreover, since $f: A \rightarrow B$ finite, a finite $B$-module will be a finite $A$ module ({\color{red}But not vice versa as $A$-module is not even necessarily a $B$-module}). So restriction of scalars induces a homomorphism $f_{!}: K(B) \rightarrow K(A)$. Then $(g \circ f)_{!} = f_{!} \circ g_{!}$ is trivial.
    \end{enumerate}
\end{proof}

\begin{problem}
    Let $A$ be a Noetherian ring and let $F_1(A)$ be the set of all isomorphism classes of finitely generated \textit{flat} $A$-modules. Repeating the construction of Problem 26 we obtain a group $K_1(A)$. Let $\gamma_1(M)$ denote the image of $(M)$ in $K_1(A)$.
    \begin{enumerate}
        \item Show that tensor product of modules over $A$ induces a commutative ring structure on $K_1(A)$, such that $\gamma_1(M) \cdot \gamma_1(N) = \gamma_1(M \otimes N)$. The identity element of this ring is $\gamma_1(A)$.
        \item Show that tensor product induces a $K_1(A)$-module structure on the group $K(A)$, such that $\gamma_1(M) \cdot \gamma(N) = \gamma(M \otimes N)$.
        \item If $A$ is a (Noetherian) local ring, then $K_1(A) \cong \mathbb{Z}$
        \item Let $f: A \rightarrow B$ be a ring homomorphism, $B$ being Noetherian. Show that extension of scalars gives rise to a ring homomorphism $f^{!}: K_1(A) \rightarrow K_1(B)$ such that $f^{!}(\gamma_1(M)) = \gamma_1(B \otimes_A M)$. If $g: B \rightarrow C$ is another ring homomorphism (with $C$ Noetherian), then $(f \circ g)^{!} = f^{!} \circ g^{!}$
        \item If $f: A \rightarrow B$ is a finite ring homomorphism then
        $$f_{!}(f^{!}(x)y) = x f_{!}(y)$$
        for $x \in K_1(A), y \in K(B)$. In other words, regarding $K(B)$ as a $K_1(A)$-module by restriction of scalars, the homomorphism $f^{!}$ is a $K_1(A)$-module homomorphism.
    \end{enumerate}
    Remark. Since $F_1(A)$ is a subset of $F(A)$ we have a group homomorphism $\epsilon: K_1(A) \rightarrow K(A)$, given by $\epsilon(\gamma_1(M)) = \gamma(M)$. If the ring $A$ is finite-dimensional and regular, i.e. if all its local rings $A_{\mathfrak{p}}$ are regular (Chapter 11) it can be shown that $\epsilon$ is an isomorphism.
\end{problem}

\begin{proof}
    \begin{enumerate}
        \item We need to verify:
        \begin{enumerate}
            \item For finitely-generated flat modules $M, N$, $M \otimes N$ is finitely-generated and flat: Clear.
            \item The multiplication is well-defined. Suppose $\gamma_1(M) = \gamma_1(M'), \gamma_1(N) = \gamma_1(N')$, then we have:
            \begin{equation} \label{eq:prob27-1}
                (M) - (M') = \sum\limits_{i = 1}^{m} (M_i') - (M_i) + (M_i'')
            \end{equation}
            and
            \begin{equation}\label{eq:prob27-2}
                (N) - (N') = \sum\limits_{i = 1}^{n} (N_i') - (N_i) + (N_i'')
            \end{equation}
            for some exact sequences $0 \rightarrow M_i' \rightarrow M_i \rightarrow M_i'' \rightarrow 0, 0 \rightarrow N_i' \rightarrow N_i \rightarrow N_i'' \rightarrow 0$. Note that:
            $$(M \otimes N) - (M' \otimes N') = (M \otimes N) - (M \otimes N') + (M \otimes N') - (M' \otimes N')$$
            Conclude by tensoring Eq \ref{eq:prob27-1} by $N'$ and tensoring Eq \ref{eq:prob27-2} by $M$ and apply flatness.
        \end{enumerate}
        \item Similar as part 1.
        \item By Problem 15, flat $A$-modules are free. Moreover, consider the exact sequence:
        $$0 \rightarrow A^n \rightarrow A^{n + 1} \rightarrow A \rightarrow 0$$
        so we have $\gamma_1(A^{n + 1}) - \gamma_1(A^n) = \gamma_1(A)$ and by induction $\gamma_1(A^{n}) = n \gamma_1(A)$. Also, $A^n \otimes A^m = A^{n + m}$. It follows that $K_1(A) \cong \mathbb{Z}$ by the isomorphism $(A^n) \mapsto n$.
        \item We need to verify:
        \begin{enumerate}
            \item $B \otimes_A M$ is a flat finitely generated $A$-module. \TODO
            \item $f^{!}$ is compatible with multiplication and addition: Clear by definition.
        \end{enumerate}
        The rest is trivial.
        \item \TODO
    \end{enumerate}
\end{proof}

\end{document}