\documentclass{solution}

\begin{document}

\begin{problem}
    Let $A$ be a Dedekind domain, $S$ a multiplicatively closed subset of $A$. Show that $S ^{-1} A$ is either a Dedekind domain or the field of fractions of $A$.

    Suppose that $S \ne A \setminus \left\lbrace 0 \right\rbrace$, and let $H, H'$ be the ideal class groups of $A$ and $S ^{-1} A$ respectively. Show that extension of ideals induces a surjective homomorphism $H \rightarrow H'$.
\end{problem}

\begin{proof}
    It is clear that $S ^{-1} A$ is also a Noetherian domain of dimension $\le 1$. If it is of dimension $1$, by Proposition 5.12, $S ^{-1} A$ is integrally closed. It follows by Theorem 9.3 that $S ^{-1} A$ is a Dedekind domain. On the other hand, if it is of dimension $0$, then it is a field (since any non-unit is contained in a maximal ideal, which has to be $0$). Moreover, it is a subfield of the field of fraction of $A$ by construction. But the field of fraction is the smallest field containing $A$, which proves that $S ^{-1} A = k(A)$.

    Let $K$ be the field of fractions of $A$ (and also of $S ^{-1}A$). Define the map $\varphi: H \rightarrow H'$ by $\overline{M} \mapsto \overline{M^e}$ where $M$ is a fractional ideal of $A$ and $M^e$ is the $S ^{-1} A$-submodule of $K$ generated by elements of $M$. We claim that $\varphi$ is a surjective homomorphism:
    \begin{enumerate}
        \item $\varphi$ is well-defined: Suppose $\overline{M} = \overline{N}$, then $M = N(u) \Leftrightarrow M = Nu$ for some $u \in K^*$. Then we clearly have $M^e = N^eu \Leftrightarrow M^e = N^e(u) \Rightarrow \overline{M^e} = \overline{N^e}$.
        \item $\varphi$ is a homomorphism: Clear.
        \item $\varphi$ is surjective: Note that by Proposition 3.11, every (integral) ideal of $S ^{-1}A$ is an extension of ideal $I$ of $A$. Take any fractional ideal $N$ of $S ^{-1} A$, by definition there is some $x \ne 0$ in $A$ such that $xN \subset S ^{-1} A = I^e$ for some (integral and hence fractional) ideal $I$ of $A$. So we have $\overline{N} = \overline{xN} = \overline{I^e} = \varphi(\overline{I})$.
    \end{enumerate}
\end{proof}

\begin{problem}
    Let $A$ be a Dedekind domain. If $f = a_0 + a_1 X + \cdots + a_n X^n$ is a polynomial with coefficients in $A$, the \textit{content} of $f$ is the ideal $c(f) = (a_0, \cdots, a_n)$ in $A$. Prove \textit{Gauss's lemma} that $c(fg) = c(f) c(g)$
\end{problem}

\begin{proof}
    It is clear that $c(fg) \subset c(f)c(g)$, so we can test equality by localizing at each maximal ideal (Apply Proposition 3.8 to the quotient).

    So we may assume $A$ is a DVR by Theorem 9.3. Let $t$ be a parameter of the DVR, suppose $g = b_0 + b_1X + \cdots + b_mX^m$, write $a_i = u_i t^{r_i}$ and $b_i = v_it^{s_i}$, then we have:
    $$c(f) = (a_0, \cdots, a_n) = (t^{\min \left\lbrace r_i \right\rbrace}), c(g) = (b_0, \cdots, b_m) = (t^{\min \left\lbrace s_i \right\rbrace})$$
    So $c(f)c(g) = (t^{\min \left\lbrace r_i \right\rbrace + \min \left\lbrace s_j \right\rbrace})$.

    Pick $i_0$ the smallest $i$ such that $r_{i} = \min \left\lbrace r_i \right\rbrace$, pick $j_0$ similarly. Now consider the coefficient of $X^{i_0 + j_0}$ in $fg$, it is
    $$\sum\limits_{i + j = i_0 + j_0} a_{i}b_j = \sum\limits_{i + j = i_0 + j_0} u_iv_j t^{r_i + s_j}$$
    Consider the terms on the RHS. If $i \ne i_0$ or $j \ne j_0$, then at least one of $i, j$ will be less than $i_0$ or $j_0$ respectively and as a result $r_i + s_j \gt r_{i_0} + s_{j_0}$. So the summation has valuation $r_{i_0} + s_{j_0} = \min \left\lbrace r_i \right\rbrace + \min \left\lbrace s_j \right\rbrace$. It follows that $c(fg) \supset c(f)c(g)$. But we already show the other direction, which completes the proof.
\end{proof}

\begin{problem}
    A valuation ring (other than a field) is Noetherian if and only if it is a discrete valuation ring.
\end{problem}

\begin{proof}
    The "if" part is clear (the ideals are all of the form $\mathfrak{m}^n$).

    For the "only if" part, note that by Proposition 5.18, the valuation ring is local and integrally closed. By definition, it is a domain. By Proposition 9.2, ISTS the valuation ring is of dimension 1. Note that by Problem 28 of Chapter 5, every finitely generated ideal of a valuation ring is principal. By Noetherian, the ring is a PID. Now take any nonzero prime ideal $(p) \ne (0)$, we must have $p \in \mathfrak{m} = (m)$, the maximal ideal. But then $p = mn$ for some $n$, by $(p)$ prime, we have $m \in (p)$ or $n \in (p)$. In the first case we clearly have $(m) = (p)$. In the second case we have $p = mpr$ for some $r$, by domain, this shows that $mr = 1$, namely $m$ is a unit, a contradiction.
\end{proof}

\begin{problem}
    Let $A$ be a local domain which is not a field and in which the maximal ideal $\mathfrak{m}$ is principal and $\bigcap\limits_{n = 1}^{\infty} \mathfrak{m}^n = 0$. Prove that $A$ is a discrete valuation ring.
\end{problem}

\begin{proof}
    Take arbitrary ideal $I$ of $A$, not $(1)$ or $0$. By locality, $I \subset \mathfrak{m}$. Then since $\bigcap\limits_{n = 1}^{\infty} \mathfrak{m}^n = 0$, we must have $I \subset \mathfrak{m}^n$ but $I \not \subset \mathfrak{m}^{n + 1}$ for some $n$. Note that $\mathfrak{m}$ is principal, we can write $\mathfrak{m} = (x)$ and thus $\mathfrak{m}^n = (x^n)$. Take $a \in I \setminus (x^{n + 1})$. Since $I \subset \mathfrak{m}^n = (x^n)$, $a$ is divisible by $x^n$, say $a = yx^n$. If $y$ is not a unit, then $y \in \mathfrak{m} \Rightarrow yx^n \in (x^{n + 1})$, a contradiction. So $y$ is a unit $\Rightarrow x^n \in I \Rightarrow \mathfrak{m}^n \subset I \Rightarrow I = \mathfrak{m}^n = (x^n)$. This proves that $A$ is a PID, thus Noetherian. It is clear that PID (that is not a field) is of dimension $1$, conclude by Proposition 9.2
\end{proof}

\begin{problem}
    Let $M$ be a finitely generated module over a Dedekind domain. Prove that $M$ is flat $\Leftrightarrow$ $M$ is torsion-free
\end{problem}

\begin{proof}
    $M$ flat $\Leftrightarrow$ $M$ locally free (By Problem 16 of Chapter 7, since the ring is Noetherian) $\Leftrightarrow$ $M$ locally torsion-free (By the lemma below)$\Leftrightarrow$ $M$ torsion-free(By Problem 13 of Chapter 3)
\end{proof}

\begin{lemma}
    Let $A$ be a PID, $M$ a finitely-generated $A$-module. Then $M$ is torsion-free if and only if $M$ is free.
\end{lemma}

\begin{proof}
    The "if" part is clear. The "only if" part is by the structure theorem of modules over PID.
\end{proof}

\begin{problem}
    Let $M$ be a finitely-generated torsion module over a Dedekind domain $A$. Prove that $M$ is uniquely representable as a finite direct sum of modules $A / \mathfrak{p}_i^{n_i}$, where $\mathfrak{p}_i$ are non-zero prime ideals of $A$.
\end{problem}

\begin{proof}
    Follow the hint. For each prime $\mathfrak{p} \in \mathrm{Spec}(A)$, we have $M_{\mathfrak{p}}$ an $A_{\mathfrak{p}}$-module, and $A_{\mathfrak{p}}$ is a DVR by Theorem 9.3, and hence a PID whose ideals are $\mathfrak{p}^i A_{\mathfrak{p}}$. By the structural theorem of modules over PID, we have
    $$M_{\mathfrak{p}} = \prod\limits_{i = 1}^{n} A_{\mathfrak{p}} / \mathfrak{p}^{r_i} A_{\mathfrak{p}}$$
    as $A_{\mathfrak{p}}$ modules. But the same map is also an $A$-module isomorphism.

    By Problem 19 in Chapter 3, we have $\mathrm{Supp}(M) = V(\mathrm{Ann}(M))$. By Corollary 9.4, we have $\mathrm{Ann}(M) = \bigcap\limits_{i = 1}^{n} \mathfrak{p}_i^{r_i}$ where $\mathfrak{p}_i \ne 0$ are primes. (Note that $\mathrm{Ann}(M) \ne 0$ as $M$ is torsion and finitely generated, we can multiply the annihilators of all the generators of $M$). Suppose prime ideal $\mathfrak{q} \supset \mathrm{Ann}(M)$, then we have $\mathfrak{q} \supset \sqrt{\mathrm{Ann}(M)} = \bigcap\limits_{i = 1}^{n} \mathfrak{p_i}$ by Exercise 1.13. By Proposition 1.11, $\mathfrak{p}_i \subset \mathfrak{q}$ for some $i$. But since every nonzero prime ideal is maximal, we must have $\mathfrak{q} = \mathfrak{p}_i$ for some $i$. It follows that $\mathrm{Supp}(M) = \left\lbrace \mathfrak{p}_i \right\rbrace_{i = 1}^{n}$.

    Now consider the map:
    $$\varphi: M \rightarrow \bigoplus_{i = 1}^n M_{\mathfrak{p}_i}$$
    We claim that this is an isomorphism. To do so, we apply Proposition 3.9 and shows that $\varphi_{\mathfrak{p}}$ is an isomorphism for arbitrary $\mathfrak{p}$. If $\mathfrak{p} \notin \mathrm{Supp}(M)$, clearly both sides are zero after localization. For $\mathfrak{p} = \mathfrak{p}_i$, for $j = i$, we have $(M_{\mathfrak{p}_j})_{\mathfrak{p}_i} = M_{\mathfrak{p}_j}$ as:
    $$(M_{\mathfrak{p}})_{\mathfrak{p}} \cong A_{\mathfrak{p}} \otimes_A A_{\mathfrak{p}} \otimes_A M$$
    and $A_{\mathfrak{p}} \otimes_A A_{\mathfrak{p}} \cong S_{\mathfrak{p}} ^{-1} (S_{\mathfrak{p}} ^{-1} (A)) \cong (S_{\mathfrak{p}}S_{\mathfrak{p}}) ^{-1} A = S_{\mathfrak{p}}^{-1} A$ where the second step is by Problem 3 in Chapter 3. For $j \ne i$, we claim that $(M_{\mathfrak{p}_j})_{\mathfrak{p}_i} = 0$, and it suffices to show that there is an element $x \notin \mathfrak{p}_i$ that annihilates $M_{\mathfrak{p}_j}$: We know that $M_{\mathfrak{p}_j} \cong \bigoplus_{l = 1}^{m_j} A_{\mathfrak{p}_j} / \mathfrak{p}_j^{r_l} A_{\mathfrak{p_j}}$, let $r$ be the maximal $r_l$. Then since $\mathfrak{p}_i, \mathfrak{p}_j$ are coprime (they are distinct nonzero prime ideals), we can take $x \in \mathfrak{p}_j \setminus \mathfrak{p}_i$, then $x^r \notin \mathfrak{p}_i$ and annihilates $M_{\mathfrak{p}_j}$, which completes the proof.

    Now we have:
    $$M = \bigoplus_{i = 1}^n \left(\bigoplus_{j = 1}^{m_i} A_{\mathfrak{p}_i} / \mathfrak{p}_i^{r_j}A_{\mathfrak{p}_i}\right)$$
    So ISTS $A / \mathfrak{p}^i A \cong A_{\mathfrak{p}} / \mathfrak{p}^i A_{\mathfrak{p}}$. Let's consider the homomorphism $A \rightarrow A_{\mathfrak{p}} / \mathfrak{p}^i A_{\mathfrak{p}}$ defined by $x \mapsto \overline{x / 1}$, simply check it is surjective with kernel $\mathfrak{p}^i$
\end{proof}

{\color{red} Note that Problem 6 gives a generalized result to the 'primary decomposition of modules over PID'. Apply CRT(Problem 9) to get the corresponding 'invariant form' of f.g. modules over Dedekind domain.}

\begin{problem}
    Let $A$ be a Dedekind domain and $I \ne 0$ an ideal in $A$. Show that every ideal in $A / I$ is principal.

    Deduce that every ideal in $A$ can be generated by at most $2$ elements.
\end{problem}

\begin{proof}
    The second part is trivial. For the first part, if $I$ is maximal or $(1)$, there is nothing to prove. Otherwise, by Corollary 9.3, we have $I = \bigcap\limits_{i = 1}^{n} \mathfrak{p}_i^{r_i}$, by the same argument as in Problem 7, we have $V(I) = \left\lbrace \mathfrak{p}_i \right\rbrace_{i = 1}^n$. Now take arbitrary $J \subset I$, we must have $J = \prod\limits_{i = 1}^{n} \mathfrak{p}_i^{s_i}$ where $s_i \le r_i$. Pick $x_i \in \prod\limits_{j = 1}^{n}\mathfrak{p}_j^{s_j} \setminus \mathfrak{p}_i \prod\limits_{j = 1}^{n} \mathfrak{p}_j^{s_j}$, and let $x = \sum\limits_{i = 1}^{n} x_i$. Use localization at $\mathfrak{p}_i$'s to show that $\overline{x}$ generates $J / I$.
\end{proof}

\begin{problem}
    Let $I, J, K$ be three ideals in a Dedekind domain. Prove that:
    $$
        \begin{aligned}
        I \cap (J + K) &= (I \cap J) + (I \cap K) \\
        I + (J \cap K) &= (I + J) \cap (I + K)
    \end{aligned}
    $$
\end{problem}

\begin{proof}
    For the first equality: It is clear that $I \cap (J + K) \supset (I \cap J) + (J \cap K)$. ISTS the converse. Localize and assume $A$ is a DVR. The rest is simply by calculating the valuation of $I, J, K$. Similar for the second equality.
\end{proof}

\begin{problem}
    (Chinese Remainder Theorem). Let $I_1, \cdots, I_n$ be ideals and let $x_1, \cdots, x_n$ be elements in a Dedekind domain $A$. Then the system of congruences $x \equiv x_i \pmod {I_i}(1 \le i \le n)$ has a solution $x$ in $A$ $\Leftrightarrow$ $x_i \equiv x_j \pmod{I_i + I_j}$ whenever $i \ne j$
\end{problem}

\begin{proof}
    Follow the hint, this is equivalent to saying the sequence of $A$-modules:
    $$A \xrightarrow{\varphi} \bigoplus_{i = 1}^n A / I_i \xrightarrow{\psi} \bigoplus_{i \lt j} A / (I_i + I_j)$$
    is exact, where $\varphi(x) = (x + I_i)_{i}$ and $\psi((x + I_i)_i) = (x_i - x_j + I_i + I_j)_{i \lt j}$.

    We can test exactness by localization. Then we can assume $A$ is a DVR, argue by calculating the valuation of the ideals.
\end{proof}

\end{document}