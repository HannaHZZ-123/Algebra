\documentclass{solution}

\begin{document}

\begin{problem}
    Show that $(\mathbb{Z} / m \mathbb{Z}) \otimes_{\mathbb{Z}} (\mathbb{Z} / n \mathbb{Z}) = 0$ if $m, n$ are coprime.
\end{problem}

\begin{proof}
    ISTS $x \otimes y = 0$ for all $x \in \mathbb{Z} / m \mathbb{Z}$ and $y \in \mathbb{Z} / n \mathbb{Z}$. Since $m, n$ coprime, we have $am + bn = 1$ for some $a, b \in \mathbb{Z}$. As a result:
    $$x \otimes y = x \otimes (am + bn)y = x \otimes amy = mx \otimes ay = 0 \otimes ay = 0$$
    where the last equation is because any bilinear map will map $(0, ay)$ to $0$
\end{proof}

\begin{problem}
    Let $A$ be a ring, $I$ an ideal, $M$ an $A$-module. Show that $(A / I) \otimes_A M$ is isomorphic to $M / IM$
\end{problem}

\begin{proof}
    Tensor the exact sequence $I \rightarrow A \rightarrow A / I \rightarrow 0$ with $M$, we have:
    $$I \otimes_A M \rightarrow A \otimes_A M \rightarrow A / I \otimes_A M \rightarrow 0$$
    exact by Proposition 2.18. Note that $A \otimes_A M \cong M$ by the isomorphism $a \otimes m \mapsto am$, and $I \otimes_A M$ is isomorphic to the submodule $IM$ by the same isomorphism. As a result, $A / I \otimes_A M \cong (A \otimes_A M) / (I \otimes_A M) \cong M / IM$
\end{proof}

Or you can verify by showing that any bilinear map $(A/I) \times M \rightarrow N$ factors uniquely through $(M / IM, (\overline{a}, m) \mapsto \overline{am})$.

Now we know that $I \otimes_A M \cong IM$ and $(A / I) \otimes_A M \cong M / IM$ so the quotient and submodule 'carries through' tensor products.

\begin{problem}
    Let $A$ be a local ring, $M$ and $N$ finitely generated $A$-modules. Prove that if $M \otimes_A N = 0$, then $M = 0$ or $N = 0$.
\end{problem}

\begin{proof}
    Let $\mathfrak{m}$ be the maximal ideal of $A$, $k = A / \mathfrak{m}$ the residue field. Note that $M \otimes_A N / \mathfrak{m} (M \otimes_A N)$ is annihilated by $\mathfrak{m}$ and thus is a $A / \mathfrak{m} = k$-vector space.

    Also note that:
    $$
        \begin{aligned}
        (M \otimes_A N) / \mathfrak{m} (M \otimes_A N) &\cong A / \mathfrak{m} \otimes_A (M \otimes_A N) \\
        &\cong (A / \mathfrak{m} \otimes_{A / \mathfrak{m}} A / \mathfrak{m}) \otimes_A (M \otimes_A N) \\
        &\cong ((A / \mathfrak{m} \otimes_{A / \mathfrak{m}} A / \mathfrak{m}) \otimes_A M) \otimes_A N \\
        &\cong (A / \mathfrak{m} \otimes_{A / \mathfrak{m}} ((A / \mathfrak{m}) \otimes_A M)) \otimes_A N \\
        &\cong ((A / \mathfrak{m}) \otimes_A M) \otimes_{A / \mathfrak{m}} ((A / \mathfrak{m}) \otimes_A N) \\
        &\cong (M / \mathfrak{m}M) \otimes_{A / \mathfrak{m}} (N / \mathfrak{m}N)
        \end{aligned}
    $$
    where steps 1, 6 follows from Problem 2, step 2 follows from Proposition 2.14 iv, step 3, 4, 5 follows from Exercise 2.15.

    If $M \otimes_A N = 0$, by the above discussion, we must have $(M / \mathfrak{m}M) \otimes_{A / \mathfrak{m}} (N / \mathfrak{m}N) = 0$, but this is a tensor product of $k$-vector spaces, by the lemma below, we must have $M / \mathfrak{m} M = 0$ or $N / \mathfrak{m}N = 0$. WLOG, if $M / \mathfrak{m}M = 0$, by Nakayama's lemma, $M = 0$
\end{proof}

Let's clear a few facts about tensor product of vector spaces.

\begin{lemma}
    Let $k$ be a field and $U, V$ two $k$-vector spaces. Then $U \otimes V$ has the following properties:
    \begin{enumerate}
        \item $x \otimes y = 0$ if and only if $x = 0$ or $y = 0$. As a result, if $U \otimes V = 0$, we must have $U = 0$ or $V = 0$
        \item Let $\left\lbrace e_{i} \right\rbrace_{i \in I}$ be a basis for $U$ and $\left\lbrace f_j \right\rbrace_{j \in J}$ be a basis for $V$, then $U \otimes V$ has a basis $\left\lbrace e_i \otimes f_j \right\rbrace_{i \in I, j \in J}$. In particular, this shows that $k^n \otimes k^m = k^{mn}$
    \end{enumerate}
\end{lemma}

\begin{proof}
    \begin{enumerate}
        \item It is clear that $x \otimes y = 0$ if and only if for any bilinear map $f: U \times V \rightarrow W$, there is $f(x, y) = 0$. Then if $x = 0$ or $y = 0$, by bilinearity, $f(x, y) = 0$ for any bilinear map $f$, this proves the 'if' part. For the 'only if' part, if $x \ne 0, y \ne 0$, consider the subspace of $U$ generated by $x$ and the subspace of $V$ generated by $y$, write $U = (x) \oplus U_0, V = (y) \oplus V_0$. Define map $f: U \times V \rightarrow k$ by $(ax + x_0, by + y_0) \mapsto ab$ where $x_0 \in U_0, y_0 \in V_0$, then it is easy to verify that this is a bilinear map and $f(x, y) \ne 0$.
        \item It span the whole space: ISTS it spans the generating set $\left\lbrace x \otimes y \right\rbrace_{x \in U, y \in V}$. Suppose $x = \sum\limits_{i \in I} c_i e_i$ and $y = \sum\limits_{j \in J} d_j f_j$ where the sums are finite. Then $x \otimes y = \sum\limits_{i \in I, j \in J} c_id_j e_i \otimes f_j$ by bilinearity.

        It is linearly independent: Suppose $\sum\limits_{i \in I, j \in J} \lambda_{ij} e_i \otimes f_j = 0$ and the sum is finite. WLOG, we may assume $\sum\limits_{i = 1}^{n} \sum\limits_{j = 1}^{m} \lambda_{ij} e_i \otimes f_j = 0$. Consider the bilinear mapping $\varphi: U \times V \rightarrow k^{n \times m}$ defined by $\left(\sum\limits_{i \in I} c_i e_i, \sum\limits_{j \in J} d_j f_j\right) \mapsto A \in k^{n \times m}$ where $A_{ij} = c_id_j$. And it satisfies $\sum\limits_{i = 1}^{n} \sum\limits_{j = 1}^{m} \lambda_{ij} \varphi(e_i, f_j) = (\lambda_{ij})_{ij} = 0$, which implies $\lambda_{ij} = 0, \forall i, j$
    \end{enumerate}
\end{proof}

\begin{problem}
    Let $M_i(i \in I)$ be any family of $A$-modules, and let $M$ be their direct sum. Prove that $M$ is flat $\Leftrightarrow$ each $M_i$ is flat.
\end{problem}

\begin{proof}
    Let $f: N' \rightarrow N$ be injective. We claim that
    $$f \otimes \mathds{1}_M: N \otimes (\bigoplus\limits_{i \in I} M_i) \rightarrow N' \otimes (\bigoplus\limits_{i \in I} M_i)$$
    is injective if each $f \otimes \mathds{1}_{M_i}: N \otimes M_i \rightarrow N' \otimes M_i$ is injective, which completes the proof by Proposition 2.19.

    To prove the claim, note that by Proposition 2.14, we have:
    $$N \otimes (\bigoplus\limits_{i \in I} M_i) \cong \bigoplus\limits_{i \in I} (N \otimes M_i)$$
    And similar for $N'$. So $f \otimes \mathds{1}_M = (f \otimes \mathds{1}_{M_i})_{i \in I}$. The rest is trivial.
\end{proof}


\begin{problem}
    Let $A[x]$ be the ring of polynomials in one indeterminate over a ring $A$. Prove that $A[x]$ is a flat $A$-algebra.
\end{problem}

\begin{proof}
    Note that $A[x] \cong \bigoplus_{i \in \mathbb{N}} A$ as $A$-module. $A$ is flat by Proposition 2.14 iv. Conclude with Problem 4.
\end{proof}

\begin{problem}
    For any $A$-module, let $M[x]$ denote the set of all polynomials in $x$ with coefficients in $M$, that is to say expressions of the form:
    $$m_0 + m_1x + \cdots + m_rx^r(m_i \in M)$$
    Defining the product of an element of $A[x]$ and an element of $M[x]$ in the obvious way. Show that $M[x]$ is an $A[x]$-module.

    Show that $M[x] \cong A[x] \otimes_A M$
\end{problem}

\begin{proof}
    The first part is a tedious verification of definition and thus is omitted. (\TODO find an elegant verification)

    Now for any $A$-bilinear map $f: A[x] \times M \rightarrow T$, we claim that it factors uniquely though the $A$-bilinear map
    $$\varphi: A[x] \times M \rightarrow M[x]: (a_0 + a_1x + \cdots + a_rx^r, m) \mapsto a_0m + a_1mx+ \cdots + a_rmx^r$$
    To define an $A$-linear map $g: M[x] \rightarrow T$, it suffices to define for every $m_ix^i \in M[x]$ and extends by linearity. Define $g: m_ix^i \mapsto f(x^i, m_i)$. By bilinerity of $f$, we have $g(am_i x^i) = f(x^i, am_i) = a f(x^i, m_i)$, so $g$ is indeed an $A$-linear map. We claim that $g \circ \varphi = f$: for any $p(x) = a_0 + a_1x + \cdots + a_rx^r \in A[x]$ and $m \in M$, we have:
    $$g \circ \varphi(p(x), m) = g (p(x)m) = \sum\limits_{i = 0}^{r} f(x^i, a_im) = \sum\limits_{i = 0}^{r} f(a_ix^i, m) = f(p(x), m)$$
    by arbitrarity of $p, m$, $g \circ \varphi = f$.

    For uniqueness, if $f = h \circ \varphi$, then $h (m_ix^i) = h \circ \varphi(x^i, m_i) = f(x^i, m_i)$, so $h$ and $g$ agrees on all $m_ix^i$ and by linearity on $M[x]$
\end{proof}

\begin{remark}
    Do not try to seek a surjective map $M \times N \rightarrow M \otimes N$, because other than the trivial case, there is no such map.

    The result is interesting to me. It shows that $M^{(\mathbb{N})} \cong A^{(\mathbb{N})} \otimes_A M$
\end{remark}

\begin{problem}
    Let $\mathfrak{p}$ be a prime ideal in $A$. Show that $\mathfrak{p}[x]$ is a prime ideal in $A[x]$. If $\mathfrak{m}$ is a maximal ideal in $A$, is $\mathfrak{m}[x]$ a maximal ideal in $A[x]$?
\end{problem}

\begin{proof}
    It is easy to verify that $I[x]$ is an ideal in $A[x]$ for ideal $I$ of $A$. Now suppose $fg \in \mathfrak{p}[x]$, WLOG, we may assume $f = a_0 + \cdots + a_rx^r, g = b_0 + \cdots + b_sx^s$, where $a_0, b_0 \ne 0$ (since $f \in \mathfrak{p}[x] \Leftrightarrow fx^n \in \mathfrak{p}[x] \text{ for some $n$}$).

    We induct on $r + s$. If $r + s = 0$, then $f = a_0, g = b_0$ and $fg = a_0b_0 \in \mathfrak{p}[x] \Rightarrow a_0b_0 \in \mathfrak{p}$, since $\mathfrak{p}$ is prime, $a_0 \in \mathfrak{p}$ or $b_0 \in \mathfrak{p}$ and thus $f \in \mathfrak{p}[x]$ or $g \in \mathfrak{p}[x]$. Now for $r + s \gt 0$. Consider the constant term of $fg$, which is $a_0b_0$. Since $f g \in \mathfrak{p}[x]$, this implies $a_0b_0 \in \mathfrak{p} \Rightarrow$ $a_0 \in \mathfrak{p}$ or $b_0 \in \mathfrak{p}$. WLOG, $a_0 \in \mathfrak{p}$. Then $a_0 g \in \mathfrak{p}[x]$ $\Rightarrow$ $(f - a_0) g \in \mathfrak{p}[x]$, but this implies $(a_1 + \cdots + a_rx^{r - 1})g \in \mathfrak{p}$, by IH, $g \in \mathfrak{p}[x]$ or $(a_1 + \cdots + a_rx^{r - 1}) \in \mathfrak{p}[x]$. But since $a_0 \in \mathfrak{p}$, the latter is equivalent to $f \in \mathfrak{p}[x]$, which completes the proof.

    If $\mathfrak{m}$ is a maximal ideal in $A$, $\mathfrak{m}[x]$ is not necessarily a maximal ideal in $A[x]$. A counter-example: $A = \mathbb{C}$ and $\mathfrak{m} = 0$
\end{proof}

\begin{problem}
    \begin{enumerate}
        \item If $M, N$ are flat $A$-modules, then so is $M \otimes_A N$.
        \item If $B$ is a flat $A$-algebra and $N$ is a flat $B$-module, then $N$ is flat as an $A$-module.
    \end{enumerate}
\end{problem}

\begin{proof}
    \begin{enumerate}
        \item Take arbitrary injective $f: P' \rightarrow P$, consider $f \otimes \mathds{1}_{M \otimes_A N}: P' \otimes_A (M \otimes_A N) \rightarrow P \otimes_{A} (M \otimes_A N)$. It is equivalent to $(f \otimes \mathds{1}_{M}) \otimes \mathds{1}_N: (P' \otimes_A M) \otimes_A N \rightarrow (P \otimes_{A} M) \otimes_A N$. By flatness of $M$, $f \otimes \mathds{1}_M$ is injective, and by flatness of $N$, $(f \otimes \mathds{1}_{M}) \otimes \mathds{1}_N$ is injective, namely $f \otimes \mathds{1}_{M \otimes_A N}$ is injective.
        \item It follows from similar argument as above by noting:
        $$M \otimes_A N \cong M \otimes_A (B \otimes_B N) \cong (M \otimes_A B) \otimes_B N$$
    \end{enumerate}
\end{proof}

\begin{problem}
    Let $0 \rightarrow M' \rightarrow M \rightarrow M'' \rightarrow 0$ be an exact sequence of $A$-modules. If $M'$ and $M''$ are finitely generated, then so is $M$.
\end{problem}

\begin{proof}
    Let $u_1, \cdots, u_r$ be a set of generators of $M'$ and $v_1, \cdots, v_s$ be a set of generators of $M''$. Denote $f: M' \rightarrow M$ and $g: M \rightarrow M''$. We claim that $f(u_1), \cdots, f(u_r), w_1, \cdots, w_s$ generates $M$ where $w_i \in g ^{-1} (v_i)$

    Take any $x \in M$, consider $g(x) = \sum\limits_{i = 1}^{s} c_i v_i$ where $c_i \in A$. Then $g(x) = g(\sum\limits_{i = 1}^{s} c_i w_i)$, as a result:
    $$x - \sum\limits_{i = 1}^{s} c_iw_i \in \ker (g) = \mathrm{im}(f)$$
    which implies:
    $$x - \sum\limits_{i = 1}^{s} c_iw_i = f(\sum\limits_{j = 1}^{r} d_ju_j) = \sum\limits_{j = 1}^{r} f(u_j)$$
    which completes the proof.
\end{proof}

\begin{problem}
    Let $A$ be a ring, $I$ an ideal contained in the Jacobson radical of $A$; let $M$ be an $A$-module and $N$ a finitely generated $A$-module, and let $u: M \rightarrow N$ be a homomorphism. If the induced homomorphism $M / IM \rightarrow N / IN$ is surjective, then $u$ is surjective.
\end{problem}

\begin{proof}
    Note that the homomorphism $M / IM \rightarrow N / IN$ is induced by noting $IM$ is in the kernel of the composition $M \rightarrow N \rightarrow N / IN$. As a result, $M / IM \rightarrow N / IN$ is surjective if and only if $M \rightarrow N / IN$ is surjective, namely $u(M) + IN = N$. Since $u(M)$ is a submodule of $N$ and $I \subset \mathfrak{R}_A$, by Corollary 2.7, $u(M) = N$, namely $u$ is surjective.
\end{proof}

\begin{problem}
    Let $A$ be a ring $\ne 0$, show that $A^m \cong A^n \Rightarrow m = n$, and

    \begin{enumerate}
        \item If $\varphi: A^m \rightarrow A^n$ is surjective, then $m \ge n$
        \item If $A^m \rightarrow A^n$ is injective, is it always the case that $m \le n$?
    \end{enumerate}
\end{problem}

\begin{proof}
    Follow the hint. Let $\varphi: A^m \rightarrow A^n$ be the isomorphism, take a maximal ideal $\mathfrak{m}$ of $A$, then:
    $$\varphi \otimes \mathds{1}: A^m \otimes_A (A / \mathfrak{m}) \rightarrow A^n \otimes_A (A / \mathfrak{m})$$
    is also an isomorphism. Note that:
    $$A^m \otimes_A (A / \mathfrak{m}) \cong A^m / \mathfrak{m} A^m \cong A^m / \mathfrak{m}^m \cong (A / \mathfrak{m})^m \cong k^m$$
    where the step 1 is by Problem 2, step 2 is by
    $$\mathfrak{m}A^m = \left\lbrace \sum\limits_{i \in I} m_i a_i: m_i \in \mathfrak{m}, a_i \in A, \left\lvert I \right\rvert \lt \infty \right\rbrace = \mathfrak{m}^m$$
    and step 3 by the lemma below.

    As a result, $\varphi \otimes \mathds{1}$ is an isomorphism between $k$-vector spaces, so $m = n$.

    \begin{enumerate}
        \item If $\varphi: A^m \rightarrow A^n$ is surjective, $\varphi \otimes \mathds{1}$ is also surjective, and by the same argument as above $m \ge n$
        \item However, even if $\varphi: A^m \rightarrow A^n$ is injective, $\varphi \otimes \mathds{1}$ may not be injective and hence we may not have $m \le n$. A counter-example is \TODO
    \end{enumerate}
\end{proof}

\begin{lemma}
    Let $\Lambda$ be an arbitrary index set, $\left\lbrace M_\lambda \right\rbrace_{\lambda \in \Lambda}, \left\lbrace N_\lambda \right\rbrace_{\lambda \in \Lambda}$ be $R$-modules and $N_\lambda$ is a submodule of $M_\lambda$, then
    $$\prod\limits_{\lambda \in \Lambda} M_\lambda / N_\lambda \cong \prod\limits_{\lambda \in \Lambda} M_{\lambda} / \prod\limits_{\lambda \in \Lambda} N_\lambda$$
    And similarly for direct sum.
\end{lemma}

\begin{proof}
    Consider the map:
    $$\varphi: \prod\limits_{\lambda \in \Lambda} M_\lambda / N_\lambda \rightarrow \prod\limits_{\lambda \in \Lambda} M_{\lambda} / \prod\limits_{\lambda \in \Lambda} N_\lambda: (\overline{x_\lambda})_{\lambda \in \Lambda} \mapsto \overline{(x_\lambda)_{\lambda \in \Lambda}}$$
    where $x_\lambda \in M_\lambda$

    \begin{enumerate}
        \item $\varphi$ is well-defined: If $(\overline{x_\lambda})_{\lambda \in \Lambda} = (\overline{y_\lambda})_{\lambda \in \Lambda}$, then $x_\lambda - y_\lambda \in N_\lambda$, as a result, $(x_\lambda)_{\lambda \in \Lambda} - (y_\lambda)_{\lambda \in \Lambda} \in \prod\limits_{\lambda \in \Lambda}N_\lambda$, namely $\overline{(x_\lambda)_{\lambda \in \Lambda}} = \overline{(y_\lambda)_{\lambda \in \Lambda}}$
        \item $\varphi$ is a homomorphism: clear
        \item $\varphi$ is surjective: clear
        \item $\varphi$ is injective: Suppose $\varphi((\overline{x_{\lambda}})_{\lambda \in \Lambda}) = \overline{(x_\lambda)_{\lambda \in \Lambda}} = 0$, then $x_\lambda \in N_\lambda, \forall \lambda \Rightarrow (\overline{x_\lambda})_{\lambda \in \Lambda} = 0$
    \end{enumerate}

    So $\varphi$ is the isomorphism.
\end{proof}

\begin{proposition}
    Let $M, N, P$ be $R$-modules, $\varphi: M \rightarrow N$ an $R$-module homomorphism. Then:
    \begin{enumerate}
        \item If $\varphi$ is an isomorphism, then $\varphi \otimes_R \mathds{1}: M \otimes_A P \rightarrow N \otimes_R P$ is also an isomorphism
        \item If $\varphi$ is surjective, then $\varphi \otimes_R \mathds{1}: M \otimes_A P \rightarrow N \otimes_R P$ is also surjective
    \end{enumerate}
\end{proposition}

\begin{proof}
    \begin{enumerate}
        \item $\varphi ^{-1} \otimes_R \mathds{1}$ is the inverse, check on the set of generators $x \otimes y$
        \item For every $x \otimes y \in N \otimes_A P$, we have $\varphi \otimes_R \mathds{1}(\varphi ^{-1}(x) \otimes y) = x \otimes y$ where $\varphi ^{-1}(x)$ denotes any $x' \in \varphi ^{-1} (x)$
    \end{enumerate}

    We can replace $\mathds{1}$ by any automorphism of $N$, the proof is basically the same.
\end{proof}

\begin{problem}
    Let $M$ be a finitely generated $A$-module and $\varphi: M \rightarrow A^n$ a surjective homomorphism, show that $\ker (\varphi)$ is finitely generated.
\end{problem}

\begin{proof}
    By our proof of Problem 9, we know that $M$ is the sum of $\ker(\varphi)$ and the submodule of $M$ generated by $\left\lbrace v_i \right\rbrace_{i = 1}^n$ where $\varphi(v_i) = e_i$, the standard set of generators of $A^n$. To show that this is a direct sum, we need to show that $\ker(\varphi) \cap \mathrm{span}(v_1, \cdots, v_n) = 0$. This is because $\sum\limits_{i = 1}^{n} c_i v_i \in \ker (\varphi) \Rightarrow \varphi \left(\sum\limits_{i = 1}^{n} c_iv_i\right) = 0 \Rightarrow \sum\limits_{i = 1}^{n} c_i \varphi(v_i) = 0 \Rightarrow \sum\limits_{i = 1}^{n} c_i e_i = 0 \Rightarrow c_i = 0, \forall i$. Conclude by the lemma below.
\end{proof}

\begin{lemma}
    Let $M$ be a finitely generated submodule, if $M = N \oplus P$, and either $N, P$ is finitely generated, then the other is also finitely generated.

    This is the 'complement' of Problem 9 and is equivalent to: If the exact sequence:
    $$0 \rightarrow M' \rightarrow M \rightarrow M'' \rightarrow 0$$
    splits, then if any two of $M', M, M''$ are finitely generated, the other is also finitely generated. (Problem 9 says that we don't need the exact sequence to split for direction $M', M'' \Rightarrow M$)
\end{lemma}

\begin{proof}
    WLOG, suppose $N$ is finitely generated. Let $e_1, \cdots, e_n$ be a set of generators of $N$. Since $M$ is also finitely generated, expand $\left\lbrace e_i \right\rbrace_{i = 1}^n$ to a larger set of generators $\left\lbrace e_1, \cdots, e_n, f_1, \cdots, f_m \right\rbrace$ of $M$. We may assume $f_i \in P$, since by $M = N \oplus P$, we may write $f_i = e + f$ where $e \in N, f \in P$, and by replacing $f_i$ with $f$, the set still generates $M$. Now we claim that $\left\lbrace f_1, \cdots, f_m \right\rbrace$ generates $P$. Otherwise, there is some $f \in P$ that cannot be represented by a linear combination of $f_i$'s, but then it cannot be represented by a linear combination of $e_i, f_i$'s, a contradiction.
\end{proof}

{\color{red}
\begin{remark}
    \begin{enumerate}
        \item The submodule of a finitely generated submodule is not necessarily finitely generated, so the lemma is non-trivial. For example, $R = k[X_1, X_2, \cdots]$ (infinitely many indeterminates) is a finitely generated $R$-module (it is generated by $1$) but the ideal $(X_1, X_2, \cdots)$ is not.
        \item In Problem 12, $A^n$ cannot be replaced by any finitely generated modules. This is because by argument in Problem 9, we only prove that $M$ is the \textbf{sum} of $\ker(\varphi)$ and $\mathrm{span}(v_i)$, it is not necessarily a direct sum. Despite $\varphi(v_i) \ne 0$, their linear combination may map to $0$ (of course this is not the case if $A$ is a field, so exact sequence of vector space always splits). A counter-example would be the composition:
        $$0 \rightarrow \mathbb{Z} \xrightarrow{\times 2} \mathbb{Z} \rightarrow \mathbb{Z} / 2 \mathbb{Z} \rightarrow 0$$
        We may take $v_1 = 1$, but clearly $\mathrm{span} \left\lbrace v_1 \right\rbrace \cap \ker (\times 2) \ne 0$
    \end{enumerate}
\end{remark}
}

\begin{problem}
    Let $f: A \rightarrow B$ be a ring homomorphism, and let $N$ be a $B$-module. Regarding $N$ as an $A$-module by restriction of scalars, form the $B$-module $N_B = B \otimes_A N$. Show that the homomorphism $g: N \rightarrow N_B$ which maps $y$ to $1 \otimes y$ is injective and that $g(N)$ is a direct summand of $N_B$.
\end{problem}

\begin{proof}
    It's clear that $g$ is a homomorphism. Consider the $A$-blinear map $\varphi: B \times N \rightarrow N: (b, n) \mapsto bn$, then $\varphi(1, y) = 0$ if and only if $y = 0$. Since $\varphi$ factors through $N_B$, $1 \otimes_A y \ne 0$ for $y \ne 0$, which completes the proof that $g$ is injective.

    Now we have the exact sequence:
    $$0 \rightarrow N \rightarrow N_B \rightarrow N_B / g(N) \rightarrow 0$$
    Define $p: N_B \rightarrow N: (b \otimes n) \mapsto bn$, then it is clear that $p \circ g = \mathds{1}_N$. It follows from the below lemma that $N_B \cong N \oplus N_B / g(N) \cong g(N) \oplus N_B / g(N)$
\end{proof}

\begin{lemma}
    Let $A, B, C$ be objects of an abelian category $\mathcal{C}$, and
    $$0 \rightarrow A \xrightarrow{i} B \xrightarrow{q} C \rightarrow 0$$
    is an exact sequence. Then the followings are equivalent
    \begin{enumerate}
        \item There is $p: B \rightarrow A$ such that $p \circ i = \mathds{1}_A$
        \item There is $j: C \rightarrow B$ such that $q \circ j = \mathds{1}_C$
        \item The sequence splits, namely $B \cong A \oplus C$ (In abelian category, this is equivalent to $B$ is the product or coproduct of $A, C$)
    \end{enumerate}
\end{lemma}

\begin{proof}
    We only show the equivalence of part 1 and part 3, part 2 is similar. It is trivial from definition that part 3 implies part 1. So we only show the converse.

    From the definition, we have:
    \begin{equation}
        q \circ i = 0
    \end{equation}
    Furthermore, $q$ is the cokernel of $i$ and $i$ is the kernel of $q$ since the sequence is exact.

    Consider the morphism: $\varphi: \mathds{1}_B - i \circ p: B \rightarrow B$. Since $\varphi \circ i = i - i \circ (p \circ i) = 0$, and $q$ is the cokernel of $i$, then $\varphi$ factors through $q$, namely there is $j: C \rightarrow B$ such that
    \begin{equation} \label{eq:lem-abelian-product}
        \mathds{1}_B = j \circ q + i \circ p
    \end{equation}
    Compose each side on the left by $p$, we have:
    $$p = p \circ j \circ q + p \circ i \circ p = p + p \circ j \circ q$$
    so we must have $p \circ j \circ q = 0$ and since $q$ is an epimorphism:
    \begin{equation}
        p \circ j = 0
    \end{equation}
    Compose each side of \ref{eq:lem-abelian-product} on the left by $q$, we have:
    $$q = q \circ j \circ q + q \circ i \circ p = q \circ j \circ q$$
    since $q \circ i = 0$. As $q$ is an epimorphism, we have:
    \begin{equation}
        q \circ j = \mathds{1}_C
    \end{equation}
    All numbered equations, together with
    \begin{equation}
        p \circ i = \mathds{1}_A
    \end{equation}
    from the hypothesis, makes $B$ a product with $(p, q)$ (or a coproduct with $(i, j)$)
\end{proof}

\begin{problem}
    A partially ordered set $I$ is said to be a directed set if for each pair $i, j$ in $I$, there exists $k \in I$ such that $i \le k$ and $j \le k$.

    Let $A$ be a ring, let $I$ be a directed set and let $(M_i)_{i \in I}$ be a family for $A$-modules indexed by $I$. For each pair $i, j$ in $I$ such that $i \le j$, let $\mu_{i, j}: M_i \rightarrow M_j$ be an $A$-homomorphism, and suppose that the following axioms are satisfied:
    \begin{enumerate}
        \item $\mu_{i, i}$ is the identity mapping of $M_i$, for all $i \in I$
        \item $\mu_{i, k} = \mu_{j, k} \circ \mu_{i, j}$ whenever $i \le j \le k$
    \end{enumerate}
    Then the modules $M_i$ and the homomorphisms $\mu_{i, j}$ are said to form a \textit{direct system} $\bm{M} = (M_i, \mu_{i, j})$ over the directed set $I$.

    We shall construct an $A$-module $M$ called the \textit{direct limit} of the direct system $\bm{M}$. Let $C$ be the direct sum of the $M_i$, and identify each module $M_i$ with its canonical image in $C$. Let $D$ be the submodule of $C$ generated by all elements of the form $x_i - \mu_{i, j}(x_i)$ where $i \le j$ and $x_i \in M_i$. Let $M = C / D$, let $\mu: C \rightarrow M$ be the projection and let $\mu_i$ be the restriction of $\mu$ to $M_i$.

    The module $M$, or more correctly the pair consisting of $M$ and the family of homomorphisms $\mu_i: M_i \rightarrow M$, is called the \textit{direct limit} of the direct system $\bm{M}$, and is written $\varinjlim M_i$. From the construction it is clear that $\mu_{i} = \mu_{j} \circ \mu_{i, j}$ whenever $i \le j$
\end{problem}

\begin{proof}
    There is not much to check about the construction. The only non-trivial fact is that $\mu_i = \mu_j \circ \mu_{i, j}$ when $i \le j$: Take arbitrary $x_i \in M_i$, then $x_i - \mu_{i, j}(x_i) \in D$, so $\mu(x_i) = \mu(\mu_{i, j} (x_i)) \Rightarrow \mu_i (x_i) = \mu_j \circ \mu_{i, j}(x_i)$. Since $x_i$ is arbitrary, we must have $\mu_i = \mu_j \circ \mu_{i, j}$
\end{proof}

\begin{problem}
    In the situation of Problem 14, show that every element of $M$ can be written in the form $\mu_i(x_i)$ for some $i \in I$ and some $x_i \in M_i$.

    Show also that if $\mu_i(x_i) = 0$ then there exists $j \ge i$ such that $\mu_{i, j}(x_i) = 0$ in $M_j$
\end{problem}

\begin{proof}
    Take any element from $M$, since $M = C/D$, it can be written as
    $$\overline{\sum\limits_{j = 1}^{n} x_{i_j}} = \sum\limits_{j = 1}^{n} \overline{x_{i_j}} = \sum\limits_{j = 1}^{n} \mu_{i_j}(x_{i_j})$$
    where $x_{i_j} \in M_{i_j}$. Since the index set $I$ is directed, by induction we may find $i_{*}$ such that $i_{j} \le i_*$ for all $j = 1, \cdots, n$. Then by Problem 14, we have:
    $$\mu_{i_j}(x_{i_j}) = \mu_{i_*}(\mu_{i_j, i_*}(x_{i_j}))$$
    As a result:
    $$\overline{\sum\limits_{j = 1}^{n} x_{i_j}} = \sum\limits_{j = 1}^{n} \mu_{i_j}(x_{i_j}) = \sum\limits_{j = 1}^{n} \mu_{i_*}(\mu_{i_j, i_*}(x_{i_j})) = \mu_{i_*} \left(\sum\limits_{j = 1}^{n} \mu_{i_j, i_*}(x_{i_j})\right)$$
    which completes the first part.

    If $\mu_i(x_i) = 0$, by definition $x_i \in D$, namely it can be written as a finite sum:
    $$x_i = \sum\limits_{k = 1}^{n} x_{i_k} - \mu_{i_k, j_k}(x_{i_k})$$
    By induction, find $i_*$ such that $i \le i_*, i_k \le i_*, j_k \le i_*$, then we have:
    $$
        \begin{aligned}
        x_i &= \sum\limits_{k = 1}^{n} x_{i_k} - \mu_{i_k, i_*}(x_{i_k}) + \mu_{i_k, i_*}(x_{i_k}) - \mu_{i_k, j_k}(x_{i_k}) \\
        &= \sum\limits_{k = 1}^{n} x_{i_k} - \mu_{i_k, i_*}(x_{i_k}) + \mu_{j_k, i_*} \circ \mu_{i_k, j_k} (x_{i_k}) - \mu_{i_k, j_k}(x_{i_k})\\
        &= \sum\limits_{k = 1}^{n} x_{i_k} - \mu_{i_k, i_*}(x_{i_k}) - \sum\limits_{k = 1}^{n} \mu_{i_k, j_k}(x_{i_k}) - \mu_{j_k, i_*} \circ \mu_{i_k, j_k} (x_{i_k})
        \end{aligned}
    $$
    So we may assume $j_k = i_*$, and $i \le i_*, i_k \le i_*$, all $i_k$ are distinct (by merging the terms) and $i_k \ne i_*$ (by deleting $x_{i_k} - \mu_{i_k, i_*}(x_{i_k}) = x_{i_k} - x_{i_k} = 0$), namely:
    \begin{equation}\label{eq:prob-15}
        x_i = \sum\limits_{k = 1}^{n} x_{i_k} - \mu_{i_k, i_*}(x_{i_k}), i_k \ne i_*, i_k \ne i_l, \forall k \ne l
    \end{equation}
    If $i = i_*$, comparing both sides on the coordinate in $M_{i_k}$, we have $x_{i_k} = 0, \forall k$, so $x_i = 0$ and clearly $\mu_{i, i}(x_{i}) = 0$. Otherwise, comparing both sides on the coordinate in $M_{i}$, we must have $i_k = i$ for some unique $k$, and thus $x_i = x_{i_k}$. Then comparing both sides on the coordinate in $M_{i_k}$ for other $k$, we must have $x_{i_k} = 0, \forall i_k \ne i$. Therefore, Eq \ref{eq:prob-15} is equivalent to $x_i = x_i - \mu_{i, i_*}(x_{i})$, namely $\mu_{i, i_*}(x_i) = 0$
\end{proof}

\begin{problem}
    Show that the direct limit is characterized (up to isomorphism) by the following property. Let $N$ be an $A$-module and for each $i \in I$ let $\alpha_{i}: M_i \rightarrow N$ be an $A$-module homomorphism such that $\alpha_{i} = \alpha_j \circ \mu_{i, j}$ whenever $i \le j$. Then there exists a unique homomorphis $\alpha: M \rightarrow N$ such that $\alpha_i = \alpha \circ \mu_i$ for all $i \in I$
\end{problem}

\begin{proof}
    If $(M, \left\lbrace \mu_i \right\rbrace_{i \in I})$ is the direct limit, and let $\alpha_i, N$ be as above. By Problem 15, elements in $M$ are all of the form $\mu_i(x_i)$, define $\alpha: M \rightarrow N: \mu_i(x_i) \mapsto \alpha_i(x_i)$. This is well-defined: For if $\mu_i(x_i) = \mu_j(x_j)$, we can take $k \ge i, j$, then $\mu_{k}(\mu_{i, k}(x_i)) = \mu_k(\mu_{j, k}(x_j)) \Rightarrow \mu_k(\mu_{i, k}(x_i) - \mu_{j, k}(x_j)) = 0$. By Problem 15, there is some $l \ge k$ such that $\mu_{k, l} (\mu_{i, k}(x_i) - \mu_{j, k}(x_j)) = 0 \Rightarrow \mu_{i, l} (x_i) = \mu_{j, l}(x_j)$. It follows that $\alpha_i(x_i) = \alpha_l(\mu_{i, l}(x_i)) = \alpha_l(\mu_{j, l}(x_j)) = \alpha_j(x_j)$.

    By the definition of $\alpha$, clearly we have $\alpha_i = \mu_i \circ \alpha, \forall i \in I$. It also follows from the definition that $\alpha$ is unique.

    Now if $(M, \mu_i)_{i \in I}, (M', \mu_i')_{i \in I}$ both satisfies the universal property. By definition, there is $\alpha: M \rightarrow M'$ and $\alpha': M' \rightarrow M$ such that $\mu_i' = \alpha \circ \mu_i, \mu_i = \alpha' \circ \mu_i'$. It follows that $\mu_i' = \alpha \circ \alpha' \circ \mu_i'$ and $\mu_i = \alpha' \circ \alpha \circ \mu_i$. Apply the universal property of $(M, \left\lbrace \mu_i \right\rbrace_{i \in I})$ to $(M, \left\lbrace \mu_i \right\rbrace_{i \in I})$ itself, we know that there is unique $\beta: M \rightarrow M$ such that $\mu_i = \beta \circ \mu_i$. But clearly $\mathds{1}_M$ is the only choice, it follows that $\mathds{1}_M = \alpha' \circ \alpha$ and similarly $\mathds{1}_M = \alpha \circ \alpha'$, namely $\alpha$ is an isomorphism $M \cong M'$ (This type of argument is standard for any universal properties.)

    Note that the second part also demonstrates that any $A$-module that satisfies the universal property must be the direct limit (up to isomorphism), which is guaranteed to exist by the construction in Problem 14.
\end{proof}

{\color{red} After appealing to Problem 16 in a few problems in later chapters, I decide that the result is still a little unhandy to apply directly. Note that the only crucial step in the proof the Problem 16 appeals to the second part of Problem 15, it is reasonable to state the following result:}

\begin{proposition}
    Let $(M_i, \mu_{i, j})$ be a directed system where $M_i$'s are $A$-modules, let $M$ be an $A$-module, and $\mu_i: M_i \rightarrow M$ $A$-module homomorphisms such that $\mu_i = \mu_j \circ \mu_{i, j}$ whenever $i \le j$. If in addition $\mu_i(x_i) = 0$ where $x_i \in M_i$ implies $\mu_{i, j} (x_i) = 0$ for some $j \ge i$, then $\varinjlim M_i \cong M$
\end{proposition}

\begin{proof}
    It suffices to verify the universal property. For arbitrary $N$ with homomorphisms $\alpha_i: M_i \rightarrow N$ such that $\alpha_i = \alpha_j \circ \mu_{i, j}$, we define $\varphi: M \rightarrow N$ as $\varphi(\mu_i(x_i)) = \alpha_i(x_i)$ for $x_i \in M_i$. By Problem 15, this defines $\varphi$ for every element in $M$. And it is easy to check that this is indeed an $A$-module homomorphism. We only need to show that it is well-defined: Suppose $\mu_i(x_i) = \mu_j(x_j)$, then by the hypothesis, there is $k \ge i, j$ such that $\mu_{i, k} (x_i) = \mu_{j, k}(x_j)$. As a result, then $\alpha_i(x_i) = \alpha_k(\mu_{i, k}(x_i)) = \alpha_k(\mu_{j, k}(x_j)) = \alpha_j (x_j)$, which completes the proof.
\end{proof}

\begin{problem}
    Let $(M_i)_{i \in I}$ be a family of submodules of an $A$-module, such that for each pair of indices $i, j$ in $I$ there exists $k \in I$ such that $M_i + M_j \subset M_k$. Define $i \le j$ to mean $M_i \subset M_j$ and let $\mu_{i, j}:M_i \rightarrow M_j$ be the inclusion. Show that:
    $$\varinjlim M_i = \sum M_i = \bigcup M_i$$
    In particular, any $A$-module is the direct limit of its finitely generated submodules.
\end{problem}

\begin{proof}
    It is routine to check that $\bm{M} = (M_i, \mu_{i, j})$ does form a direct system. We first prove that $\sum M_i = \bigcup M_i$: It is clear that $\bigcup M_i \subset \sum M_i$. For the other direction, note that by definition
    $$\sum\limits_{i \in I} M_i = \bigcup\limits_{J \subset I, \left\lvert J \right\rvert \lt \infty} \sum\limits_{j \in J} M_j$$
    But since $I$ is directed and $J$ is finite, by induction we have $\sum\limits_{j \in J} M_j \subset M_k$ for some $k$. It follows that $\sum\limits_{i \in I} M_i \subset \bigcup\limits_{i \in I} M_i$

    Now let's prove that $\bigcap_{i \in I} M_i$ is the direct limit. By Problem 16, it suffices to show that it satisfies the universal property. Define $\mu_i: M_i \hookrightarrow M$ to be the inclusion. Then clearly $\mu_j \circ \mu_{i, j} = \mu_i$ as they are both inclusions. Now suppose $(N, \alpha_i)$ satisfies $\alpha_j \circ \mu_{i, j} = \alpha_i$, define $\alpha (x) = \alpha_j(x)$ where $x \in M_j$. This is well-defined since suppose $x \in M_k$ for some $k \ne j$, we can find $l \ge k, j$ and $\alpha_j(x) = \alpha_{l}(x) = \alpha_k(x)$ by properties of $\alpha$. It is easy to check $\alpha \circ \mu_i = \alpha_j$ and $\alpha$ is uniquely determined by $\alpha_j$'s.
\end{proof}

\begin{problem}
    Let $\bm{M} = (M_i, \mu_{i, j}), \bm{N} = (N_i, \nu_{i, j})$ be direct systems of $A$-modules over the same directed set. Let $M, N$ be the direct limits and $\mu_i: M_i \rightarrow M, \nu_i: N_i \rightarrow N$ the associated homomorphisms.

    A \textit{homomorphism} $\bm{\Phi}: \bm{M} \rightarrow \bm{N}$ is by definition a family of $A$-module homomorphisms $\phi: M_i \rightarrow N_i$ such that $\phi_j \circ \mu_{i, j} = \nu_{i, j} \circ \phi_i$ whenever $i \le j$. Show that $\bm{\Phi}$ defines a unique homomorphism $\phi = \varinjlim \phi_i: M \rightarrow N$ such that $\phi \circ \mu_i = \nu_i \circ \phi_i$ for all $i \in I$.
\end{problem}

\begin{proof}
    By Problem 15, all elements in $M$ are of the form $\mu_i(x_i)$ for some $x_i \in M_i$. Define $\phi: M \rightarrow N: \mu_i(x_i) \mapsto \nu_i(\phi_i(x_i))$
    
    \begin{enumerate}
        \item $\phi$ is well-defined: Suppose $\mu_i(x_i) = \mu_j(x_j)$, by similar argument as in Problem 16, we have $\mu_{i, k}(x_i) = \mu_{j, k}(x_j)$ for some $k \ge i, j$., then $\nu_i \circ \phi_i(x_i) = \nu_i \circ \phi_k \circ \nu_{i, k} (x_i) = \nu_k \circ \nu_{i, k} \circ \varphi_i (x_i) = \nu_k \circ \phi_k \circ \nu_{i, k}(x_i) = \nu_k \circ \phi_k \circ \nu_{j, k}(x_j) = \nu_j \circ \phi_j(x_i)$ (See the commutative diagram below)
        % https://q.uiver.app/#q=WzAsNyxbMCwwLCJNX2kiXSxbMSwwLCJNX2siXSxbMiwwLCJNX2oiXSxbMSwxLCJOX2siXSxbMCwxLCJOX2kiXSxbMiwxLCJOX2oiXSxbMSwyLCJOIl0sWzAsMSwiXFxtdV97aSwga30iXSxbMiwxLCJcXG11X3tqLCBrfSIsMl0sWzEsMywiXFxwaGlfayIsMl0sWzAsNCwiXFxwaGlfaSIsMl0sWzIsNSwiXFxwaGlfaiJdLFs0LDMsIlxcbnVfe2ksIGt9Il0sWzUsMywiXFxudV97aiwga30iLDJdLFszLDYsIlxcbnVfayJdLFs0LDYsIlxcbnVfaSIsMl0sWzUsNiwiXFxudV9qIl1d
        \[\begin{tikzcd}
            {M_i} & {M_k} & {M_j} \\
            {N_i} & {N_k} & {N_j} \\
            & N
            \arrow["{\mu_{i, k}}", from=1-1, to=1-2]
            \arrow["{\phi_i}"', from=1-1, to=2-1]
            \arrow["{\phi_k}"', from=1-2, to=2-2]
            \arrow["{\mu_{j, k}}"', from=1-3, to=1-2]
            \arrow["{\phi_j}", from=1-3, to=2-3]
            \arrow["{\nu_{i, k}}", from=2-1, to=2-2]
            \arrow["{\nu_i}"', from=2-1, to=3-2]
            \arrow["{\nu_k}", from=2-2, to=3-2]
            \arrow["{\nu_{j, k}}"', from=2-3, to=2-2]
            \arrow["{\nu_j}", from=2-3, to=3-2]
        \end{tikzcd}\]
        \item $\phi$ is a homomorphism: clear
        \item $\phi$ is unique: By definition, the value of $\phi$ on each $\mu_i(x_i)$ is determined by $\phi_i$, and all elements in $M$ is of the form $\mu_i(x_i)$
    \end{enumerate}
\end{proof}

\begin{problem}
    A sequence of direct systems and homomorphisms
    $$\bm{M} \rightarrow \bm{N} \rightarrow \bm{P}$$
    is \textit{exact} if the corresponding sequence of modules and module homomorphisms is exact for each $i \in I$. Show that the sequence $M \rightarrow N \rightarrow P$ of direct limits is then exact.
\end{problem}

\begin{proof}
    By Problem 15, ISTS for every $y_i \in N_i$, $\nu_i(y_i) \in \ker (g)$ if and only if $\nu_i(y_i) \in \mathrm{im}(f)$. In the below arguments, we use $\omega$ to represents homomorphisms related to $P$.

    The proof is an easy exercise of diagram chase of the below commutative diagram. To save matters, I will only show the proof of one direction.

    % https://q.uiver.app/#q=WzAsOSxbMSwwLCJNIl0sWzIsMCwiTiJdLFszLDAsIlAiXSxbMSwxLCJNX2kiXSxbMiwxLCJOX2kiXSxbMywxLCJQX2kiXSxbMiwyLCJOX2oiXSxbNCwyLCJQX2oiXSxbMCwyLCJNX2oiXSxbMywwLCJcXG11X2kiLDFdLFs0LDEsIlxcbnVfaSIsMV0sWzUsMiwiXFxvbWVnYV9pIiwxXSxbMCwxLCJmIl0sWzEsMiwiZyJdLFszLDQsImZfaSIsMl0sWzQsNSwiZ19pIiwyXSxbNCw2LCJcXG51X3tpLCBqfSIsMl0sWzUsNywiXFxvbWVnYV97aSwgan0iLDFdLFszLDgsIlxcbXVfe2ksIGp9IiwxXSxbOCwwLCJcXG11X2oiXSxbNywyLCJcXG9tZWdhX2oiLDJdLFs2LDcsImdfaiIsMl0sWzgsNiwiZl9qIiwyXV0=
    \[\begin{tikzcd}
        & M & N & P \\
        & {M_i} & {N_i} & {P_i} \\
        {M_j} && {N_j} && {P_j}
        \arrow["f", from=1-2, to=1-3]
        \arrow["g", from=1-3, to=1-4]
        \arrow["{\mu_i}"{description}, from=2-2, to=1-2]
        \arrow["{f_i}"', from=2-2, to=2-3]
        \arrow["{\mu_{i, j}}"{description}, from=2-2, to=3-1]
        \arrow["{\nu_i}"{description}, from=2-3, to=1-3]
        \arrow["{g_i}"', from=2-3, to=2-4]
        \arrow["{\nu_{i, j}}"', from=2-3, to=3-3]
        \arrow["{\omega_i}"{description}, from=2-4, to=1-4]
        \arrow["{\omega_{i, j}}"{description}, from=2-4, to=3-5]
        \arrow["{\mu_j}", from=3-1, to=1-2]
        \arrow["{f_j}"', from=3-1, to=3-3]
        \arrow["{g_j}"', from=3-3, to=3-5]
        \arrow["{\omega_j}"', from=3-5, to=1-4]
    \end{tikzcd}\]

    Suppose $g (\nu_i(y_i)) = 0$, then $\omega_i \circ g_i(y_i) = 0$. By Problem 15, there is some $j \ge i$ such that $\omega_{i, j} \circ g_i(y_i) = 0$. It then follows that $g_j \circ \nu_{i, j}(y_i) = 0$. Denote $y_j = \nu_{i, j}(y_i)$. By exactness of $M_j \rightarrow N_j \rightarrow P_j$, there is some $x_j \in M_j$ such that $f_j(x_j) = y_j$. It then follows that $f \circ \mu_j(x_j) = \nu_j \circ f_j(x_j) = \nu_j (y_j) = \nu_j \circ \nu_{i, j}(y_j) = \nu_i(y_i)$.
\end{proof}

\begin{problem}
    Keeping the same notation as in Problem 14, let $N$ be any $A$-module. Then $(M_i \otimes N, \mu_{i, j} \otimes \mathds{1})$ is a direct system. Let $P = \varinjlim (M_i \otimes N)$ be its direct limit. For each $i \in I$ we have a homomorphism $\mu_i \otimes \mathds{1}: M_i \otimes N \rightarrow M \otimes N$, hence by Problem 16 a homomorphism $\psi: P \rightarrow M \otimes N$. Show that $\psi$ is an isomorphism, so that:
    $$\varinjlim (M_i \otimes N) \cong (\varinjlim M_i) \otimes N$$
\end{problem}

\begin{proof}
    For each $i \in I$, let $g_i: M_i \times N \rightarrow M_i \otimes N$ be the canonical bilinear mapping. Fix $y \in N$, $g_{i, y}: M_i \rightarrow M_i \otimes N$ is a linear mapping that satisfies $g_{j, y} \circ \mu_{i, j} = (\mu_{i, j} \otimes \mathds{1}_N) \circ g_{i, y}$. By Problem 18, it defines a homomorphism $g_y: M \rightarrow P$. Since this definition is linear in $y$, it defines a bilinear map $M \times N \rightarrow P$ and hence a homomorphism $\varphi: M \otimes N \rightarrow P$. We claim that $\varphi$ and $\psi$ are inverse to each other. Denote $\overline{\mu_i}$ as the project $M_i \otimes N \rightarrow P$.

    The proof below is just an easy diagram chase of the following commutative diagram:

    % https://q.uiver.app/#q=WzAsNSxbMCwwLCJNX2kiXSxbMiwxLCJNXFxvdGltZXMgTiJdLFsxLDAsIk1faVxcb3RpbWVzIE4iXSxbMCwxLCJNIl0sWzEsMSwiUCJdLFswLDIsImdfe2ksIHl9Il0sWzAsMywiXFxtdV9pIiwyXSxbMiw0LCJcXG92ZXJsaW5le1xcbXVfaX0iXSxbMyw0LCJnX3kiLDJdLFsyLDEsIlxcbXVfaSBcXG90aW1lcyAxX04iXSxbNCwxLCJcXHBzaSIsMl1d
    \[\begin{tikzcd}
        {M_i} & {M_i\otimes N} \\
        M & P & {M\otimes N}
        \arrow["{g_{i, y}}", from=1-1, to=1-2]
        \arrow["{\mu_i}"', from=1-1, to=2-1]
        \arrow["{\overline{\mu_i}}", from=1-2, to=2-2]
        \arrow["{\mu_i \otimes 1_N}", from=1-2, to=2-3]
        \arrow["{g_y}"', from=2-1, to=2-2]
        \arrow["\psi"', from=2-2, to=2-3]
    \end{tikzcd}\]

    \begin{enumerate}
        \item $\varphi \circ \psi = \mathds{1}_P$: ISTS $\varphi \circ \psi(\overline{\mu_i}(x_i \otimes y)) = \overline{\mu_i}(x_i \otimes y)$ for arbitrary $x_i \in M_i, y \in N$. Note that:
        $$
            \begin{aligned}
            \varphi(\psi(\overline{\mu_i}(x_i \otimes y))) &= \varphi(\mu_i \otimes \mathds{1}_N(x_i \otimes y)) = \varphi(\mu_i(x_i) \otimes y) \\
            &= g_y(\mu_i(x_i)) = \overline{\mu_i} (g_{i, y}(x_i))\\
            &= \overline{\mu_i}(x_i \otimes y)
            \end{aligned}
        $$
        \item $\psi \circ \varphi = \mathds{1}_{M \otimes N}$: ISTS $\psi \circ \varphi(x \otimes y) = x \otimes y$ for arbitrary $x \in M, y \in N$. Suppose $x = \mu_i(x_i)$, note that:
        $$
            \begin{aligned}
            \psi \circ \varphi(x \otimes y) &= \psi (g_y(x)) = \psi (g_y(\mu_i(x_i))) \\
            &= \psi(\overline{u_i}(g_{i, y}(x))) = (\mu_i \otimes \mathds{1}_N)(g_{i, y}(x_i)) \\
            &= (\mu_i \otimes \mathds{1}_N)(x_i \otimes y) = \mu_i(x_i) \otimes y = x \otimes y
            \end{aligned}
        $$
    \end{enumerate}
\end{proof}

\begin{problem}
    Let $(A_i)_{i \in I}$ be a family of rings indexed by a directed set $I$, and for each pair $i \le j$ in $I$ let $\alpha_{i, j}: A_i \rightarrow A_j$ be a ring homomorphism, satisfying conditions 1 and 2 of Problem 14. Regarding each $A_i$ as a $\mathbb{Z}$-module we can then form the direct limit $A = \varinjlim A_i$. Show that $A$ inherits a ring structure from the $A_i$ so that the mappings $A_i \rightarrow A$ are ring homomorphisms. The ring $A$ is the \textit{direct limit} of the system $(A_i, \alpha_{i, j})$.

    If $A = 0$, prove that $A_i = 0$ for some $i \in I$.
\end{problem}

\begin{proof}
    Denote $\alpha_i: A_i \rightarrow A$ the projection as in Problem 14. By Problem 15, we only need to define the product of $\alpha_i(a_i), \alpha_j(a_j)$ for some $a_i \in A_i$ and $a_j \in A_j$: Find $k \ge i, j$, then $\alpha_i(a_i) = \alpha_k(\alpha_{i, k}(a_i)), \alpha_j(a_j) = \alpha_k(\alpha_{j, k}(a_j))$, now define $$\alpha_i(a_i) \alpha_j(a_j) = \alpha_k(\alpha_{i, k}(a_i) \alpha_{j, k}(a_j))$$
    
    We claim that this is well-defined:
    \begin{enumerate}
        \item It is irrelevant to the selection of $k$: Let $k'$ be another selection, and suppose $l \ge k, k'$. We have:
        $$
            \begin{aligned}
            \alpha_{k'}(\alpha_{i, k'}(a_i) \alpha_{j, k'}(a_j))
            &= \alpha_{l} \circ \alpha_{k', l}(\alpha_{i, k'}(a_i) \alpha_{j, k'}(a_j)) \\
            &= \alpha_l (\alpha_{i, l}(a_i) \alpha_{j, l}(a_j)) \\
            &= \alpha_{k}(\alpha_{i, k}(a_i) \alpha_{j, k}(a_j))
            \end{aligned}
        $$
        where step 1 is by the property of $\alpha_l$(as $\mathbb{Z}$-module homomorphism, and hence as a map), step 2 follows since $\alpha_{i, j}$'s are \textbf{ring} homomorphisms, and step 3 is by symmetry.
        \item It is irrelevant to the selection of representatives $i, j$. WLOG, we only show that it is irrelevant to the selection of $i$. Suppose $\alpha_{i}(a_i) = \alpha_{i'}(a_{i'})$, then by part 1 and the same argument as in Problem 16, we may assume $k \ge i, i', j$ and $\alpha_{i, k}(a_i) = \alpha_{i', k}(a_{i'})$, then
        $$\alpha_{k}(\alpha_{i, k}(a_i) \alpha_{j, k}(a_j)) = \alpha_{k}(\alpha_{i', k}(a_{i'})\alpha_{j, k}(a_j))$$
    \end{enumerate}

    $\alpha_i$ is a ring homomorphism: That $\alpha_i$ preserves multiplication follows directly from the definition by setting $i = j = k$. Also, $\alpha_i$ preserves: Take arbitrary $a_j$, as before, let $k \ge i, j$, we have:
    $$\alpha_{i}(1) \alpha_j(a_j) = \alpha_k(\alpha_{i, k}(1) \alpha_{j, k}(a_j)) = \alpha_k(\alpha_{j, k}(a_j)) = \alpha_j(a_j)$$
    since $\alpha_{i, k}$ is a ring homomorphism and hence preserves $1$.

    If $A = 0$, then $1_A = 0_A$, as a result, $\alpha_{i}(1_{A_i}) = 0$. By Problem 15, $\alpha_{i, j}(1_{A_i}) = 1_{A_{j}} = 0 \Rightarrow A_j = 0$
\end{proof}

{\color{red} Is the last proposition holds for the direct limit of modules? The proof clearly does not hold anymore.}

\begin{problem}
    Let $(A_i, \alpha_{i, j})$ be a direct system of rings and let $\mathfrak{N}_i$ be the nilradical of $A_i$. Show that $\varinjlim \mathfrak{N}_{i}$ is the nilradical of $\varinjlim A_i$.

    If each $A_i$ is an integral domain, then $\varinjlim A_i$ is an integral domain.
\end{problem}

\begin{proof}
    Denote $\mathfrak{N} = \varinjlim \mathfrak{N}_i$. Note that $\mathfrak{N}_i$ is a $\mathbb{Z}$-submodule of $A_i$, and therefore $\varinjlim \mathfrak{N}_i$ shares the same projection $\alpha_i: \mathfrak{N}_i \rightarrow \mathfrak{N}$ as $A_i \rightarrow A$.

    $\mathfrak{N} \subset \mathfrak{N}_A$: Take arbitrary $\alpha_i(n_i) \in \mathfrak{N}$ where $n_i \in \mathfrak{N}_i$. Then $n_i^m = 0$ for some $m \gt 0$. Since $\alpha_i$ is a ring homomorphism, we have $\alpha_i(n_i)^m = 0$ and hence $\alpha_i(n_i) \in \mathfrak{N}_A$.

    $\mathfrak{N}_A \subset \mathfrak{N}$: Take arbitrary $\alpha_i(n_i) \in \mathfrak{N}_A$, then we have $\alpha_i(n_i)^m = 0$ for some $m \gt 0$. Since $\alpha_i$ is a ring homomorphism, we have $\alpha_i(n_i^m) = 0$. By Problem 15, this implies $\alpha_{i, j}(n_i^m) = 0$ for some $j \ge i$. Since $\alpha_{i, j}$ is a ring homomorphism, we have $n_j^m = 0$ where $n_j = \alpha_{i, j}(n_i)$, namely $n_j \in \mathfrak{N}_j$. But then $\alpha_i(n_i) = \alpha_j(n_j) \in \mathfrak{N}$.

    For the second part, ISTS if $a \ne 0$ is a zero-divisor of $A$, then $a = \alpha_i(a_i)$ for some zero-divisor $a_i \in A_i$ (it has to be nonzero since $a \ne 0$). Suppose $ab = 0$ for $b \ne 0, b \in A$, WLOG, we may take $k$ such that $a = \alpha_k(a_k)$ and $b = \alpha_k(b_k)$ and $ab = \alpha_k(a_kb_k) = 0$, then by Problem 15, we have $\alpha_{k, l}(a_kb_k) = 0$. But this implies $a_lb_l = 0$ where $a_l = \alpha_{k, l}(a_k), b_l = \alpha_{k, l}(b_k)$. Since $\alpha_l(a_l) = a, \alpha_l(b_l) = b$, we must have $a_l \ne 0, b_l \ne 0$, which completes the proof.
\end{proof}

\begin{problem}
    Let $(B_\lambda)_{\lambda \in \Lambda}$ be a family of $A$-algebras. For each finite subset $J$ of $\Lambda$ let $B_J$ denote the tensor product (over $A$) of the $B_\lambda$ for $\lambda \in J$. If $J'$ is another finite subset of $\Lambda$ and $J \subset J'$, there is a canonical $A$-algebra homomorphism $B_J \rightarrow B_{J'}$. Let $B$ denote the direct limit of the rings $B_J$ as $J$ runs through all finite subsets of $\Lambda$. The ring $B$ has a natural $A$-algebra structure for which the homomorphisms $B_J \rightarrow B$ are $A$-algebra homomorphisms. The $A$-algebra $B$ is the tensor product of the family $(B_\lambda)_{\lambda \in \Lambda}$
\end{problem}

\begin{proof}
    The canonical homomorphism $B_J \rightarrow B_{J'}$ is induced by the "$J$-linear" map $(x_j)_{j \in J} \mapsto \left(\otimes_{j \in J} (x_j)\right) \otimes \left(\otimes_{j \notin J} (1)\right)$. It is easy to verify that the homomorphism satisfies condition (1) and (2) in Problem 14 with the order on $\mathcal{P}(\Lambda)$ defined by inclusion. Note that $B_J$'s are equipped with an $A$-algebra structure (see the content on Page 31 of Atiyah's book). Then it is easy to verify that the $A$-module structure of $B$ is compatible with the ring structure of $B$.
\end{proof}

{\color{red} It was when I tried to solve Problem 23 that I realize that Atiyah got the homomorphism $A \rightarrow B \otimes_A C$ wrong on Page 31 of his book. Since the $A$-module structure of $B \otimes_A C$ is clear, we can recover the homomorphism $A \rightarrow B \otimes_A C$ by considering the action on $1 \otimes 1$. Then the homomorphism would be $a \mapsto f(a) \otimes 1 = 1 \otimes g(a)$ instead of $a \mapsto f(a) \otimes g(a)$}

\begin{problem}
    If $M$ is an $A$-module, the followings are equivalent:
    \begin{enumerate}
        \item $M$ is flat.
        \item $\mathrm{Tor}_{n}^{A} = 0$ for all $n \gt 0$ and all $A$-modules $N$.
        \item $\mathrm{Tor}_{1}^{A}(M, N) = 0$ for all $A$-modules $N$
    \end{enumerate}
\end{problem}

\begin{proof}
    (1) $\Rightarrow$ (2): Take a projective resolution of $N$:
    $$\rightarrow P_1 \rightarrow P_1 \rightarrow N \rightarrow 0$$
    Since $M$ is flat, we can tensor the long exact sequence with $M$ and get:
    $$\rightarrow M \otimes_A P_1 \rightarrow M \otimes_A P_1 \rightarrow M \otimes_A N \rightarrow 0$$
    Note that $\mathrm{Tor}_{n}^{A}$ is the homology at $P_{n} \otimes M$ for $n \gt 0$, since the above sequence is exact, we have $\mathrm{Tor}_{n}^{A}(M, N) = 0$

    (2) $\Rightarrow$ (3): Trivial.

    (3) $\Rightarrow$ (1): Take arbitrary exact sequence $0 \rightarrow N' \rightarrow N \rightarrow N'' \rightarrow 0$, we have the Tor exact sequence:
    $$\mathrm{Tor}_{1}^{A}(M, N) \rightarrow M \otimes_A N' \rightarrow M \otimes_A N \rightarrow M \otimes_A N'' \rightarrow 0$$
    since $\mathrm{Tor}_{0}^{A}(M, P) = M \otimes_A P$. Then $\mathrm{Tor}_{1}^{A}(M, N) = 0$ implies that the above sequence is a short exact one. It follows that $M$ is flat.
\end{proof}

\begin{problem}
    Let $0 \rightarrow N' \rightarrow N \rightarrow N'' \rightarrow 0$ be an exact sequence, with $N''$ flat. Then $N'$ is flat $\Leftrightarrow$ $N$ is flat.
\end{problem}

\begin{proof}
    Take arbitrary $A$-module $P$, consider the Tor exact sequence:
    $$
        \begin{aligned}
        \cdots \rightarrow \mathrm{Tor}_{2}^{A}(P, N'') &\rightarrow \mathrm{Tor}_{1}^{A}(P, N') \rightarrow \mathrm{Tor}_{1}^{A}(P, N) \rightarrow \mathrm{Tor}_{1}^{A}(P, N'') \\
        &\rightarrow P \otimes_A N' \rightarrow P \otimes_A N \rightarrow P \otimes_A N'' \rightarrow 0
        \end{aligned}
    $$
    Then the statement is clear by the exactness of the first line and Problem 24.
\end{proof}

\begin{problem}
    Let $N$ be an $A$-module. Then $N$ is flat $\Leftrightarrow$ $\mathrm{Tor}_{1}^{A}(A / I, N) = 0$ for all finitely generated ideals $I$ in $A$
\end{problem}

\begin{proof}
    Note that by Proposition 2.19, $N$ is flat if and only if tensoring with $N$ preserves injective maps between finitely generated modules. By the Tor exact sequence, this is equivalent to $\mathrm{Tor}_{1}^{A}(M, N) = 0$ for all finitely generated module $M$ (For injective $M' \rightarrow M$ where $M, M'$ finitely generated, we always have exact sequence $0 \rightarrow M' \rightarrow M \rightarrow M / M' \rightarrow 0$ and $M / M'$ is finitely generated).

    Suppose $M$ is generated by $x_1, \cdots, x_n$, denote $M_i = (x_1, \cdots, x_i)$, then we have the exact sequence:
    $$0 \rightarrow M_i \rightarrow M_{i + 1} \rightarrow M_{i + 1} / M_{i} \rightarrow 0$$
    Then consider the Tor exact sequence:
    $$\mathrm{Tor}_{1}^{A}(M_i, N) \rightarrow \mathrm{Tor}_{1}^{A}(M_{i + 1}, N) \rightarrow \mathrm{Tor}_{1}^{A}(M_{i + 1} / M_i, N) \rightarrow M_i \otimes_A N$$
    If $\mathrm{Tor}_{1}^{A}(M, N) = 0$ for all modules generated by less than $i$ element, by the exactness of the above sequence, we also have $\mathrm{Tor}_{1}^{A}(M_{i + 1}, N) = 0$. By induction, $N$ is flat if and only if $\mathrm{Tor}_{1}^{A}(M, N) = 0$ for modules $M$ generated by one element.
    
    Note that if $M$ is generated by a single element $m$, then $r \rightarrow rm$ defines a surjective map $A \rightarrow M$, and hence induces an isomorphism $A / \mathrm{Ann}(m) \cong M$. Namely, $N$ is flat if and only if $\mathrm{Tor}_{1}^{A}(A / I, M) = 0$ for all ideals $I$. But this is equivalent to $\otimes_A M$ preserves the exactness of:
    $$0 \rightarrow I \rightarrow A \rightarrow A / I \rightarrow 0$$
    By similar argument as in Proposition 2.19, we may assume $I$ is finitely generated.
\end{proof}

\begin{problem}
    A ring $A$ is \textit{absolutely flat} if every $A$-module is flat. Prove that the followings are equivalent:
    \begin{enumerate}
        \item $A$ is absolutely flat.
        \item Every principal ideal is idempotent.
        \item Every finitely generated ideal is a direct summand of $A$
    \end{enumerate}
\end{problem}

\begin{proof}
    (1) $\Rightarrow$ (2): Let $x \in A$, note that $(x)^2 = (x) \Leftrightarrow (x) / (x)^2 = 0 \Leftrightarrow (x) \otimes_A A / (x) = 0$ by Problem 2. Consider the homomorphism $i \otimes_A \mathds{1}_{A / (x)}: (x) \otimes_A A / (x) \rightarrow A \otimes_A A / (x) \cong A / (x)$ where $i$ is the inclusion. Then for all $ax \otimes_A \overline{b} \in (x) \otimes_A A / (x)$, we have $ax \otimes_A \overline{b} = 1\otimes_A \overline{abx} = 0$ in $A \otimes_A A / (x)$. As a result, $i \otimes_A \mathds{1}_{A / (x)} = 0$. However, since $A / (x)$ is flat, $i \otimes_A \mathds{1}_{A / (x)}$ is injective, which completes the proof that $(x) \otimes_A A / (x) = 0$

    (2) $\Rightarrow$ (3): Just follow the hint: By hypothesis, for all $x \in A$, there is some $a \in A$ such that $x = ax^2$, then $ax$ would be an idempotent element. Moreover, $x = ax^2 = (ax)x \Rightarrow x \in (ax)$, therefore $(x) = (ax)$. As a result, take any finitely generated $A$-module $M = (x_1, \cdots, x_n)$. We may assume $x_i$'s are idempotent. Then by similar arguments as in Problem 11 of Chapter 1, $M$ is principal $(e)$ (we may also assume $e$ is idempotent). Then clearly $M$ is a direct summand of $A$ since $A = (e) \oplus (1 - e)$

    (3) $\Rightarrow$ (1): By Problem 26, ISTS for any $A$-module $M$, tensoring $M$ preserves the exactness of:
    $$0 \rightarrow I \rightarrow A \rightarrow A / I \rightarrow 0$$
    for all finitely generated ideal $I$ of $A$. Since tensor product is right exact, ISTS $I \otimes_A M \rightarrow A \otimes_A M$ is still right exact. Since $A = I \oplus J$ for some $J$ by the hypothesis, we have:
    $$A \otimes_A M = (I \otimes_A M) \oplus (J \otimes_A M)$$
    and clearly $(I \otimes_A M) \rightarrow (I \otimes_A M) \oplus (J \otimes_A M)$ is injective.
\end{proof}

\begin{problem}
    A Boolean ring is absolutely flat. The ring of Chapter 1, Problem 7 is absolutely flat (For all $x$, $x^n = x$ for some $n \gt 1$). Every homomorphic image of an absolutely flat ring is absolutely flat. If a local ring is absolutely flat, then it is a field.

    If $A$ is absolutely flat, every non-unit in $A$ is a zero-divisor.
\end{problem}

\begin{proof}
    A Boolean ring is absolutely flat: This follows from Problem 27 and Problem 11 of Chapter 1.

    The ring of Chapter 1, Problem 7 is absolutely flat: For all $x \in A$, we have $x = x^{n - 2} x^2 \Rightarrow x \in (x^2)$. Conclude by part 2 of Problem 27.

    Every homomorphic image of an absolutely flat ring is absolutely flat: The homomorphic image can be regarded as a quotient of the original ring. It follows easily from Problem 27, the equivalence of 1 and 2.

    If a local ring is absolutely flat, then it is a field: Take arbitrary non-unit $x$, by part 2 of Problem 27, $x = ax^2$ for some $a$ $\Rightarrow x(1 - ax) = 0$. But since $x \in \mathfrak{m}$ the maximal ideal of $A$, it also belongs to the Jacobson radical of $A$ (as there is only one maximal ideal). It follows that $1 - ax$ is a unit (by Proposition 1.9) and $x = 0$. As a result, $A$ is a field.

    If $A$ is absolutely flat, every non-unit in $A$ is a zero-divisor: Argue as before, $x(1 - ax) = 0$. Since $x$ is non-unit, $1 - ax \ne 0$. Then $x$ is a zero-divisor.
\end{proof}

\end{document}