\documentclass{solution}

\begin{document}

\begin{problem}
    Let $f \in k[X_1, \cdots, X_n]$ be an irreducible polynomial over an algebraically closed field $k$. A point $P$ on the variety $f(x) = 0$ is \textit{non-singular} $\Leftrightarrow$ not all the partial derivatives $\frac{\partial f}{\partial X_i}$ vanish at $P$. Let $A = k[X_1, \cdots, X_n] / (f)$, and let $\mathfrak{m}$ be the maximal ideal of $A$ corresponding to the point $P$. Prove that $P$ is non-singular $\Leftrightarrow$ $A_{\mathfrak{m}}$ is a regular local ring.
\end{problem}

\begin{proof}
    $A_{\mathfrak{m}}$ is clearly local. It is Noetherian since $A$ is (by Proposition 7.3). By Corollary 11.18 and 11.27, $\dim A_{\mathfrak{m}} = \dim A = n - 1$. ISTS $\dim_k(\mathfrak{m} / \mathfrak{m}^2) = n - 1$ if and only if $f$ has non-vanishing derivatives at $P$ by Theorem 11.22. WLOG, we may assume $P = 0$. Then we have $\mathfrak{m} = (X_1, \cdots, X_n) / (f)$. and $\mathfrak{m}^2 = ((X_1, \cdots, X_n)^2 + (f)) / (f)$. As a result:
    $$\mathfrak{m} / \mathfrak{m}^2 = (X_1, \cdots, X_n) / ((X_1, \cdots, X_n)^2 + (f))$$
    If $f$ is singular at $P$, then all derivatives of $f$ vanishes at $P = 0 \Rightarrow f \in (X_1, \cdots, X_n)^2 \Rightarrow \dim_k \mathfrak{m} / \mathfrak{m}^2 = \dim_k (X_1, \cdots, X_n) / (X_1, \cdots, X_n)^2 = n$. If $f$ is non-singular at $P$, then $f = \sum\limits_{i = 1}^{n} c_i X_i + \text{ higher terms}$ where $c_i \ne 0$ for some $i$. Suppose $c_1 \ne 0$, then $X_2, \cdots, X_n$ forms a basis of $\mathfrak{m} / \mathfrak{m}^2$ and hence $\dim_k \mathfrak{m} / \mathfrak{m}^2 = n - 1$.
\end{proof}

\begin{problem}
    In (11.21) assume that $A$ is complete. Prove that the homomorphism $k[[T_1, \cdots, T_d]] \rightarrow A$ given by $T_i \mapsto x_i$ is injective and that $A$ is a finitely-generated module over $k[[T_1, \cdots, T_d]]$
\end{problem}

\begin{proof}
    Suppose $x_1, \cdots, x_n$ generates $\mathfrak{m}$-primary $\mathfrak{q}$. Note that a monomial of degree $m$ of $x_1, \cdots, x_n$ is contained in $\mathfrak{q}^m$, so we clearly have homomorphism $k[[T_1, \cdots, T_d]] \rightarrow \hat{A}$ (where $\hat{A}$ is the completion of $A$ for the $\mathfrak{q}$-topology. But since $A$ is Noetherian, $\mathfrak{m}^r \subset \mathfrak{q} \subset \mathfrak{m}$, so the $\mathfrak{q}$-topology is equivalent to the $\mathfrak{m}$-topology and hence $\hat{A}$ is also the completion of $A$ for the $\mathfrak{m}$-topology) defined by $f \mapsto (a_m)$ where $a_m = f_0(x_1, \cdots, x_d) + f_1(x_1, \cdots, x_d) + \cdots + f_{m - 1}(x_1, \cdots, x_d) + \mathfrak{q}^{m}$ where $f_i$'s are the homogeneous component of $f$. Since $A$ is complete, $A \cong \hat{A}$ and so the map $\varphi: k[[T_1, \cdots, T_d]] \rightarrow A: T_i \mapsto x_i$ is well-defined.

    Now consider polynomial $f$ that maps to $0$. We must have $a_m = 0$ for all $m \ge 1$. Apply Proposition 11.20 to each $f_{m}$, the coefficients of $f_m$ must be $\mathfrak{m}$. But the coefficients is also in $k$, then we must have $f_m = 0$, so $f = 0$. This proves that the map $\varphi$ is injective.

    Note that the maximal ideal of $k[[T_1, \cdots, T_d]]$ is $\mathfrak{n} = (T_1, \cdots, T_d)$, which extends to $\mathfrak{q}$. So $A$ is Hausdorff for the $\mathfrak{n}$-filtration topology (which is exactly the $\mathfrak{q}$-topology, then by completeness). Also, it is clear that $k[[T_1, \cdots, T_d]]$ is complete for the $\mathfrak{n}$-topology. By Proposition 10.24, ISTS $G(A)$ (with respect to the $\mathfrak{n}$-filtration $\mathfrak{q}^n$) is finitely generated over $G_{\mathfrak{n}}(k[[T_1, \cdots, T_d]])$.
    
    Consider the $m$-th component of the map $G_{\mathfrak{n}}(k[[T_1, \cdots, T_n]]) \rightarrow G(A)$, we have:
    $$\mathfrak{n}^m / \mathfrak{n}^{m + 1} \rightarrow \mathfrak{q}^m / \mathfrak{q}^{m + 1}$$
    If $m \gt 0$, simply note that $\mathfrak{q}^m = (\mathfrak{n}^e)^m = (\mathfrak{n}^m)^e = A (\mathfrak{n}^m)$, so $\mathfrak{q}^m / \mathfrak{q}^{m + 1}$ is generated by $A / \mathfrak{q}$ as $G_{\mathfrak{n}}(k[[T_1, \cdots, T_d]])$-module. So ISTS $A / \mathfrak{q}$ is finitely generated as $G_{\mathfrak{n}}(k[[T_1, \cdots, T_d]])$-module, by the graded property, this is equivalent to $A / \mathfrak{q}$ finitely generated as $k$-vector space.

    Take arbitrary element $x \in A$, we have $x = b_0 + m_1$ for some $b_1 \in k, m_1 \in \mathfrak{m}$ by $k \cong A / \mathfrak{m}$. Let $m_{1, i}$ be a set of generators for $\mathfrak{m}$ as an ideal of $A$ (by $A$ Noetherian, the set of generators can be finite, say $n_1$ of them). Then we have $m_1 = \sum\limits_{i = 1}^{n_1} x_{1, i} m_{1, i}$ for $x_{1, i} \in A$. By the same argument, we have $x_{1, i} = b_{1, i} + m_{2, i}$ for some $b_{1, i} \in k, m_{2, i} \in \mathfrak{m}$. Then we have: $x = b_0 + \sum\limits_{i = 1}^{n_1} b_{1, i} m_{1, i} + m_2$ where $m_2 \in \mathfrak{m}^2$. Continue the process, suppose $\mathfrak{q} \supset \mathfrak{m}^r$, then we have:
    $$\overline{x} = \overline{b_0 + \sum\limits_{i = 1}^{r - 1} \sum\limits_{j = 1}^{n_i} b_{i, j} m_{i, j}}$$
    in $A / \mathfrak{q}$, which shows that $1$ together with the generators of $\mathfrak{m}^i, i = 1, \cdots, r - 1$ generates $A / \mathfrak{q}$ as $k$-vector space.
\end{proof}

\begin{problem}
    Extend (11.25) to non-algebraically-closed fields.
\end{problem}

\begin{proof}
    The extension: For any irreducible variety $V$ over $k$ (not necessarily algebraically closed), the local dimension of $V$ at any point is equal to $\dim V$

    Let $\Omega = \overline{k}$ be the algebraic closure of $k$. Then since $X_1, \cdots, X_n$ and every element in $\overline{k}$ is integral over $k[X_1, \cdots, X_n]$ and the set of integral elements form a subring, $\overline{k}[X_1, \cdots, X_n]$ is integral over $k[X_1, \cdots, X_n]$.

    Then the proof is basically the same as Proposition 11.25, where $B$ is taken as $\overline{k}[X_1, \cdots, X_d]$.
\end{proof}

\begin{problem}
    An example of a Noetherian domain of infinite dimension (Nagata). Let $k$ be a field and let $A = k[X_1, X_2, \cdots, X_n]$ be a polynomial ring over $k$ in a countably infinite set of indeterminates. Let $m_1, m_2, \cdots$ be an increasing sequence of positive integers such that $m_{i + 1} - m_i \gt m_i - m_{i - 1}$ for all $i \gt 1$. Let $\mathfrak{p}_i = (X_{m_i + 1}, \cdots, X_{m_{i + 1}})$ and let $S$ be the complement in $A$ of the union of the ideals $\mathfrak{p}_i$.

    Each $\mathfrak{p}_i$ is a prime ideal and therefore the set $S$ is multiplicatively closed. The ring $S ^{-1}A$ is Noetherian by Chapter 7, Problem 9. Each $S ^{-1} \mathfrak{p}_i$ has height equal to $m_{i + 1} - m_i$, hence $\dim S ^{-1} A = \infty$.
\end{problem}

\begin{proof}
    "$S ^{-1} A$ is Noetherian": By Chapter 7, Problem 9, ISTS:
    \begin{enumerate}
        \item For every maximal ideal $\mathfrak{n}$ of $A$, $A_{\mathfrak{n}}$ is Noetherian: First let's note that prime ideals in $S ^{-1} A$ corresponds to prime ideals in $A$ that avoids $S$ by Proposition 3.11. But $I \cap S = \emptyset$ $\Leftrightarrow$ $I \subset \bigcup\limits_{i = 1}^{\infty} \mathfrak{p}_i$. Check that $I \subset \mathfrak{p}_i$ for some $i$(not hard but a bit tedious). It follows that $S ^{-1} \mathfrak{p}_i$'s are the maximal ideals in $S ^{-1} A$. Note that $S ^{-1} A$ localized at $S ^{-1} \mathfrak{p}_i$ is equivalent to $A$ localized at $A - \mathfrak{p}_i$ by Chapter 3, Problem 4, which is $A_{\mathfrak{p}_i} \cong k'[X_{m_{i} + 1}, \cdots, X_{m_{i + 1}}]$ where $k' = k(X_i: i \notin [m_{i} + 1, m_{i + 1}])$, a Noetherian ring.
        \item For every $x \in S ^{-1} A$, there are only finitely many $S ^{-1} \mathfrak{p}_i$ that contains $x$: Suppose $x = a / s$, then $x \in S ^{-1} \mathfrak{p}_i \Leftrightarrow a / 1\in S ^{-1} \mathfrak{p}_i \Leftrightarrow a \in (S ^{-1} \mathfrak{p}_i )^c = \mathfrak{p}_i$. Clearly there are only finitely many $\mathfrak{p}_i$ that contains $a$.
    \end{enumerate}

    "$S ^{-1} \mathfrak{p}_i$ has height $m_{i + 1} - m_i$": Simply note that $\mathrm{ht} (S ^{-1} \mathfrak{p}_i) = \dim (S ^{-1} A)_{S ^{-1} \mathfrak{p}_i} = \dim A_{\mathfrak{p}_i} = \dim k'[X_{m_{i} + 1}, \cdots, X_{m_{i + 1}}] = m_{i + 1} - m_i$.
\end{proof}

\begin{problem}
    Reformulate (11.1) in terms of the Grothendieck group $K(A_0)$ (Chapter 7, Problem 26)
\end{problem}

\begin{proof}
    By the universal property of Grothendieck group (part 1 of Problem 26 in Chapter 7), the additive function $\lambda$ from $A_0$-modules to $\mathbb{Z}$ is in a one-to-one correspondence to group homomorphisms: $K(A_0) \rightarrow \mathbb{Z}$. Let $A$ be a graded ring with $A_0$ as the first components such that $A$ is finitely generated over $A_0$ (equivalent to $A$ Noetherian by Proposition 10.7). Let $K_{gr}(A)$ be the 'graded' Grothendieck group of $A$, namely the graded finitely generated module $M$ over $A$ module out the isomorphism and the relation defined by exact sequence. Then by $\lambda$ we define a homomorphism $\overline{\lambda}: K_{gr}(A) \rightarrow \mathbb{Z}[[T]]$, the Poincare series. What Proposition 11.1 states is that the image of the $\overline{\lambda}$ lies in $S ^{-1} \mathbb{Z}[T]$ where $S$ is the multiplicative subset of $\mathbb{Z}[T]$ consists of all elements of the form $\prod\limits_{i = 1}^{n} (1 - T^{n_i})$
\end{proof}

{\color{red} I am not particularly satisfied by the results in Problem 5, I feel like the author means more than that because my reformulation does not correspond exactly to the original one.}

\begin{problem}
    Let $A$ be a ring (not necessarily Noetherian). Prove that
    $$1 + \dim A \le \dim A[X] \le 1 + 2 \dim A$$
\end{problem}

\begin{proof}
    Follow the hint: Consider the embedding $f: A \rightarrow A[X]$. Let $\mathfrak{p}_1 \subsetneq \mathfrak{p}_2 \subsetneq \cdots \subsetneq \mathfrak{p}_d$ be an arbitrary chain of primes in $A$, then by Problem 7, part 2 in Chapter 4, their extensions
    $$\mathfrak{p}_1[X] \subsetneq \mathfrak{p}_2 [X] \subsetneq \cdots \subsetneq \mathfrak{p}_d[X]$$
    is a chain of primes. Now consider the fiber of $\mathfrak{p}_1$ in $A[X]$. Since $\mathrm{Spec}(f) ^{-1} (\mathfrak{p}_1) \cong k(\mathfrak{p}_1) \otimes_A A[X] \cong k(\mathfrak{p}_1) \otimes \bigoplus\limits_{i = 1}^{\infty} A \cong \bigoplus\limits_{i = 1}^{\infty} k(\mathfrak{p}_1) \cong k(\mathfrak{p}_1)[X]$, which has dimension $1$ as it is a PID. Note that $\mathfrak{p}_1^e = \mathfrak{p}_1[X]$ corresponds to the maximal element in $\mathrm{Spec}(f) ^{-1} (\mathfrak{p}_1)$, so there must be some $\mathfrak{q} \in \mathrm{Spec}(A[X])$ such that $\mathfrak{q} \subsetneq \mathfrak{p}_1[X]$, which proves that $\dim A[X] \ge 1 + \dim A$

    For the other side, consider a chain of primes $\mathfrak{q}_1 \subsetneq \mathfrak{q}_2 \subsetneq \cdots \subsetneq \mathfrak{q}_d$ in $A[X]$, let $\mathfrak{p}_i = \mathfrak{q}_i \cap A$ be their contractions in $A$, which are all primes. By our previous argument, at most two consecutive primes can contract to the $\mathfrak{p}_i$. So after we delete all the redundant elements from the chain, there are at least $d - \lfloor d / 2 \rfloor$ primes, which proves that $\dim A[X] \ge 1 + 2 \dim A$.

    Note that both proves holds for the case where $\dim A[X] = \infty$ or $\dim A = \infty$, as we are taking supreme of $d$.
\end{proof}

\begin{problem}
    If $A$ is a Noetherian ring. Then:
    $$\dim A[X] = 1 + \dim A$$
    and hence, by induction on $n$,
    $$\dim A[X_1, \cdots, X_n] = n + \dim A$$
\end{problem}

\begin{proof}
    Follow the hint, first we prove that $\mathrm{ht}(\mathfrak{p}) = \mathrm{ht}(\mathfrak{p}[X])$ for arbitrary prime ideal $\mathfrak{p} \in \mathrm{Spec}(A)$. By our previous argument in Problem 6, it is clear that $\mathrm{ht}(\mathfrak{p}) \le \mathrm{ht}(\mathfrak{p}[X])$. For the other direction, consider $A_{\mathfrak{p}}$, which is a Noetherian local ring with maximal ideal $\mathfrak{p} A_{\mathfrak{p}}$. Suppose $\mathrm{ht}(\mathfrak{p}) = \dim A_{\mathfrak{p}} = m$. By Theorem 11.14, there is $a_1, \cdots, a_m \in A$ (first take them in $A_{\mathfrak{p}}$, then discover that we can remove the denominators) such that their images generate a $\mathfrak{p} A_{\mathfrak{p}}$-primary ideal in $A_{\mathfrak{p}}$. By Proposition 3.11, we can write it as $\mathfrak{q}^e$, and by Proposition 4.8, its contraction in $A$ is $\mathfrak{q}$, which is $\mathfrak{p}$-primary. It follows that $a_1, \cdots, a_m \in \mathfrak{q}$ and generates $\mathfrak{q}$. It is clear (by taking radicals) that $\mathfrak{p} A_{\mathfrak{p}}$ is a minimal prime over $\mathfrak{q}^e$ and by the correspondence between prime ideals of $A$ and $A_{\mathfrak{p}}$, $\mathfrak{p}$ is a minimal prime over $\mathfrak{q}$. Now by part 5 of Problem 7 in Chapter 4, $\mathfrak{p}[X]$ is minimal over $\mathfrak{q}[X]$, localize $A[X]$ at $\mathfrak{p}[X]$ to derive $\mathrm{ht}(\mathfrak{p}[X]) \le m$.

    Now consider any maximal ideal $\mathfrak{n}$ in $A[X]$. We have $\mathfrak{p} = \mathfrak{n} \cap A$ a prime ideal in $A$. Then $\mathfrak{p}[X] = \mathfrak{n}^{ce} \subset \mathfrak{n}$. By our argument in Problem 6, there is no prime ideals between $\mathfrak{n}$ and $\mathfrak{p}[X]$. This proves that $\mathrm{ht}(\mathfrak{n}) \le \mathrm{ht}(\mathfrak{p}[X]) + 1 = \mathrm{ht}(\mathfrak{p}) + 1$. Taking supreme over $\mathfrak{n}$ will conclude the proof.
\end{proof}

\end{document}