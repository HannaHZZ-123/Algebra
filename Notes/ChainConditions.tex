\documentclass{note-eng}

\title{Category}
\author{Jingyi Long}


\begin{document}

\maketitle
\tableofcontents

Previously, we have given a series of increasing refinements from domain to a field. However, even domains are uncommon in general, so the properties of those rings are hard to apply. In this chapter, we introduce a special type of conditions on rings that is more common and easy to apply.

\newpage

\section{Chain Conditions}

We start with chain conditions for arbitrary poset:

\begin{definition}[Chain Condition]
    Let $(I, \le)$ be a partially ordered set. If every chain of elements
    $$x_0 \le x_1 \le x_2 \le \cdots$$
    of $I$ \textbf{stabilizes}, namely there is $n_0 \gt 0$ such that $x_n = x_{n_0}$ for all $n \ge n_0$, then we call $(I, \le)$ satisfies the \textbf{chain condition}.
\end{definition}

\begin{proposition}[Equivalent Definition of Chain Condition] \label{prop:equiv-chain}
    Let $(I, \le)$ be a partially ordered set, then TFAE:
    \begin{enumerate}
        \item $(I, \le)$ satisfies the chain condition;
        \item Every non-empty subset of $I$ has a maximal element.
    \end{enumerate}
\end{proposition}

\begin{proof}
    $(1) \Rightarrow (2)$: Let $S$ be a non-empty subset of $I$, suppose it does not contain a maximal element, we can construct inductively an infinite chain of elements that does not stabilize, which contradict $(1)$.

    $(2) \Rightarrow (1)$: Let the set be the chain of elements.
\end{proof}

\begin{remark}
    The reader may find the proposition above quite similar to Zorn's Lemma. However, Zorn's lemma requires much less: It only requires any chain to be bounded above, while chain condition requires any chain to be \textit{finite}, which is clearly stronger.

    That being said, the reader would expect that our proof of the previous proposition does not appeal to Zorn's lemma or its equivalent: the Axiom of Choice. It may seem that our proof of $(1) \Rightarrow (2)$ does apply the Axiom of Choice implicitly. In fact, we do not need the full power of AOC here, all we need is something called 'Axiom of dependent choice', which I will not go into details here.
\end{remark}

Now for modules:

\begin{definition}[Ascending / Descending Chain Condition (A.C.C. / D.C.C), Noetherian / Artinian]
    Let $M$ be an $R$-module and $\Sigma$ be the set of all submodules of $M$.
    \begin{enumerate}
        \item If $(\Sigma, \subset)$ satisfies the chain condition, we call $M$ satisfies the \textbf{ascending chain condition (a.c.c.)} and $M$ a \textbf{Noetherian $R$-module}
        \item If $(\Sigma, \supset)$ satisfies the chain condition, we call $M$ satisfies the \textbf{descending chain condition (d.c.c.)} and $M$ an \textbf{Artinian $R$-module}
    \end{enumerate}

    A ring $R$ is called \textbf{Noetherian / Artinian} if and only if it is Noetherian / Artinian as an $R$-module, namely the ideals satisfy a.c.c. / d.c.c.
\end{definition}

Let's see a few examples of (non-)Noetherian and (non-)Artinian modules / rings:

\begin{example}[Noetherian but not Artinian]
    \begin{enumerate}
        \item $\mathbb{Z}$ as a $\mathbb{Z}$-module(= abelian group). It is Noetherian as larger ideal (= subgroup) will be generated by smaller number. It is not Artinian as $(p) \supset (p^2) \supset (p^3) \supset \cdots$ clearly does not stabilize.
        \item $k[X]$ (or $k[[X]]$) as a ring. It is Noetherian because we will see later that PID's are Noetherian (or apply Hilbert Basis Theorem introduced later). It is not Artinian similar to $\mathbb{Z}$.
    \end{enumerate}
\end{example}

\begin{example}[Artinian but not Noetherian]
    Let $G$ be the subgroup of $\mathbb{Q} / \mathbb{Z}$ consisting of all elements whose denominator is a power of $p$ where $p$ is a fixed prime. It is Artinian but not Noetherian as a $\mathbb{Z}$-module. Let's consider the subgroups of $\mathbb{Q} / \mathbb{Z}$. For any subgroup $H$ of $G$, suppose the elements in $H$ are all in lowest term, let $a / p^n \in H$ be the element such that $n$ is maximal. Then $H = G_n$ where $G_n = \left\lbrace a / p^k: k \le n \right\rbrace$ (For this part we need to quotient out $\mathbb{Z}$ from $\mathbb{Q}$). So any subgroup of $G$ has the form $G_n$ for some $n \ge 0$. Note that $G_0 \subset G_1 \subset G_2 \subset \cdots$ is a non-stable ascending chain of subgroups, so it is not Noetherian but clearly Artinian.
\end{example}

\begin{example}[Neither Noetherian nor Artinian]
    \begin{enumerate}
        \item Let $k[X_1, X_2, \cdots]$ be the polynomial ring with infinite indeterminates. Then it is not Noetherian as a ring due to the ascending chain of ideals $(X_1) \subset (X_1, X_2) \subset \cdots$. It is neither Artinian as a ring due to the descending chain $(X_1) \supset (X_1^2) \supset \cdots$
        \item If we do not quotient out $\mathbb{Z}$ in the previous example, then we do not get artinian property, we could well have an ascending chain $(a) \supset (a^2) \supset (a^3) \supset \cdots$ where $a \in \mathbb{Z}$.
    \end{enumerate}
\end{example}

\begin{example}[Both Noetherian and Artinian]
    Clearly a field is both Noetherian and Artinian as a ring, any finite group is both Noetherian and Artinian. More generally, $k[X] / (X^n)$ or $\mathbb{Z} / (n)$ are both Noetherian and Artinian as a ring and an abelian group respectively.
\end{example}

The above has shown that there is no relation between Noetherian modules and Artinian modules in general. However, as will be shown later (really late, we are still a few chapters away from that), any Artinian ring will be Noetherian.

In the examples above, the Noetherian and Artinian properties are established by verifying the definitions, which is not desirable since we have to consider all possible chain conditions. It would be nice if we can obtain Noetherian / Artinian properties from those known Noetherian / Artinian modules:

\begin{proposition}
    Let $M', M, M''$ be $R$-modules such that there is an exact sequence:
    $$0 \rightarrow M' \rightarrow M \xrightarrow{p} M'' \rightarrow 0$$
    Then we have:
    \begin{enumerate}
        \item $M$ is Noetherian $\Leftrightarrow$ $M', M''$ is Noetherian.
        \item $M$ is Artinian $\Leftrightarrow$ $M', M''$ is Artinian.
    \end{enumerate}
\end{proposition}

\begin{proof}
    "$\Rightarrow$": Let $\left\lbrace M_n'' \right\rbrace_n$ be an ascending (resp. descending) chain in $M''$ and $\left\lbrace M_n' \right\rbrace_n$ be an ascending (resp. descending chain) in $M'$, then $\left\lbrace p ^{-1}(M_n'') \right\rbrace_n, \left\lbrace M_n' \right\rbrace_n$ are also ascending (resp. descending) chains in $M$ and they stabilize only if $\left\lbrace M_n'' \right\rbrace_n, \left\lbrace M_n' \right\rbrace_n$ stablize respectively.

    "$\Leftarrow$": Let $\left\lbrace M_n \right\rbrace_n$ be an ascending (resp. descending) chain in $M$. Let $M_n' = M_n \cap M', M_n'' = p(M_n)$, then they are ascending (resp. descending) chains in $M', M''$ respectively. By Noetherian (resp. Artinian), they stablize. WLOG, they both stablize after some $n_0$. Then consider the commutative diagram with exact rows for $n \ge n_0$ (for descending chains, reverse the vertical arrows):
    % https://q.uiver.app/#q=WzAsMTAsWzAsMCwiMCJdLFswLDEsIjAiXSxbMSwwLCJNX24nIl0sWzIsMCwiTV9uIl0sWzMsMCwiTV9uJyciXSxbNCwwLCIwIl0sWzEsMSwiTV97biArIDF9JyJdLFsyLDEsIk1fe24gKyAxfSciXSxbMywxLCJNX3tuICsgMX0nJyJdLFs0LDEsIjAiXSxbMCwyXSxbMiwzXSxbMyw0XSxbNCw1XSxbMSw2XSxbNiw3XSxbNyw4XSxbOCw5XSxbMiw2LCIiLDEseyJzdHlsZSI6eyJ0YWlsIjp7Im5hbWUiOiJob29rIiwic2lkZSI6InRvcCJ9fX1dLFs0LDgsIiIsMSx7InN0eWxlIjp7InRhaWwiOnsibmFtZSI6Imhvb2siLCJzaWRlIjoidG9wIn19fV0sWzMsNywiIiwxLHsic3R5bGUiOnsidGFpbCI6eyJuYW1lIjoiaG9vayIsInNpZGUiOiJ0b3AifX19XV0=
    \[\begin{tikzcd}
        0 & {M_n'} & {M_n} & {M_n''} & 0 \\
        0 & {M_{n + 1}'} & {M_{n + 1}'} & {M_{n + 1}''} & 0
        \arrow[from=1-1, to=1-2]
        \arrow[from=1-2, to=1-3]
        \arrow[hook, from=1-2, to=2-2]
        \arrow[from=1-3, to=1-4]
        \arrow[hook, from=1-3, to=2-3]
        \arrow[from=1-4, to=1-5]
        \arrow[hook, from=1-4, to=2-4]
        \arrow[from=2-1, to=2-2]
        \arrow[from=2-2, to=2-3]
        \arrow[from=2-3, to=2-4]
        \arrow[from=2-4, to=2-5]
    \end{tikzcd}\]
    Apply the five lemma to conclude.
\end{proof}

\begin{corollary}[Submodule / Quotient Inherits Chain Conditions] \label{cor:sub-quo-inherit-chain}
    Submodules / Quotients of Noetherian (resp. Artinian) modules are Noetherian (resp. Artinian)
\end{corollary}

\begin{corollary}
    Let $\left\lbrace M_i \right\rbrace_{i = 1}^n$ be Noetherian (resp. Artinian) $R$-modules, then $\bigoplus\limits_{i = 1}^{n} M_i$ is also Noetherian (resp. Artinian).
\end{corollary}

\begin{proof}
    Prove by induction and considering the split exact sequence:
    $$0 \rightarrow \bigoplus\limits_{i = 1}^{k - 1} M_i \rightarrow \bigoplus\limits_{i = 1}^{k} M_i \rightarrow M_k \rightarrow 0$$
\end{proof}

\begin{corollary} [Finitely Generated Modules over Noetherian / Artinian Rings are Noetherian / Artinian]
    Let $R$ be a Noetherian (resp. Artinian) ring, and $M$ be a finitely generated $R$-module, then $M$ is also Noetherian (resp. Artinian)
\end{corollary}

\begin{proof}
    Simply note that $M = A^n / N$ for some $n$ and submodule $N$ of $A^n$.
\end{proof}

However, chain conditions \textbf{as rings} are not easy to inherit.

\begin{example}[Subring of Noetherian / Artinian ring need not be Noetherian / Artinian]
    Consider the previous example $k[X_1, X_2, \cdots]$, it is a subring of $k(X_1, \cdots)$, which is a field and hence is Noetherian and Artinian.
\end{example}

And the reasons are that:
\begin{enumerate}
    \item A subring of $R$ is not a submodule of $R$ as $R$-module (the ideals are).
    \item When considering the chain conditions of a subring $S$ of $R$, we consider $S$ as an $S$-module, not an $R$-module.
\end{enumerate}

Following the second reason, we now consider a ring $R$ as modules over different rings: It is clear that $R$ is a module over any subring $S$ of $R$, and we have:

\begin{proposition}
    Let $R$ be a ring, $S_1, S_2$ be two subrings of $R$ such that $S_1 \subset S_2$. Then $R$ is Noetherian (resp. Artinian) as $S_1$-module implies $R$ is Noetherian (resp. Artinian) as $S_2$-module.
\end{proposition}

\begin{proof}
    Simply note that any $S_2$-submodule of $R$ will also be an $S_1$-module.
\end{proof}

In this sense, Noetherian / Artinian as a ring is the weakest form of chain condition of the ring as an algebra (opps, if you know what that mean).

On the other hand, any quotient ring $S$ of $R$ is an $R$-module, moreover, the $R$-submodules are exactly the $S$-submodules (ideals), so we have:

\begin{proposition}
    Let $R$ be a ring, $S$ be a quotient ring of $R$. Then if $R$ is Noetherian (resp. Artinian), so is $S$(as a ring).
\end{proposition}

\begin{proof}
    Simply note that $S$ is Noetherian as an $R$-module by Cor \ref{cor:sub-quo-inherit-chain}
\end{proof}

It should be noted that even in Noetherian rings, there could be ascending chains of arbitrary length: Despite all chains are finite, the supremum of the length could be infinite. But the example is complicated so we have to postpone it to later chapters. Here we concern the special case where the length is bounded above:

\begin{definition}[Chain, Composite Series, Simple Module]
    Let $M$ be an $R$-module, then we call:
    $$M = M_0 \supsetneq M_1 \supsetneq M_2 \supsetneq \cdots \supsetneq M_n = 0$$
    a \textbf{chain} of submodules (note that it has to start with $0$ and end in $M$ and strictly descending). We call $n$ the \textbf{length} of the chain.
    
    A chain is called a \textbf{composite series} if its length cannot be increased by inserting submodules into the chain, namely it is maximal. 

    An $R$-module $M$ is called \textbf{simple} if it has only trivial submodules $0$ and $M$ (could be the same). Then a chain of length $n$ is a composite series if and only if $M_{i - 1} / M_i$ is simple for all $i = 1, \cdots, n$.
\end{definition}

\begin{definition}[Finite Length]
    Let $M$ be an $R$-module and $l(M)$ be the least length of composite series of $M$ and denote $l(M) = +\infty$ if there is no composite series. If $l(M) \lt \infty$, then $M$ is called \textbf{an $R$-module of finite length}, and $l(M)$ is called the \textbf{length} of $M$.
\end{definition}

Clearly, if $l(M) = +\infty$, then any chain can be lengthened, so there could be chains of arbitrarily large length. What is interesting is that when $l(M) \lt \infty$, then every composite series is of length $l(M)$, similar to the case of vector space:

\begin{proposition}
    Let $M$ be an $R$-module of finite length. Then:
    \begin{enumerate}
        \item Every chain of $M$ has length $\le l(M)$.
        \item A chain is a composite series if and only if it has length $l(M)$.
        \item Every chain of $M$ can be extended to a composite series.
    \end{enumerate}
\end{proposition}

We first prove a lemma:

\begin{lemma}
    Let $N \subsetneq M$ be $R$-modules and $M$ is of finite length, then $l(N) \lt l(M)$ and the equality holds if and only if $N = M$.
\end{lemma}

\begin{proof}
    Let $\left\lbrace M_i \right\rbrace_{i = 0}^n$ be a composite series of $M$ of least length (namely $n = l(M)$). Consider $N_i = M_i \cap N$, then $\left\lbrace N_i \right\rbrace_{i}$ is a chain of $N$. Note that for arbitrary $i = 0, \cdots, n - 1$, $N_i / N_{i + 1} \subset M_i / M_{i + 1}$ (by considering the kernel of $N_i \hookrightarrow M_i \rightarrow M_i / M_{i + 1}$), so $N_i / N_{i + 1} = 0$ or $N_i / N_{i + 1} = M_i / M_{i + 1}$ since $M_i / M_{i + 1}$ is simple. Remove the redundant terms from $N_i$'s, we obtain a composite series of $N$ of length $\le n$, so $l(N) \le l(M)$. If $N = M$, clearly the equality holds. On the other hand, if $l(N) = l(M)$, then we must have $N_i / N_{i + 1} = M_i / M_{i + 1}$ for each $i = 0, \cdots, n - 1$. Note that $N_n = M_n = 0$, this shows that $N_{n - 1} = M_{n - 1}$. Argue inductively by applying the five lemma to the commutative diagram below ith exact rows:

    % https://q.uiver.app/#q=WzAsMTAsWzAsMCwiMCJdLFsxLDAsIk5fe2kgKyAxfSJdLFswLDEsIjAiXSxbMSwxLCJNX3tpICsgMX0iXSxbMiwwLCJOX2kiXSxbMiwxLCJNX2kiXSxbMywwLCJOX3tpfSAvIE5fe2kgKyAxfSJdLFszLDEsIk1faSAvIE1fe2kgKyAxfSJdLFs0LDAsIjAiXSxbNCwxLCIwIl0sWzAsMV0sWzIsM10sWzEsNF0sWzMsNV0sWzQsNl0sWzUsN10sWzYsOF0sWzcsOV0sWzEsMywiIiwxLHsic3R5bGUiOnsidGFpbCI6eyJuYW1lIjoiaG9vayIsInNpZGUiOiJ0b3AifX19XSxbNCw1LCIiLDEseyJzdHlsZSI6eyJ0YWlsIjp7Im5hbWUiOiJob29rIiwic2lkZSI6InRvcCJ9fX1dLFs2LDcsIiIsMSx7InN0eWxlIjp7InRhaWwiOnsibmFtZSI6Imhvb2siLCJzaWRlIjoidG9wIn19fV1d
    \[\begin{tikzcd}
        0 & {N_{i + 1}} & {N_i} & {N_{i} / N_{i + 1}} & 0 \\
        0 & {M_{i + 1}} & {M_i} & {M_i / M_{i + 1}} & 0
        \arrow[from=1-1, to=1-2]
        \arrow[from=1-2, to=1-3]
        \arrow[hook, from=1-2, to=2-2]
        \arrow[from=1-3, to=1-4]
        \arrow[hook, from=1-3, to=2-3]
        \arrow[from=1-4, to=1-5]
        \arrow[hook, from=1-4, to=2-4]
        \arrow[from=2-1, to=2-2]
        \arrow[from=2-2, to=2-3]
        \arrow[from=2-3, to=2-4]
        \arrow[from=2-4, to=2-5]
    \end{tikzcd}\]
\end{proof}

Now for the proof of the proposition:

\begin{proof}[Proof of the Proposition]
    Take arbitrary chain $\left\lbrace M_i \right\rbrace_{i = 0}^{n}$ of $M$. Note that $l(M) = l(M_0) \gt l(M_1) \gt \cdots \gt l(M_{n}) = 0$ by or previous lemma, which implies $l(M) \ge n$. So all chains have length $\le l(M)$.

    It follows easily that a chain is a composite series if and only if it has length $l(M)$.

    Now given arbitrary chain, if it is not a composite series, then we can always extend its length by definition. After finite steps we obtain a chain of length $l(M)$, which is a composite series.
\end{proof}

As a corollary:

\begin{corollary}[Finite Length $\Leftrightarrow$ Noetherian + Artinian]
    Let $M$ be an $R$-module. Then $M$ is of finite length if and only if $M$ is both Noetherian and Artinian.
\end{corollary}

\begin{proof}
    "if": Note that a chain $\left\lbrace M_i \right\rbrace$ of $M$ is a composite series if and only if $M_{i + 1}$ is a maximal proper submodule of $M_i$ for each $i$. So we can construct a composite series of $M$ as follows: Let $M_0 = M$. For each $i \ge 0$, if $M_i = 0$, pick $M_{i + 1} = 0$; otherwise pick $M_{i + 1}$ a maximal proper submodule of $M$. $M_{i + 1}$ exists since $M$ and therefore all its submodules are Noetherian (Prop \ref{prop:equiv-chain}). Then we obtain a descending chain $\left\lbrace M_i \right\rbrace$, which stabilizes by Artinian. Finally, by our construction, it can only stabilize at some $M_{n_0} = 0$, which gives us a composite series of $M$.

    "only if": By the previous proposition, the length of chains are bounded, so any ascending or descending chains must stabilize.
\end{proof}

We close the section with the special case of vector spaces:

\begin{proposition}
    Let $k$ be a field and $V$ be a $k$-vector space. Then TFAE:
    \begin{enumerate}
        \item $V$ is of finite dimension;
        \item $V$ is of finite length;
        \item $V$ is Noetherian;
        \item $V$ is Artinian.
    \end{enumerate}
    Moreover, if these conditions are satisfied, then $l(V) = \mathrm{dim}(V)$
\end{proposition}

\begin{proof}
    $(1) \Rightarrow (2)$: Clear. (Note that $k$ itself is a simple $k$-vector space)

    $(2) \Rightarrow (3), (2) \Rightarrow (4)$: By the previous Corollary.

    $(3) \Rightarrow (1), (4) \Rightarrow (1)$: Suppose $(1)$ is false, then there are linearly independent vectors $x_1, x_2, \cdots$. Then $\left\lbrace (x_1, \cdots, x_n) \right\rbrace_{n = 1}^{\infty}$ and $\left\lbrace (x_n, \cdots) \right\rbrace_{n = 1}^{\infty}$ will be non-stable ascending chain and descending chain respectively.
\end{proof}

For now our proof is basically templated for both ascending and descending chain conditions. Next we are going to talk about some properties of Noetherian and Artinian modules that they do not share.

\section{Noetherian}

We start with an alternate definition of Noetherian modules:

\begin{proposition}\label{prop:equiv-Noetherian}
    Let $M$ be an $R$-module, then TFAE:
    \begin{enumerate}
        \item $M$ is Noetherian
        \item Every submodule of $M$ is finitely generated.
    \end{enumerate}
\end{proposition}

\begin{proof}
    $(1) \Rightarrow (2)$: Suppose there is a submodule $M'$ of $M$ not finitely generated. Then we can take $x_1, x_2, \cdots \in M'$ such that $x_i \notin (x_1, \cdots, x_{i - 1})$ for all $i$. Then $\left\lbrace (x_1, \cdots, x_i) \right\rbrace_{i = 1}^{\infty}$ will be a non-stable ascending chain.

    $(2) \Rightarrow (1)$: Let $M_1 \subset M_2 \subset \cdots$ be an ascending chain of submodules of $M$. Then $M_{\infty} = \bigcup\limits_{i = 1}^{\infty} M_i$ is also a submodule of $M$. By our assumption $M_{\infty}$ is generated by some $x_1, \cdots, x_r$. But then there is $n_0$ such that $x_i \in M_{n}$ for all $i$ and $n \ge n_0$. It follows that $M_n = M_{\infty}$ for all $n \ge n_0$.
\end{proof}

With the above proposition in mind, we can replace submodules in some defining properties of Noetherian modules with finitely generated ones, for example:

\begin{proposition}
    Let $M$ be an $R$-module. Then $M$ is a Noetherian module if and only if every non-empty set of finitely generated submodules has a maximal element (or equivalently every ascending chain of finitely generated submodules stabilize).
\end{proposition}

\begin{proof}
    The "or equivalently" part is by Prop \ref{prop:equiv-chain} applied to the finitely generated submodules.

    The "only if" part is by Prop \ref{prop:equiv-chain}. For the "if" part, we apply Prop \ref{prop:equiv-Noetherian} and prove every submodule is finitely generated. The proof is similar as in Prop \ref{prop:equiv-Noetherian}: Suppose otherwise, there is submodule $M'$ not finitely generated. Then we can pick $x_1, x_2, \cdots$ so that $x_i \notin (x_1, \cdots, x_{i - 1})$. By our assumption, $\left\lbrace (x_1, \cdots, x_i) \right\rbrace_{i = 1}^{\infty}$ has a maximal element, or equivalently, stabilizes, a contradiction.
\end{proof}

As a corollary, we can verify a ring is Noetherian by verifying whether every ideal is finitely generated.

\section{Artinian}

\end{document}