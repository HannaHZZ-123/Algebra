\documentclass{note-eng}

\title{Ring Extension}
\author{Jingyi Long}

\begin{document}

\maketitle
\tableofcontents

In this chapter, we discuss the extension of rings $R \rightarrow S$ and various conditions involved.

\newpage

\section{Algebra}

\begin{definition}[Algebra]
    Let $R$ be a ring, an $R$-algebra is a pair $(S, f: R \rightarrow S)$ where $S$ is a ring and $f$ a ring homomorphism.
\end{definition}

\begin{remark}
    \begin{enumerate}
        \item A nonempty $k$-algebra is a ring containing $k$ as a subring. (Note that a ring homomorphism must preserve $1$ and thus is nonzero)
        \item Any ring is a $\mathbb{Z}$-algebra
    \end{enumerate}
\end{remark}

By the definition, $S$ is an $R$-module by restriction of scalars.

\begin{notation}
    Let $f: R \rightarrow S$ be an algebra, $x_1, \cdots, x_n \in S$:
    \begin{enumerate}
        \item Denote $\sum\limits_{i = 1}^{n} Rx_i$ the submodule of $S$ generated by $\left\lbrace x_i \right\rbrace$'s. This notation is compatible with our notation in modules.
        \item Denote $R[x_1, \cdots, x_n]$ the subring of $S$ generated by $\left\lbrace x_i \right\rbrace$'s, namely all the polynomials with coefficients in $k$ and indeterminates replaced by $x_i$'s.
        \item If $R, S$ are fields, denote $R(x_1, \cdots, x_n)$ the field of quotients of $R[x_1, \cdots, x_n]$.
    \end{enumerate}
\end{notation}

\begin{definition}[Finiteness]
    Let $(S, R \rightarrow S)$ be an $R$-algebra.
    \begin{enumerate}
        \item Module-finite / Finite: If $S$ is finitely generated as a $R$-module, namely there are $x_1, \cdots, x_n \in S$ for some $n$ such that $S = \sum\limits_{i = 1}^{n} R x_i$, we say that $S$ is \textbf{module-finite} / \textbf{finite} over $R$, or $S$ is a \textbf{finite} $R$-algebra. The homomorphism $f: R \rightarrow S$ is said to be \textbf{finite}.
        \item Ring-finite / Finitely-generated: If there are $x_1, \cdots, x_n \in S$ for some $n$ such that $S = R[x_1, \cdots, x_n]$, then we say that $S$ is \textbf{ring-finite} / \textbf{of finite type} over $R$, or $S$ is a \textbf{finitely-generated} $S$-algebra. The homomorphism $f: R \rightarrow S$ is said to be \textbf{of finite type}.
        \item Finitely-generated Field Extension: If $R, S$ are fields, and there are $x_1, \cdots, x_n \in S$ such that $S = R(x_1, \cdots, x_n)$. Then $S$ is said to be a \textbf{finitely-generated field extension} of $R$
    \end{enumerate}

    A finitely generated $R$-algebra is also called an \textbf{affine} $R$-algebra. Equivalently, an affine $R$-algebra is a quotient of $R[X_1, \cdots, X_n]$ for some $n \ge 0$
\end{definition}

\begin{remark}
    When discussing finiteness, we may assume that $R$ is a subring of $S$ (or equivalently $f$ is injective) because the action of $R$ on $S$ is only relevant to the image of $R$ in $S$(which is a subring), not $R$ itself.
\end{remark}

The finiteness defined above clearly gets weaker and weaker:

\begin{proposition}[Module-finite $\supset$ Ring-finite $\supset$ Field-finite]
    Let $(S, f: R \rightarrow S)$ be an $R$-algebra, then:
    \begin{enumerate}
        \item If $S$ is module-finite over $R$, then $S$ is ring-finite over $R$
        \item If $S, R$ are fields, and $S$ is ring-finite over $R$, then $S$ is a finitely-generated field extension of $R$
    \end{enumerate}
\end{proposition}

But not vice versa:

\begin{example}[Ring Finite but not Module Finite]
    $R[X]$ is ring-finite over $R$, but is not module finite over $R$, for $X^n$ cannot be expressed as a linear combination of $1, X, \cdots, X^{n - 1}$.
\end{example}

\begin{example} [Field Finite but not Ring Finite]
    If $k$ is a field, $k(X)$ is a finitely-generated field extension of $k$, but is not ring-finite over $k$. See the proposition below
\end{example}

\begin{proposition} \label{prop:rational-not-ring-finite}
    Let $k$ be a field, then $k(X)$ is not ring-finite over $k$. In fact, it is not even ring-finite over $k[X]$.
\end{proposition}

\begin{proof}
    Suppose $k(X)$ is ring-finite over $k[X]$. Let $P_1, \cdots, P_m$ be the irreducible components of all the generators of $k(X)$ over $k[X]$. Then all $F \in K(X)$ can be represented by $G(X) / H(X)$ where $G, H \in K[X]$ and $H$ is of the form $P_1^{n_1}(X) P_2^{n_2}(X) \cdots P_m^{n_m}(X)$. But there are infinite many irreducible polynomials in $K[X]$, so there is some $1 / P(X)$ that cannot be represented as above. (Note that $k[X]$ is a UFD, then elements in $k(X)$ has a unique representation)
\end{proof}

Finiteness is transitive:

\begin{proposition}[Finiteness is transitive]
    Let $K \subset R \subset S$ be rings, then:
    \begin{enumerate}
        \item If $R$ is module-finite over $K$ and $S$ is module-finite over $R$, then $S$ is module-finite over $K$
        \item If $R$ is ring-finite over $K$ and $S$ is ring-finite over $R$, then $S$ is ring-finite over $K$
        \item If $K, R, S$ are fields, $R$ is a finitely-generated field extension of $K$ and $S$ is a finitely-generated field extension of $R$, then $S$ is a finitely-generated field extension of $K$.
    \end{enumerate}
\end{proposition}

\begin{proof}
    It should be easy to check:
    \begin{enumerate}
        \item Suppose $R = \sum\limits_{i = 1}^{n} K u_i$ and $S = \sum\limits_{i = 1}^{m} R v_i$, then $S = \sum\limits_{i = 1}^{n} \sum\limits_{j = 1}^{m} Ku_iv_j$
        \item Suppose $R = K[u_1, \cdots, u_n]$ and $S = R[v_1, \cdots, v_m]$, then 
        $$S = K[u_1, \cdots, u_n, v_1, \cdots, v_m]$$
        \item Suppose $R = K(u_1, \cdots, u_n)$ and $S = R(v_1, \cdots, v_m)$, then
        $$S = K(u_1, \cdots, u_n, v_1, \cdots, v_m)$$
    \end{enumerate}
\end{proof}

However, finiteness does not descend to sub-structures easily:

\begin{example}[Submodule of finite module may not be finite]
    Let $M = K = k[\underline{X}]$ where $\underline{X}$ is infinite. Then $M$ is finite over $K$ (it is generated by $1$). Let $N$ be the submodule of $M$ consists of all polynomials with zero constant term. Then $N$ is not finite over $K$, since $X_{n + 1}$ cannot be expressed as linear combination of $X_1, \cdots, X_n$ with coefficients in $K$ and any finite number of elements of $K$ at best generate some $k[X_1, \cdots, X_n]$
\end{example}

\begin{example}[Subalgebra of a finitely generated algebra may not be finitely generated] \label{exp:subalgebra-not-fg}
    Let $K$ be a field, consider $R = K[X, Y]$, which is of finite type over $K$. Then consider the subring $K[X^iY^j: i \ge 1, i \gt j]$, it is not of finite type over $K$, since $X^nY^{n - 1}$ cannot be represented as a polynomial of $\left\lbrace X^iY^j: 1 \le i \lt n, i \gt j \right\rbrace$ with coefficients in $K$ (the term in the polynomial that contributes to $X^nY^{n - 1}$ must be a product of at least two $X^iY^j$, but then the difference in the degree of $X, Y$ will be at least $2$) Note that this example can be generalized to $\left\lbrace X^iY^j: 1 \le i \lt n, \frac{j}{i} \lt c \right\rbrace$ for some constant $c$
\end{example}

\iffalse
\begin{example} \label{exp:finiteness-not-descend}
    \begin{enumerate}
        \item Subring of a ring $R$ finite over $K$ may bot be finite over $K$: \TODO
        \item Subfield of finitely-generated field extension may not be a finitely-generated field extension. \TODO 
    \end{enumerate}
\end{example}
\fi

Aside from the conditions above, there is another way of viewing the extensions:

\begin{definition}[Integral, Algebraic]
    Let $R$ be a subring of $S$. An element $x$ of $S$ is said to be \textbf{integral} over $R$ if there is a \textit{monic} polynomial $F = X^n + a_1X^{n - 1} + \cdots + a_n \in R[X]$ such that $F(x) = 0$. When $R, S$ are fields, $x$ is also called \textbf{algebraic} over $R$.
    
    If every element in $S$ is integral over $R$, then we say that $S$ is \textbf{integral} over $R$. When $R, S$ are fields, $S$ is also said to be \textbf{algebraic} over $R$, or an \textbf{algebraic field extension} of $R$
\end{definition}

It is relatively easy to test whether an element is integral over a subring:

\begin{proposition}\label{prop:integral-test}
    Let $R$ be a subring of $S$, $x \in S$. Then the following conditions are equivalent:
    \begin{enumerate}
        \item $x$ is integral over $R$
        \item $R[x]$ is finite over $R$
        \item There is a subring $S'$ of $S$ containing $R[x]$ that is finite over $S$.
        \item There is a faithful $R[x]$-module that is finitely generated as an $R$-module.
    \end{enumerate}
\end{proposition}

Note that $(2) \Leftrightarrow (3)$ is non-trivial, see Exp \ref{exp:subalgebra-not-fg}.

\begin{proof}
    "$(1) \Rightarrow (2)$": Suppose $x^n + a_1x^{n - 1} + \cdots + a_nx^n = 0$, then it is easy to verify that $R[x] = \sum\limits_{i = 0}^{n - 1} Rx^i$

    "$(2) \Rightarrow (3)$": Take $R' = K[x]$

    "$(3) \Rightarrow (4)$": Take $S'$ the faithful $R[x]$ module.

    "$(4) \Rightarrow (1)$": Let $M$ be the faithful $R[x]$-module. Consider $\varphi_x: M \rightarrow M: m \mapsto xm$. By Hamilton-Cayley, $F(\varphi_x) = \varphi_{F(x)} = 0$ for some monic polynomial $F \in R[X]$. Then we must have $F(x) = 0$ as $M$ is faithful.
\end{proof}

An immediate implication is:

\begin{corollary}[Finiteness implies Integral]\label{cor:finite-integral}
    Let $R$ be a subring of $S$, and $S$ is finite over $R$, then $S$ is integral over $R$.
\end{corollary}

\begin{corollary} [Integral elements form a subring] \label{cor:integral-subring}
    Let $R$ be a subring of $S$, then the set of elements of $S$ integral over $R$ forms a subring $R'$ of $S$
\end{corollary}

\begin{proof}
    Take arbitrary $a, b \in S$, integral over $R$. Then $b$ is also integral over $R[a]$, which implies $R[a, b]$ finite over $R$. Since $a \pm b, ab \in R[a, b]$, they are all integral over $K$ by the previous proposition.
\end{proof}

If $R, S$ are fields, from the previous corollary, we only know the algebraic elements form a subring, is it a field? The answer is yes, and it follows from the following lemma and the fact that subring of a field is a domain.

\begin{lemma}[Integral Extension of Fields + Domain = Field]\label{lem:integral-domain-field}
    Let $k$ be a subring of $R$ and $k$ be a field, if $R$ is integral over $k$ and $R$ is a domain, then $R$ is a field.
\end{lemma}

\begin{proof}
    Take $x \in R$, $x \ne 0$, then it satisfies some monic polynomial $x^n + a_1x^{n - 1} + \cdots + a_0 = 0$ where $a_i \in k$. We may assume $a_0 \ne 0$ (otherwise we can replace the polynomial by $x^{n - 1} + a_1 x^{n - 2} + \cdots + a_1$, this is where we use the fact that $R$ is a domain) Then we have:
    $$x(x^{n - 1} + a_1 x^{n - 2} + \cdots + a_1) = -a_0$$
    which proves that $x$ is invertible (Note that we need $k$ to be a field since $-a_0$ is not necessarily $1$)
\end{proof}

It should be noted that the converse of Cor \ref{cor:finite-integral} is not true: if $R$ is integral over $K$, $R$ may not be finite over $K$.

\begin{example} [Integral / Algebraic but not finite] \label{exp:integral-not-finite}
    Let's consider the case when $k = \mathbb{Q}$ and $R$ is the set of algebraic numbers (It is known that $R \subset \mathbb{C}$, then $R$ is a field by the previous lemma). Then $R$ is integral over $k$ but is not finite over $k$, since $\sqrt{p}$'s where $p$ is prime are independent.
\end{example}

The key point of the above example is that $R$ may contain infinite many generators. In fact, if $R$ is ring-finite over $K$, integral does imply finite.

\begin{lemma} [Integral + Ring-finite = Module-finite]\label{lem:integral-finite}
    Let $R$ be a subring of $S$, and $S$ is ring-finite over $R$, integral over $R$, then $S$ is module-finite over $K$. In particular, any algebraic extension of finite type is a finite extension.
\end{lemma}

\begin{proof}
    Assume $S = R[x_1, \cdots, x_n]$ for $x_1, \cdots, x_n \in S$. Since $R[x_1, \cdots, x_n] = R[x_1, \cdots, x_{n - 1}][x_n]$ and finiteness is transitive, we can argue by induction, ISTS to show the case of $n = 1$, but it follows from Prop \ref{prop:integral-test}
\end{proof}

The above lemma clearly has a special case for fields. But in fact more is true, we shall prove later that \textbf{for fields, ring-finite = module finite}.

Integrability is also transitive:

\begin{proposition}[Transitivity of Integrability]
    Let $K \subset R \subset S$ be rings, $R$ is integral over $K$ and $S$ is integral over $R$, then $S$ is integral over $K$.
\end{proposition}

\begin{proof}
    Take arbitrary $x \in S$, we have $x^n + a_1x^{n - 1} + \cdots + a_0 = 0$ for some $a_i \in R$. Then consider $K[a_1, \cdots, a_n]$, by Lem \ref{lem:integral-finite}, it is finite over $K$. However, by Prop \ref{prop:integral-test}, $K[a_1, \cdots, a_n, x]$ is finite over $K[a_1, \cdots, a_n]$, by transitivity of finiteness, $K[a_1, \cdots, a_n, x]$ is finite over $K$. By Prop \ref{prop:integral-test} again, $x$ is integral over $K$.
\end{proof}

Now let's focus on the special case of field extension. Let's first consider the case of the field extension $k \hookrightarrow k(v)$ where $k$ is a subfield of field $l$ and $v \in l$:

\begin{enumerate}
    \item If $v$ is algebraic over $k$, then $k(v) = k[v]$: Note that $k[v]$ is a domain as $k[v] \subset l$. Then $k[v]$ is a field by Lem \ref{lem:integral-domain-field} and thus $k[v] = k(v)$, then $K[v]$ is module-finite over $K$ by Prop \ref{prop:integral-test}
    \item If $v$ is not algebraic over $k$, also consider the evaluation map $k[X] \rightarrow k[v]$, then we have $k[X] \cong k[v] \Rightarrow k(v) \cong k(X)$. Then $k(X)$ is not ring-finite over $k$. (It is not even ring-finite over $k[X]$, as proved in Prop \ref{prop:rational-not-ring-finite})
\end{enumerate}

As we can see, $K(v)$ is either not ring-finite (when $v$ is not algebraic) or module-finite (when $v$ is algebraic). This suggests ring-finiteness and module finiteness are the same:

\begin{theorem}[Zariski's Lemma, Ring-finite = Module-finite for field extension]
    If a field $l$ is ring-finite over a subfield $k$, then $l$ is module-finite over $k$
\end{theorem}

\begin{proof}
    Suppose $l = k[v_1, \cdots, v_n]$. Induct on $n$, the case of $n = 1$ is solved by previous discussions. Now consider the inclusion
    $$k \hookrightarrow k(v_1) \hookrightarrow k(v_1)[v_2, \cdots, v_n] = l$$
    (We have to use $k(v_1)$ instead of $k[v_1]$ since the IH only deals with field extension) By induction, the second extension is finite and thus algebraic. If $v_1$ is algebraic over $k$, then the proof is complete by previous argument. So we may assume $v_1$ not algebraic over $k$ and $k(v_1) \cong k(X)$
    
    Since $l$ is algebraic over $k(v_1)$, for any $i = 1, \cdots, n$, we have:
    \begin{equation}\label{eq:Zariski}
        v_i^{n_i} + a_{i, 1}v_i^{n_i - 1} + \cdots + a_{i, n_i} = 0
    \end{equation}
    where $a_{i, j} \in k(v_1)$. Then take $a$ to be the product of all denominators of $a_{i, j}$. Multiply both sides of Eq \ref{eq:Zariski} by $a^{n_i}$, we have:
    $$(av_i)^{n_i} + a_{i, 1}a (av_i)^{n_i - 1} + \cdots + a_{i, n_i}a^{n_i} = 0$$
    Namely $av_i$ is integral over $k[v_1]$. By Cor \ref{cor:integral-subring}, for every $x \in l = k[v_1,\ cdots, v_n]$, there is $n_x$ such that $a^{n_x} x$ is integral over $k[v_1]$. In particular, take $x \in k(v_1)$. However, note that $k(v_1) \cong k(X)$, where $a$ corresponds to polynomial $A$. Take some irreducible polynomial $P$ that does not divide $A$ (always possible since there are infinite irreducible polynomials, the proof is similar to the proof that there are inifinite prime numbers), then $A^n / P(X)$ is not integral over $k[X]$ for all possible $n$, a contradiction.
\end{proof}

\iffalse

Turns out that the case for $n = 1$ is general. For finite extension of field $K \hookrightarrow L$ (or equivalently by Zariski's Lemma, an extension that is ring-finite), $L$ can always be represented by $K(z)$ for some $z \in L$.

\begin{theorem}
    (Theorem of the Primitive Element) Let $K$ be a field of characteristic zero, $L$ a finite extension of $K$, then there is $z \in L$ such that $L = K(z)$
\end{theorem}

\begin{proof}
    \TODO See Problem 6.31 in Fulton's book
\end{proof}

\fi

\begin{corollary}
    Let $k \hookrightarrow l$ be a field extension, and $k$ is algebraically closed, then if $x \in l$ is algebraic over $k$, we must have $x \in l$
\end{corollary}

\begin{proof}
    $k[x]$ is a field by Lem \ref{lem:integral-domain-field} and Cor \ref{cor:integral-subring}. But then $k[x]$ is finite over $k$ by Zariski's Lemma. Then $k[x] = l$ by algebraically closed.
\end{proof}

We now know that a field extension that is ring-finite must be finite, and thus algebraic. What about field extension that is not ring-finite? Example \ref{exp:integral-not-finite} shows that such field extension can be algebraic. Clearly it can also be transcendent (not algebraic) like $K(\underline{X})$ where $\underline{X}$ is infinite. In fact, we can always factor the extension into a transcendent one and an algebraic one. The proof is similar to finding the Hamel basis of arbitrary vector space, where linearly dependence is replaced by \textit{algebraically dependence}.

\begin{definition}[Algebraically dependent]
    Let $K \hookrightarrow L$ be a field extension, $v_1, \cdots, v_n \in L$, if there is a nonzero polynomial $P \in K[X_1, \cdots, X_n]$ such that $P(v_1, \cdots, v_n) = 0$, then we say that $v_1, \cdots, v_n$ are \textbf{algebraically dependent}, otherwise, we say that $v_1, \cdots, v_n$ are \textbf{algebraically independent}.

    More generally, we say a set of element $\left\lbrace v_{\lambda} \right\rbrace_{\lambda \in \Lambda}$ is algebraically dependent if there is a finite subset that is algebraically dependent
\end{definition}

\begin{theorem}[Existence of Transcendence Basis]
    Let $K \hookrightarrow L$ be a field extension, then there is a set of element $S = \left\lbrace v_\lambda \right\rbrace_{\lambda \in \Lambda}$ of $L$ such that:
    \begin{enumerate}
        \item $S$ is algebraically independent.
        \item $L$ is algebraic over $K(S)$
    \end{enumerate}
    We call $S$ a \textbf{transcendence basis} of $L$ over $K$. Any two transcendence basis has the same cardinality, we call this cardinality the \textbf{transcendence degree} of $L$ over $K$ and denote it as $\mathrm{tr}.\mathrm{deg}_KL$
\end{theorem}

\begin{iproof}
    Note that $L$ is algebraic over $K(S)$ if and only if for any $x \in L \setminus K$, $S \cup \left\lbrace x \right\rbrace$ is algebraically dependent. We will state the idea of the proof and left the details to the readers. It is very similar to the proof of existence of Hamel basis for vector space:
    \begin{enumerate}
        \item Show that $S$ is a transcendence basis if and only if $S$ is a maximal algebraically independent set, or $S$ is a minimal set such that $L$ is algebraic over $K(S)$.
        \item Use Zorn's lemma to take a maximal algebraically independent set. This proves the existence of transcendence basis. 
        \item More generally, use Zorn's lemma to show that any algebraically independent set can be extended to a transcendence basis and any set $S$ with $L$ algebraic over $K(S)$ contains a transcendence basis.
        \item The last part is a bit tricky, to prove that two transcendence basis has the same cardinality, we follow the steps:
        \begin{enumerate}
            \item Show that if $L$ has a finite transcendence basis $A$ over $K$, then any transcendence basis $B$ is finite and has the same cardinality: Suppose $\left\lvert A \right\rvert \lt \left\lvert B \right\rvert$, each time add one element of $B$ into $A$ and delete at least one element from $A$ to make it a basis again(possible by part 4). Repeat until all elements of $A$ are replaced, use minimality to contradict.
            \item Show that if $L$ has an infinite transcendence basis $A$ over $K$, then any transcendence basis $B$ is infinite and has the same cardinality: Use the previous part to show that $B$ is infinite. Each element $a$ in $A$ is algebraic over $K(B_a)$ for some \textbf{finite} subset $B_a \subset B$. Then $A$ is finite over $K(B')$ where $B' = \bigcup\limits_{a \in A} B_a$, by transitivity of algebraic, $L$ is algebraic over $K(B')$. By minimality, $B' = B$. But this implies $\left\lvert B \right\rvert \le \left\lvert A \right\rvert$ (check). Close the proof by symmetry and Bernstein's Theorem.
        \end{enumerate}
    \end{enumerate}

    For a full proof, see Problem 6.33 of Fulton's Algebraic Curve (it does not cover the infinite case though).
\end{iproof}

\begin{corollary}\label{cor:field-extension-decompose}
    Let $K \hookrightarrow L$ be a field extension, then it can factor into a pure transcendence one and an algebraic one:$K \hookrightarrow K(\underline{X}) \hookrightarrow L$ where $\underline{X}$ is a transcendence basis of $L$ over $K$
\end{corollary}

When the extension is of finite type, we can actually replace $k(\underline{X})$ with $k[\underline{X}]$ (of course, we are talking about ring extension instead of field extension). This is an implication of the Noether Normalization Lemma. 

\begin{theorem}[Noether Normalization Lemma]
    Let $R$ be a finitely generated $k$-algebra where $k$ is a field. Then there are $x_1, \cdots, x_d \in R$ algeraically independent, such that $R$ is integral over (or equivalently finite over as the extension is ring-finite) $k[x_1, \cdots, x_d]$
\end{theorem}

\begin{proof}
    We assume $k$ has characteristic $0$ (so $k$ is infinite). For the case of characteristic $p$, \TODO.

    Let $y_1, \cdots, y_n \in R$ such that $R = k[y_1, \cdots, y_n]$. By reordering if necessary, we may assume $y_1, \cdots, y_d$ are algebraically independent and $y_{d + 1}, \cdots, y_n$ are integral over $k[y_1, \cdots, y_d]$. If $n = d$, then we are done. Otherwise we have $f(y_1, \cdots, y_n) = 0$ for some non-zero polynomial $f$. We want to make it into a monic polynomial of $y_n$. Let $F$ be the homogeneous part of $f$ of highest degree. Find $\lambda_1, \cdots, \lambda_{n - 1}$ such that $F(\lambda_1, \cdots, \lambda_{n - 1}, 1) \ne 0$ (always possible when $k$ is inifite). Then take $y_i' = y_i - \lambda_i y_n$ for $i = 1, \cdots, n - 1$ and we have:
    $$f(y_1' + \lambda_1 y_n, \cdots, y_{n - 1}' + \lambda_{n - 1} y_{n}, y_n) = 0$$
    and $f$ will be a polynomial of $y_n$ whose highest term is $cy_n^m$ where $c \ne 0$ and $m$ is the degree of $F$. This implies $y_n$ integral over $k[y_1', \cdots, y_{n - 1}']$. Continue the process.
\end{proof}

\end{document}
